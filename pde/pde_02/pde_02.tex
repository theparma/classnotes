\documentclass[12pt,fleqn]{article}
\setlength{\parindent}{0pt}
\usepackage{graphicx}
\usepackage{cancel}
\usepackage{listings}
\usepackage[latin5]{inputenc}
\usepackage{color}
\setlength{\parskip}{8pt}
\setlength{\parsep}{0pt}
\setlength{\headsep}{0pt}
\setlength{\topskip}{0pt}
\setlength{\topmargin}{0pt}
\setlength{\topsep}{0pt}
\setlength{\partopsep}{0pt}
\setlength{\mathindent}{0cm}

\begin{document}
PDE - Ders 2

Denklem soyle idi

\[ a(x,y)u_x + b(x,y)u_y + c(x,y)u = f(x,y) \]

Bu denklem homojen degil, cunku denklemin sol tarafi $f \ne 0$, homojenlik
icin nihai test tabii ki $u=0$ koyunca $0=0$ cikip cikmayacagi.

Cozum icin kullandigimiz fikir neydi? Kordinat sistemini transform etmek,
ki 

$\bigg(x,y\bigg) \to \bigg(\xi(x,y), \eta(x,y)\bigg)$

olsun. Bu degisimi yaparken oyle bir degisim ariyoruz ki boylece transform
edilmis PDE'miz cozulmesi kolay bir hale gelsin. 

Amac

Denklemi sadece tek bir bagimsiz degiskene gore turevi icerecek sekilde
yazmak 

\[ w_\xi + h(\xi,\eta)w  = R(\xi,\eta) \]

\[ w \equiv u \bigg( x(\xi,\eta),y(\xi,\eta) \bigg)  \]

\[ 
J = 
\left|\begin{array}{rr}
\xi_x & \xi_y \\
\eta_x & \eta_y
\end{array}\right| =
\xi_x \eta_y - \xi_y \eta_x \ne 0
 \]

Turev Transformasyonu

\[ \frac{\partial }{\partial x} = 
\frac{\partial \xi}{\partial x}  \partial_\xi + 
\frac{\partial \eta}{\partial x}\partial_\eta
\]

\[ \frac{\partial }{\partial y} = 
\frac{\partial \xi}{\partial y}  \partial_\xi + 
\frac{\partial \eta}{\partial y}\partial_\eta
\]

Bunu yapinca PDE su hale gelecek

\[ a \bigg[ a\xi_x + b\xi_y \bigg]w_\xi + 
a \bigg[ a\eta_x + b\eta_y \bigg]w_\eta + 
cw = f
 \]

Simdi $\eta$ kordinatini oyle bir sekilde secmek istiyoruz ki ustteki sag
koseli parantez icindeki terimler yokolsun. Boylece PDE'yi $\xi$ bir ODE'ye
indirgemis oluruz. Bunu elde edince entegrasyon kolaylca yapilabilir. 

\[ a \eta_x + b\eta_y = 0 \]

\[=> \frac{\eta_x}{\eta_y} = - \frac{b(x,y)}{a(x,y)} \]


Yani ikinci kordinat sistemi her ne ise, onun ortaya cikardigi kismi
turevlerinin birbirine orani katsayi fonksiyonlarinin oraninin negativine
esit.  


















\end{document}
