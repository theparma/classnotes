\documentclass[12pt,fleqn]{article}
\setlength{\parindent}{0pt}
\usepackage{graphicx}
\usepackage{cancel}
\usepackage{listings}
\usepackage[latin5]{inputenc}
\usepackage{color}
\setlength{\parskip}{8pt}
\setlength{\parsep}{0pt}
\setlength{\headsep}{0pt}
\setlength{\topskip}{0pt}
\setlength{\topmargin}{0pt}
\setlength{\topsep}{0pt}
\setlength{\partopsep}{0pt}
\setlength{\mathindent}{0cm}

\begin{document}
Analitik PDE - Ders 1

Dersin kullanacagi ana kitap L. C. Evans'in Kismi Turevsel Denklemler
(partial differential equations -PDE-) kitabi olacak. Bir sonraki ders icin
okuma odevi soyle:

1. Sf. 1-13'teki ozet

2. Alt bolum 2.1 sf. 17-19

3. Bolum 3 sf. 91-115 arasini tamamen. 

PDE'leri incelerken cogunlukla onlarin temsil ettigi fiziksel fenomenleri
de inceleyecegiz. Mesela transportasyon (transport) denklemleri, ki

\[ \partial_t u + \vec{b} \cdot \vec{\nabla} u = 0 \]

Ustteki ifadede gradyan operatoru var, bu bilindigi gibi

\[ \vec{\nabla} = \bigg( 
\frac{\partial }{\partial x_1},.., 
\frac{\partial }{\partial x_n},
\bigg)
\]

$\vec{b}$ icinde sabitler olan bir vektor olabilir

\[ \vec{b} = (b_1,...,b_n)
 \]

Bu denklem 1. derece PDE'lerin ozel bir durumudur bu arada. 1. derece
PDE'ler 

\[ F(x, u(x), Du(x)) = 0 \]

seklindedir. $D$ notasyonu Evans'in gradyan icin kullandigi notasyon,
alissak iyi olur. Yani

\[ Du = \nabla u \]

Ustteki gibi denklemlere bakacagiz, cozumlerini gorecegiz. Mesela ustteki
denklem direk ODE yaklasimi ile cozulebiliyor. Bu hakikaten ilginc bir sey,
yani ustteki gibi genis bir PDE kategorisi, $n$ tane degisken icerebilen
turden denklemler ODE'lere indirgenerek cozulebiliyor. Bu yonteme
``karakteristikler metotu (method of characteristics)'' ismi veriliyor.

Bu arada

\[ F(x, u(x), Du(x)) = 0 \]

formu oldukca geneldir, cok genel gayri lineer, normal (ordinary)
diferansiyel denklemin formudur. Bir ornek

\[ |\nabla u|^2 = n^2(x) \]

denklemidir, ki bu denklem dalga denkleminde dalgalarin uc noktalarini
(wavefront) incelerken ortaya cikar. 

Lineer PDE

Lineer PDE'lerin ornekleri mesela isi denklemi (heat equation), ya da
yayilma / difuzyon (diffusion) denklemi.

\[ \frac{\partial u}{\partial t} = 
\nabla  u
\]

Bir diger ornek dalga denklemi (wave equation)

\[ \frac{1}{c^2} \frac{\partial ^2}{\partial t^2}u = \nabla u\]

yayilmanin hizi $c$ sabiti olarak gosteriliyor. 

Shrodinger denklemi ise soyle

\[ i \frac{\partial }{\partial t}\psi  = -\nabla \psi \]

Isi denklemiyle alakali onemli bir nokta denge (equilibrium)
noktasidir. Denklem uzun zaman zarfi baglaminda incelendiginde bir denge
noktasina eristigi gorulecektir. Ki bu bizi Laplaca denklemine goturur. 

\[ \nabla u = 0 \]

Ya da daha genel olarak 

\[ \nabla u = f \]

ise bu denkleme Poisson denklemi adi veriliyor. 

Tum bu denklemler aslinda pek cok uygulama alaninda ortaya cikan, cok genis
belli basli bazi kategorilerin prototipidirler. Mesela Isi denklemine
parabolik (parabolic) kismi denklem kategorisi deniyor. Dalga denklemi
hiperbolik (hyperbolic) kategorisi, Schrodinger ise dagilan (dispursive)
PDE kategorisi olarak aniliyor. Laplaca, Poisson denklemlerine elliptik
(elliptic) kategorisi ismi veriliyor. 

Tum bu denklemleri incelerken onlari temsil ettikleri daha genis
kategorinin ornekleri olarak gorecegiz. 

Simdi PDE'lerin ortaya ciktigi cok basit bir ornegi gorelim. 

Elimizde bir $\Omega$ bolgesi / alani oldugunu hayal edelim, ki $\Omega \in
\Re^n$ olsun, 
yani alttaki resimde cizdigimiz $\Re^2$.

Yine diyelim ki bu $\Omega$ bolgesinin etrafinda bir tur siviyla kapli, ve
bu sivi hareket ediyor. Bu hareketi sabit bir hiz vektor alani olarak
gosteriyoruz. 

\includegraphics[height=2cm]{pde_01.png}

















\end{document}
