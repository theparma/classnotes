\documentclass[12pt,fleqn]{article}
\setlength{\parindent}{0pt}
\usepackage{graphicx}
\usepackage{cancel}
\usepackage{listings}
\usepackage[latin5]{inputenc}
\setlength{\parskip}{8pt}
\setlength{\parsep}{0pt}
\setlength{\headsep}{0pt}
\setlength{\topskip}{0pt}
\setlength{\topmargin}{0pt}
\setlength{\topsep}{0pt}
\setlength{\partopsep}{0pt}
\setlength{\mathindent}{0cm}

\begin{document}
PDE - Ders 1

Konumuz Kismi Turevsel Denklemler (partial differential equtions -PDE-). Bu
dersin on gerekliliklerinden en onemlisi normal diferansiyel denklemlerdir
(ordinary differential equtions -ODE-), cunku pek cok PDE'yi cozmenin
teknigi onlari bir ODE sistemine indirgemekten geciyor. Yani PDE cozmek
icin ODE cozme tekniklerini de bilmek gerekiyor. Bir diger gerekli bilgi
Lineer Cebir dersi.

Bu dersin ana amaci, bir muhendislik dersi olarak, denklem cozmek, ve pek
cok denklemin cikis noktasi fiziksel problemler. Mesela sicaklik yayilmasi
(heat diffusion), dalga hareketi (wave motion), titresen hucre zari
(vibrating membrane) gibi. Fakat PDE kavrami finansta bile ortaya cikabilen
bir kavram, mesela Black-Sholes denklemlerinde oldugu gibi. 

Yani dersimiz cok teori odakli olmayacak, bazi ispatlardan bahsedecegiz,
ama onun haricinde teori uzerinde fazla durmayacagiz. 

PDE nedir? Ilk once ODE tanimindan baslayalim. 

\[ y = y(x) \]

\[ \frac{dy}{dx} = y \]

Baslangic sartlari 

\[ y(0) = y_0 \]

Cozum 

\[ y = y_0e^x \]

Bu bir ODE cunku sadece bir tane bagimsiz degisken var ($x$), ve bir tane
bagimli degisken var ($y$). 

PDE ise icinde kismi turevleri, ve bir veya {\em birden fazla} bagimsiz
degiskeni barindiran bir denklemdir.

Eger gunes etrafindaki yorungeleri temsil etmek istiyorsaniz gezegenleri
boyutsuz parcaciklar gibi kabul ederek ODE'ler ile temsil etmek yeterli
olabilir, ama diger problemlerde daha fazla bagimsiz degisken gerekecegi
icin ODE yetmez, mesela zaman, cismin 3D uzaydaki boyutlari gibi.

Mesela bir PDE

\[ u = u(x,y) \]

Cogunlukla problem taniminin ilk basinda fonksiyonel iliskiyi hemen
gostermek iyi olur, mesela ustte bagimsiz degiskenler $x,y$, ve $u$ bu iki
degiskene bagimli. Devam edelim PDE soyle olsun

\[ \frac{\partial^3 u}{\partial x^3} + 
cos(y)\frac{\partial u}{\partial y} + 3 = 0
\]

Bir PDE problemine cogunlukla ek olarak sinir kosullari (boundary condition
-BC-) ve baslangic kosullari (initial conditions -IC-) eklemek de gerekir. 

Kismi Turev nedir? 

\[ u = u(x_1, x_2,...,x_n) \]

\[ 
\frac{\partial u}{\partial x_i} = 
\lim_{\Delta x_i \to 0} 
\frac{
u(x_1,..,x_i+\Delta x_i,x_{i+1},...,x_n) - u(x_1,..,x_i,x_{i+1},...,x_n)}
{\Delta x_i}  \]

Yani bir fonksiyonun kismi turevini almak istedigimiz degisken haricinde
tum diger degiskenlerinin sabit tutuldugu bir durum. 

Ornek

\[ u = x_1^2 + x_1sin(x_2) \]

\[ 
\frac{\partial u}{\partial x_1} = 2x_1 + sin(x_2)
 \]

\[ 
\frac{\partial u}{\partial x_2} = x_1 cos(x_2)
 \]

Notasyon

Cogunlukla kismi turevler 3 farkli sekilde gosteriliyor. 

\[ \frac{\partial u}{\partial x} \equiv u_x \equiv \partial_x u \]

Ustte soldaki tanimi gorduk, bazen ortadaki de tercih edilebiliyor, ya da
bazen en sagdaki. 

PDE Derecesi

Bir PDE'nin derecesi, o denklemdeki kismi turevlerin en yuksek dereceli
olanin derecesi neyse o'dur.

Mesela

\[ u_{xxx} + u_y = 5 \]

derecesi 3. Ayni zamanda bu lineer ve homojen olmayan (inhomogeneous) bir
PDE. Bu son iki kavrami birazdan tanimlayacagim. 

Ornek 

\[ (u_{xx})^2 + u_xu_y = u \]

Bu 2. derece. Bu bazi insanlarin kafasini karistiriyor, cunku $u_{xx}$'in
karesi var. Bu ayni zamanda homojen, ve gayri lineer. Bu dersteki cogu PDE
lineer olacak. 

Lineer ve gayri lineerlikten bahsetmisken, sunu ekleyelim. 

\includegraphics[height=3cm]{pde_01.png}

Simdi diyelim ki bir girdi (input) fonksiyonu $I(t)$ bir isleme giriyor
($L$ operatoru) ve cikti (output) olarak $R(t)$ cikiyor. Yani sistem

\[ R = L \ I \]

Bir lineer sistemde eger girdiyi iki ile carparsaniz, cikti da iki katina
cikar. O zaman kurallar

\begin{enumerate}

   \item $L(\alpha I) = \alpha \ L(I)$, ki $\alpha$ bir sabit.

   \item $L(I_1 + I_2) = L(I_1) + L(I_2)$, ki buna ust uste eklenebilme
     (superposition) prensibi deniyor. Bu prensibi bu dersteki cogu PDE'yi
     cozmek icin kullanacagiz. Bir lineer sistem varsa cogu zaman arka
     planda bir yerlerde ust uste eklenebilme prensibi geziniyordur. 

\end{enumerate}

Diyelim ki PDE'nizi soyle yazdiniz

\[ Lu = f(\vec{x}) \]

Burada $u$ bagimli degisken, $\vec{x}$ bir vektor, $\vec{x} \in \Re ^n$, ve
bu vektorun icinde birden fazla degisken var, bu degiskenlerin hepsi
bagimsiz.

\[ 
\vec{x} = 
\left(\begin{array}{r}
x_1,\\
.. \\
x_n
\end{array}\right)
 \]

Bu denkleme benzer bir diger denklem lineer cebirdeki $A\vec{x} = \vec{b}$
denklemidir.  PDE sisteminde de cevabini aradigimiz, lineer cebir
sisteminde ``$A$ ile carpilip $b$ sonucunu verecek $\vec{x}$ hangisidir?''
sorusuna benzer bir sekilde ``$L$ operatoru uygulanip $f(\vec{x})$
sonucunu verecek $u$ hangisidir?'' sorusudur.

Bu analojiden devam etmek gerekirse, belli bir noktada $u$'nun icinde
oldugu ``fonksiyon uzayi'' hakkinda dusunmemiz gerekebilir, $\vec{x}$'in
icinde oldugu $\Re^n$ uzayi gibi. Lineer cebir durumunda operatorun
ozelliklerine bakilir, mesela `` $b$'nin icinde oldugu ve $A$ operatoru
uygulanip hic sonuc alinamayacak uzayin belli kisimlari var
midir?'' gibi sorularla ugrasilabilir, bunlar $A$'nin ``ulasamadigi
yerlerdir'' vs. PDE'deki $L$ operatoru icin de benzer sorular sorulabilir. 

Yani lineer cebirle pek cok kavram PDE dunyasina benziyor, orada vektor
uzayi var, burada fonksiyon uzayi var. 

\end{document}
