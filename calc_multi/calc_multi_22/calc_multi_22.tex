\documentclass[12pt,fleqn]{article}
\setlength{\parindent}{0pt}
\usepackage{graphicx}
\usepackage{cancel}
\usepackage{listings}
\usepackage[latin5]{inputenc}
\usepackage{color}
\setlength{\parskip}{8pt}
\setlength{\parsep}{0pt}
\setlength{\headsep}{0pt}
\setlength{\topskip}{0pt}
\setlength{\topmargin}{0pt}
\setlength{\topsep}{0pt}
\setlength{\partopsep}{0pt}
\setlength{\mathindent}{0cm}

\begin{document}
Cok Degiskenli Calculus - Ders 22

Diyelim ki genel bir vektor alanimiz var, ve bu alan muhafazakar degil, biz
kapali bir egri icinde cizgi entegrali hesaplamak istiyoruz, bu entegral
sifir olmayacak. Bu hesabi iki metotla yapabiliriz. Ya direk entegral
hesabi yapariz, ya da Green'in Teoremi adli bir teknigi kullaniriz. 

Green'in Teorisi

Eger $C$ egrisi $R$ alanini tanimlayan kapali bir egriyse, bu egri
uzerindeki hareket saat yonunun tersi ise, ve elimde egrinin ve icindeki
alanin her noktasinda tanimli ve turevi alinabilir bir vektor alani
$\vec{F}$ var ise, o zaman 

\[ \oint_C \vec{F} \cdot d\vec{r} = \int \int_R curl \ \vec{F} \ dA \]

Diger bir formda 

\[ \oint_C M dx + N dy = \int \int_R (N_x - M_y) \ dA \]

Ustteki her iki formulde de esitligin iki tarafinin birbirinden ne kadar
farkli hesaplar olduguna dikkat edelim. Mesela bir ustteki formulun sol
tarafi bir cizgi entegrali, bir (mesela) $t$ uzerinden birbirine bagli iki
$x,y$ degiskeninin, bir sekilde birbirine bagli sekilde degisen $x,y$
degiskenleri uzerinden bir entegral hesabi bu. Ama ayni formulde esitligin
sag tarafi egrinin ``icindeki'' bolgede, bu bolgedeki her nokta icin
gecerli. Bu bolgede $x,y$ birbirinden bagimsiz. 

Peki niye saat yonu tersi dedik? Bu aslinda herhangi bir yon olabilir, ama
bu yon $M_x-N_y$ ile uyumlu, eger saat yonu olursa, ona gore bir $M_x,N_y$
ifadesinin kullanilmasi gerekir. 

Ornek

$C$ egrisi $(2,0)$ noktasinda cizilmis 1 yaricapli bir cember
olsun. Gidisat saat yonu tersinde. 

\includegraphics[height=2cm]{22_1.png}

\[ \oint_C ye^{-x} dx + (\frac{1}{2}x^2 - e^{-x})dy \]

Direk olarak

\[ x = 2 + cos\theta, dx = -sin\theta \ d\theta \]

\[ y = sin\theta, dy = cos\theta \ d\theta \]

Sonra ustteki iki satirdaki formulleri alip entegrale koyardim, ve 0 ile
$\pi$ sinirlari arasinda entegrali hesaplardim. 

Tum bu islemler yerine Green'in Teorisini kullanalim. Yani cift entegral
hesabini yapalim. $M,N$ nedir? 

\[ \oint_C 
\underbrace{ye^{-x}}_{M} dx + 
\underbrace{(\frac{1}{2}x^2 - e^{-x})}_{N}dy \]

Cift entegral o zaman soyle

\[ \int \int_R (N_x - N_y) \ dA =
\]

$R$ nedir? Cemberin icindeki her seydir (cizgili alan). 

\includegraphics[height=2cm]{22_2.png}

\[=  \int \int_R  \bigg( x+ e^{-x} \bigg) - e^{-x} dA \]

\[=  \int \int_R  x dA \]

Peki bu cift entegrali nasil hesaplayacagiz? Tum cebirsel islemi
yapabiliriz, ya da, bir kisayol dusunelim. 

\[ = Alan(R) \ \bar{x} \]

Cunku cift entegral bir alan hesabi yapiyor. $\bar{x}$ ise $R$ alaninin
yatay eksendeki kutle merkezi, ki ornegimiz icin bu merkez 2
noktasinda. Alan zaten $\pi$, sonuc

\[ = 2\pi \]

Bu arada $\bar{x}$'in tanimi

\[ \bar{x} = \frac{1}{Alan} \int \int x \ dA \]


Simdi Green Teorisinin niye isledigini, yani ispatini gorelim. 

Ozel Durum 

$curl \ \vec{F} = 0$

Bu durumda $\vec{F}$ muhafazakar. 

\[ \oint_C \vec{F} \cdot d\vec{r} = \int \int_R curl \ \vec{F} \ dA \]

Ama curl sifir ise, o zaman sifirin entegrali alinir, bu da sifirdir. 

\[ = 0 \]

Bu ayni zamanda eger kapali egri icindeki her noktada $curl \ \vec{F} = 0$ ise, o
zaman $\oint_C \vec{F} \cdot d\vec{r} = 0$ oldugunun ispatidir, bu da $\vec{F}$'in muhafazakar oldugunun
ispatidir. 

Problem Set 8 Problem 2 icin Green Teorisi uygulanamiyor, cunku $C$
icinde orijin var. 

Ispat

Sunu ispatlamaya ugrasiyoruz

\[ \oint_C M dx + N dy = \int \int_R (N_x - M_y) \ dA \]

Isimizi kolaylastirmak icin daha basit bir ifadeyi ispat edelim. Ustteki
formulun ozel bir hali

\[ \oint_C M dx  = \int \int_R -M_y \ dA \]

bu durumda $N=0$, ve vektor alaninda sadece $x$ bileseni var. 

Peki bu ozel sart niye yeterli? Iddia o ki eger bu daha basit ifadeyi ispat
edebilirsem, sadece $y$ bilesenin oldugu diger sarti da ispat edebilirim,
sonra bu iki ozel sarti toplarsam genel sarti elde ederim. 

Diger ozel sart

\[ \oint_C N dy  = \int \int_R N_x \ dA \]

Bir problem daha var, eger egri cetrefil ise cift entegrali olusturmak zor
olabilir. 

\includegraphics[height=2cm]{22_3.png}

O zaman daha basit egrilerle ise baslamak daha iyi. 
















\end{document}
