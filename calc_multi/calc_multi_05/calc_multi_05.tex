\documentclass[12pt,fleqn]{article}
\setlength{\parindent}{0pt}
\usepackage{graphicx}
\usepackage{listings}
\usepackage[latin5]{inputenc}
\setlength{\parskip}{8pt}
\setlength{\parsep}{0pt}
\setlength{\headsep}{0pt}
\setlength{\topskip}{0pt}
\setlength{\topmargin}{0pt}
\setlength{\topsep}{0pt}
\setlength{\partopsep}{0pt}
\setlength{\mathindent}{0cm}

\begin{document}
MIT OCW Cok Degiskenli Calculus - Ders 5

Bir cizginin formulunu iki duzlemin kesisimi olarak gorduk, fakat bu
sekilde bir tanim cogunlukla bir cizgiyi tanimlamak icin en rahat / uygun
yol degildir, cunku elinizde bazi denklemler var, bunlari cozmekle ugrasmak
lazim, vs. 

Soyle bir yontem daha iyi olmaz mi? Cizgi uzerinde bir nokta hayal edelim,
ve bu noktanin, her zaman adiminda, cizgimizin oldugu yerlerden gectigini
dusunelim. Bu tur denklemlere parametrik denklem ismi veriliyor. 

Ornek

Cizgi uzerinde iki nokta verelim. 

\[ Q_0 = (-1,2,2) \]

\[ Q_1 = (1,3,-1) \]

Guzel, bu iki nokta var ama otekilerini nasil tanimlariz? Bu iki noktatinin
arasinda, sonrasinda, oncesinde olan tum noktalar da cizgiye dahildir. 

\includegraphics[height=4cm]{5_1.png}

Zaman araliklarini oyle dusunelim ki zaman indeksi sifir ($t=0$)
noktasinda, cizgi $Q_0$ uzerinde, tek birim adim atildiginda ($t=1$)  $Q_1$
uzerinde, gibi. O zaman yarim birim zamanda tam iki nokta ortasinda. 

Boylece cizgiyi temsil etmenin yolu onu $t$ bazinda hareket eden noktanin
gectigi yerler olarak tanimlamak. Bu temsilin en basit hali eger hareket
sabit hizda olursa olur. 

$t$ anindaki pozisyon $Q(t)$ nedir? 

Sorunun cevabini soyle vermeye baslayabiliriz: $\vec{Q_0Q(t)}$ vektoru
$\vec{Q_0Q_1}$ birbiriyle ortalidir. Bu oranti neye esittir? 

Bu oran $t$'ye esittir. O zaman

\[ \vec{Q_0Q(t)} = t \ \vec{Q_0Q_1}  \]

\includegraphics[height=4cm]{5_2.png}

O zaman iddia ediyorum ki bu formulu kullanarak ornegimizdeki hareket eden
noktanin yer formulunu bulabilirim.

\[ \vec{Q_0Q(t)} = t \ <2,1,-3>  \]

Simdi cizgi uzerinde hareket eden noktanin formulu $Q(t)$'yi su sekilde
temsil edelim

\[ Q(t) = <x(t),y(t),z(t)> \]

O zaman

\[ x(t) + 1 = t \ 2 \]

\[ y(t) - 2 = t \]

\[ z(t) - 2 = -3t \]

Usttekiler, alttaki su formun acilimindan ibaret aslinda

\[ Q(t) = Q_0 + t \ \vec{Q_0Q_1}  \]

Ustteki uc formul bu derste gordugumuz ilk parametrik cizgi
formulu. Formulun parcalari olan $x(t),y(t),z(t)$ sadece $t$'nin
fonksiyonudurlar, ve hep $t$ ile bir katsayinin carpimi + bir sabit
formundadirlar. $t$'nin katsayilari cizgi uzerindeki vektor hakkinda bilgi
verir, ve sabitler ise $t=0$ aninda nerede oldugumuzu gosteren baslangic
degerleridirler. 

Uygulama - Bir Duzlem ile Kesisme

Duzlem $x+2y+4z=7$. Cizgi biraz onceki formul olsun. Kesisme var midir, var
ise nerededir? 

Once su soruyu soralim kendimize. $x+2y+4z=7$ duzlemine gore, 
$Q_0 =
(-1,2,2)$ ve $Q_1 = (1,3,-1)$ noktalari duzlemin

\begin{enumerate}
   \item Ayni tarafinda
   \item Farkli taraflarinda
   \item Bir tanesi duzlem uzerinde
   \item Karar veremiyorum
\end{enumerate}

Cevaplayin. 

$Q_0$ ve $Q_1$ noktalarini duzlem formulunun sol tarafina sokariz. $Q_0$
icin sonuc $>7$, duzlem uzerinde degil, $Q_1$ icin sonuc $<7$, yine duzlem
uzerinde degil. Peki noktalar duzlemin hangi tarafinda? Ters tarafinda,
cunku biri $<7$, oteki $>7$ sonuc verdi. Bir duzlem uzayi iki yari-parcaya
(halfspace) ayirir ve noktalar bu ayri parcalardadirlar. Dogru cevap 2. 

Uygulamamizda cevaplanmayan bir soru daha var. Kesisme noktasi neresi?
$Q(t)$ nedir? Soyle

\[ x(t) + 2y(t) + 4z(t) \]

\[ = (-1+2t) + 2(2+t) + 4(2-3t) \]

Basitlestirelim

\[ = -8t + 11 \]

Bu formulu $7$ ile karsilastiralim cunku $Q(t)$ nin duzlem uzerinde oldugu
an $-8t + 11 = 7$ oldugu andir. Cebirsel olarak $t$'yi elde edebiliriz,
sonuc $t=1/2$. Bu degeri $Q(t)$'ye koyarsak

\[ Q(\frac{1}{2}) = (0,\frac{5}{2},\frac{1}{2}) \]

Kesisim noktasi esitligin sagindaki degerdir.

Yani eger cizginin parametrik denklemini biliyorsak, onu duzlem formulune
sokariz, ve kesisimin hangi noktada oldugunu hemen hesaplayabiliriz. 

Simdiye kadar gorduklerimizden parametrik denklemlerin cizgileri temsil
etmek icin iyi bir yontem olduklari belli olmustur herhalde. Bunun otesinde
parametrik denklemler uzaydaki herhangi bir egri (curve), herhangi bir
gidisati, yolu (trajectory) temsil etme kabiliyetine de sahiptir. 

Genel baglamda soylemek gerekirse, parametrik denklemleri uzayda icinde ya
duzlem uzerindeki herhangi (arbitrary) bir hareketi temsil etmek icin
kullanabiliriz.

Cycloid (Yuvarlanma Egrisi)

$a$ yari capindaki bir tekerlek yerde (x-ekseninde) donerek ilerliyor, $P$ bu
tekerlegin dis ceperinde (rim) bir nokta, baslangic noktasi 0 uzerinde. Ne
olur? Daha detayli olarak sormak gerekirse $P$ noktasinin hareketini
$t$'nin bir fonksiyonu $x(t),y(t)$ olarak hesaplayabilir miyiz?

\includegraphics[height=4cm]{5_3.png}

[x-ekseni biraz saga yatik cikmis ama bu video kamerasinin acisi
yuzunden]. Bu ornegi bir bisikletin tekerligine takilmis bir isigin,
bisiklet gece giderken ortaya cikartabilecegi goruntuyu dusunurek te hayal
edebiliriz. Yani hem donus hareketi var, hem de yatay olarak duz bir gidis
hareketi var.

Bu $P$ noktasinin gidis yolunu hesaplamak icin tekerlegin ne kadar hizli
dondugu onemli mi? Hayir degil. Yavas ta hizli da dondursek, $P$ ayni
noktalardan gececektir. 

Bu problemde en onemli faktor zaman degil, mesafe, tekerlegin ne kadar
mesafe katettigi. Ya da daha bile iyisi, mesafe ile donus birbirine
baglantili olduguna gore, ve problemdeki en cetrefil, girift olus donme
(rotation) olduguna icin, belki de tekerlegin ne kadar dondugunu gosteren
bir aci degeri, bu buyuklugu kullanirsak belki daha faydali olacak. Pek cok
degisik temsil yontemi olabilir, fakat aciya gore parametrize edersek en
temiz formulu elde etmek mumkun olur. O zaman $x(t),y(t)$ yerine
$x(\theta),y(\theta)$ kullanalim. 











\end{document}
