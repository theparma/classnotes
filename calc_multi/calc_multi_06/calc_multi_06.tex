\documentclass[12pt,fleqn]{article}
\setlength{\parindent}{0pt}
\usepackage{graphicx}
\usepackage{listings}
\usepackage[latin5]{inputenc}
\setlength{\parskip}{8pt}
\setlength{\parsep}{0pt}
\setlength{\headsep}{0pt}
\setlength{\topskip}{0pt}
\setlength{\topmargin}{0pt}
\setlength{\topsep}{0pt}
\setlength{\partopsep}{0pt}
\setlength{\mathindent}{0cm}

\begin{document}
MIT OCW Cok Degiskenli Calculus - Ders 6

Bir onceki derste cycloid konusunu isledik. 

\includegraphics[height=4cm]{6_1.png}

Hareket eden bir noktanin pozisyonu

\[ (x(t), y(t), z(t)) \]

Bu noktayi takip etmenin diger yollarindan biri onu pozisyonu vektoru
olarak gormek, ki bu vektorun bilesenleri noktanin kordinatlari. 

\[ \vec{r}(t) = <x(t),y(t),z(t)> \]

Vektor orijin (baslangic) noktasindan gelinen noktayi isaret eden bir
vektor (resimde $\vec{OP}$). 

Onceki dersteki cycloid problemimiz icin, tekerlek yaricapi 1 olsun ve
birim hizda ilerliyor olalim, ki boylece aci $\theta$ ve zaman ayni sey
haline gelsin

\[ \vec{r}(t) = <t-sin(t), 1-cos(t) \]

Tamam. Simdi, noktanin pozisyonunu zaman acisindan bildigimize gore, onun
degisimini inceleyebiliriz, mesela hizina, ivmesine bakabiliriz. Ilk once
hiza bakabiliriz. Fakat, aslinda, hizdan daha iyisini hesaplayabiliriz. Hiz
tek bir sayidir sadece, ama eger su icinde GPS olan satafatli spor
arabalarindan birine sahip degilseniz, size hizinizin ``hangi yonde''
oldugunu soylemez. Sadece gittiginiz yonde ne kadar hizli oldugunuzu
soyler. 

O zaman biz hizimizi hesaplarken, hem yonu, hem hizi ayni anda goze
alabiliriz. Bu demektir ki vektor kavrami tekrar isimize yarayacak. 









\end{document}
