\documentclass[12pt,fleqn]{article}
\setlength{\parindent}{0pt}
\usepackage{graphicx}
\usepackage{cancel}
\usepackage{listings}
\usepackage[latin5]{inputenc}
\usepackage{color}
\setlength{\parskip}{8pt}
\setlength{\parsep}{0pt}
\setlength{\headsep}{0pt}
\setlength{\topskip}{0pt}
\setlength{\topmargin}{0pt}
\setlength{\topsep}{0pt}
\setlength{\partopsep}{0pt}
\setlength{\mathindent}{0cm}

\begin{document}
Cok Degiskenli Calculus - Ders 21

Bir onceki ders yol bagimsizligi ozelligi ve muhafazakarligin birebir
iliskide oldugunu gorduk. Bu derste bir alana bakarak o alanin gradyan
alani olup olmadigini anlamamizi saglayacak bir matematiksel kriter
gorecegiz, ve eger alan bir gradyan alani ise, onun bagli oldugu potansiyel
alani hesaplamanin yolunu isleyecegiz. 

Bir vektor alani $\vec{F} = <M,N>$ eger gradyan alani ise 

\[ M = f_x \]

\[ N = f_y \]

Bunu biliyoruz. Kismi turevlerin daha once ogrendigimiz ozelligine
gore, sunu da biliyoruz. $f_{xy} = f_{yx}$ mesela. O zaman elimizde bir
gradyan alani var ise

\[ M_y = N_x \]

dogru olmali. 

Tek kontrol etmemiz gereken bu. Tabii vektor alani $\vec{F} = <M,N>$ her
yerde tanimli ve turevi alinabilir bir formda olmali. Tanimlilik hakkinda -
odevlerimizden birisi bu konuyu isliyor mesela, size tek bir nokta
haricinde her yerde tanimli bir vektor alani veriyoruz, ve tum bu
anlattiklarimiz o noktada ise yaramaz hale geliyor. Bu konuyu daha derin
sekilde inceleyecegiz, mesela ``basit sekilde bagli bolgeler (simply
connected regions)'' konusuna bakacagiz. 

Simdilik su yeterli, eger alan her yerde tanimli ise ve $M_y = N_x$ ise,
alan bir gradyan alanidir. 

Ornek 

\[ \vec{F} = -y\hat{i} + x\hat{j} \]

\includegraphics[height=3cm]{21_1.png}

Bir onceki derste bu alanin muhafazakar olmadigina (yani gradyan alani
olamayacagina) karar vermistik, cunku ustteki cember etrafinda bir cizgi
entegrali alinca sonuc sifir degil, pozitif bir deger cikiyordu. Ama biz
yine de, biraz once ogrendigimiz teknigi kullanarak bunu kontrol edelim.

\[ \vec{F} = \underbrace{-y}_{M}\hat{i} + 
\underbrace{x}_{N}\hat{j} 
\]

\[ \frac{\partial M}{\partial y} = -1 \]

\[ \frac{\partial N}{\partial x} = 1 \]

Bu iki sonuc birbirinin aynisi degil. Demek ki alan bir gradyan alani
degil. 

Ornek

\[ \vec{F} = (4x^2 + axy)\hat{i} + (3y^2 + 4x^2)\hat{j} \]

Hangi $a$ degerleri icin bu alan bir gradyan alani olur? 

\[ M_y = ax \]

\[ N_x = 8x \]

$a=8$. Bu arada, ikinci kismi turevleri birbirine esit yapinca, bu esitlik
her noktada dogru olmalidir. Yani cebirsel olarak $ax = 8x$ seklinde bir
esitlik hazirlayip cebirsel numaralarla $x$'i bulmuyoruz, cunku bu cebirsel
cozumde $x=0$ da islerdi, ama bu dogru cevap olmazdi. Biz ikinci kismi
turevlerin ``ayni ifade'' olmasini istiyoruz.

Devam edelim. Alanimiz artik soyle

\[ \vec{F} = (4x^2 + 8xy)\hat{i} + (3y^2 + 4x^2)\hat{j} \]

Peki bu gradyan alaniniyla baglantili potansiyel alani nasil buluruz? 

Bir yontem tahmin etmektir, cogunlukla ise yarar. Ama daha sistematik olan
iki yontem gorecegiz simdi, sinavda bunlardan birini kullanin, cunku tahmin
her zaman ise yaramiyor, hatta bazi durumlarda tahmin yanlis yollara bile
goturebiliyor. 

1. Yontem - cizgi entegralini hesaplayarak 

\includegraphics[height=4cm]{21_2.png}

Alanimizda cizgi entegral alalim, benim en favori noktam, orijinden
baslasin, $x_1,y_1$ noktasina giden $C$ uzerinde yapilan isi hesaplasin.
Sonuc soyle olmali

\[ \int_C \vec{F} \cdot d\vec{r} = 
f(x_1,y_1) - f(0,0)
\]

Eger bu dogruysa sunu da yazabilmeliyim (basit bir cebirsel manipulasyon)

\[ f(x_1,y_1) = \int_C \vec{F} \cdot d\vec{r} + f(0,0) \]

ki $f(0,0)$ bir sabittir. Bu sabitin ne oldugunu bilmiyoruz, ama onun ne
olacagini aslinda ``tanimlayabiliriz''. 

Diyelim ki bir potansiyel fonksiyonumuz var. Bu fonksiyona 1 eklemek, ya da
hernangi baska bir sayi eklemek bu potansiyeli degistirmez, cunku
gradyanlar ayni kalacaktir (sabitler kismi turev alinirken yokolacagi
icin). o zaman $f(0,0)$'i bu eklenen sabit olarak gorebiliriz, ya da
entegrasyon sabiti olarak gorebiliriz. Aynen Calculus'ta anti-turev
alindigi zaman oldugu gibi, cebirsel olarak elde edilen formul ``sabit
haricinde'' elde edilen bir sonuctur. 

Fakat ustteki ihtiyaclar baglaminda, bir entegral alinmistir, ama sabit
haricinde formulde geri kalan aradigimiz potansiyel fonksiyonu olarak
gorulebilir. 

O zaman cizgi entegralini hesaplayacagiz. Bu hesapta ustteki resimdekinden
daha basit bir $C$ kullanmak daha iyi olur, mesela soyle (sari cizgi)

\includegraphics[height=4cm]{21_3.png}

Bu $C$ niye daha basit. Cunku iki parcanin ayri ayri entegralini alirken,
bir parcada $dy$, diger parcada $dx$ sifir olacak, cunku o eksende degisim
olmayacak, boylece cebirsel olarak isimiz daha kolaylasacak. 

Parcalari soyle tanimlayalim

\includegraphics[height=3cm]{21_4.png}

\[ \vec{F} = <4x^2 + 8xy, 3y^2 + 4x^2>\]

Entegal

\[ \int_C \vec{F} \cdot d\vec{r} = 
(4x^2 + 8xy) dx + (3y^2 + 4x^2) dy
\]

Parca $c_1$: $x$, 0'dan $x_1$'e gidiyor, $y=0$, $dy = 0$

\[ \int_{c_1} \vec{F} \cdot d\vec{r} = 
\int_0^{x_1} 4x^2 dx 
\]

Bir suru terim iptal oldu ve bu sayede ustteki ifade bayagi basitlesti. 

Bu arada niye $x,y$ yerine $x_1,y_1$ kullandigim herhalde belli oluyor,
$x,y$ entegrasyon degiskenlerim, $x_1,y_1$ ise sabitler. Neyse, sonuc

\[ = \frac{4}{3} x_1^3\]

Parca $c_2$: $y$, 0'dan $y_1$'e gidiyor, $x=x_1$, $dx = 0$. 

\[ \int_{c_2} \vec{F} \cdot d\vec{r} = 
\int_0^{y_1} (3y^2 + 4x_1^2) dy
\]

\[ = \bigg[ y^3 + 4x_1^2y \bigg]_0^{y_1} \]


\[ = y_1^3 + 4x_1^2 \]
















\end{document}
