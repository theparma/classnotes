\documentclass[12pt,fleqn]{article}
\setlength{\parindent}{0pt}
\usepackage{graphicx}
\usepackage{listings}
\usepackage[latin5]{inputenc}
\setlength{\parskip}{8pt}
\setlength{\parsep}{0pt}
\setlength{\headsep}{0pt}
\setlength{\topskip}{0pt}
\setlength{\topmargin}{0pt}
\setlength{\topsep}{0pt}
\setlength{\partopsep}{0pt}
\setlength{\mathindent}{0cm}

\begin{document}
MIT OCW Cok Degiskenli Calculus - Ders 14

Bagimsiz Olmayan Degiskenler (Non-independent Variables)

Ornek

Fizikteki $f(P,V,T)$ formulu, ki bu degiskenler 

\[ PV = nRT \]

seklinde ilintili. Daha genel olarak bir $f(x,y,z)$ formulu var, ve
degiskenler $x,y,z$ birbiriyle $g(x,y,z) = c$ uzerinden baglantili. Aslinda
bir onceki dersteki ayni durum, sadece bu sefer min, maks degil, kismi
turevlere neler oldugunu inceleyecegiz. 

Yine onceki dersteki gibi, belki $g$'yi cebirsel olarak degistirip, $f$'e
sokup degisken yoketmek mumkun degil. Eger oyle yapabilsek, bir $z =
z(x,y)$ 
olabilirdi, ve onun kismi turevlerine bakabilirdik,

\[ \frac{\partial z}{\partial x}, \frac{\partial z}{\partial y}, .. \]

gibi. Peki ya $z$'yi bulamiyorsak? Belki ustteki kismi turevleri $z$'yi
bulmadan elde edebiliriz. 

Ornek

\[ x^2 + yz + z^3 = 8 \]

$(2,3,1)$ noktasina bakalim (yerine koyunca hakikaten 8 ciktigini
goruyoruz). Fakat bu degerlerde azicik degisiklik yapinca, $z$ nasil
degisir? Bu soruyu nasil cevaplarim? 

Formulden $z$'yi cekip cikarmak gerekir, kupsel (cubic) formullerde bunu
yapmanin bir yolu var, fakat cok karmasik bir formul ortaya
cikartiyor. Aradigimiz sonuca ulasmanin daha kolay bir yolu var. 

$g$'nin tam diferansiyeline, yani $dg$'ye bakalim (ustteki formulu $g$
kabul ediyoruz). Tam diferansiyel

\[ 2x dx + z dy + (y+3z^2) dz = 0\]

Sag taraf sifir cunku ustteki $g$ bir sabite esit, $g=8$, sabitin degisimi
sifir, yani $dg=0$. 

Tam diferansiyele $(2,3,1)$ degerini verelim

\[ 4dx + dy + 6dz = 0 \]

Bu formul bize her degiskenin degisiminin digeri ile nasil baglantili
oldugunu gosteriyor. Mesela $dx$ ve $dy$'yi biliyorsak, $dz$'yi, yani
$z$'nin degisimini hesaplayabiliriz. Yani $z=z(x,y)$ uzerinden 

\[ dz = -\frac{1}{6}(4dx + dy) \]

Bu formul bize kismi turevleri de gostermis oluyor aslinda, cunku tam
diferansiyel formulunde kismi turevler vardir, ustteki formulde $dx,dy$'nin
yaninda yer alan degerler onlardir. O zaman

\[ \frac{\partial z}{\partial x} = -\frac{4}{6} = -\frac{2}{3} \]

\[ \frac{\partial z}{\partial y} = -\frac{1}{6} \]

Bunu dusunmenin bir diger yolu su. $\partial z/\partial x$ $z$'nin $x$'e
gore degisimi ise, $y$ sabit demektir, ustteki $dz$ formulunde $dy=0$
deriz, geri kalanlar

\[ dz = -\frac{2}{3}dx \]

ki bu formul $z$'nin $x$'teki degisime gore nasil degistigini gosteriyor. 

Genel olarak 

\[ g(x,y,z) = c \]

ise, o zaman 

\[ dg = g_x dx + g_y dy + g_z dz \]

formulu sifira esitlenir, ve bir diferansiyel digerinin formunda elde
edilebilir. 

\[ dz = -\frac{g_x}{g_z}dx -\frac{g_y}{g_z}dy \]

O zaman $\frac{\partial z}{\partial x}$'i gormek istiyorsak, $dx$'in katsayisina bakabiliriz, ya 
da $y=sabit$ yani $dy=0$ deriz, ve geri kalanlar

\[ dz =  -\frac{g_x}{g_z}dx, \ 
\frac{\partial z}{\partial x} = -\frac{g_x}{g_z}dx
\]

Daha fazla ilerlemeden, simdiye kadar gordugumuz notasyonun bazi
problemlerini inceleyelim. 

\[ f(x,y) = x+y \]

\[ \frac{\partial f}{\partial x} = 1\]

Degisken degisim (change of variables) yapalim

\[ x = u \]

\[ y = u+v \]

Pek cetrefilli bir degisim degil bu. O zaman 

\[ f = x + y = 2u + v \]

\[ \frac{\partial f}{\partial u} = 2\]

Bu nasil oldu? $x=u$ dedigimize gore, $x,u$ birbiriyle esitler, o zaman
kismi turevleri de ayni olmaliydi. 

Bu uyusmazligin niye ortaya ciktigini anlamak icin notasyonun ne demek
istedigine yakindan bakmamiz lazim. $\partial f/\partial x$ ile $x$'i degistiriyor, 
ama $y$'yi sabit tutuyoruz. $\partial f/\partial u$ ile $u$'yu degistiriyor, ama $v$'yi 
sabit tutuyoruz.

Yani evet, $x$ ile $u$'yu degistirmek ayni sey olabilir, ama $v$'yi sabit
tutmak ile $y$'yi sabit tutmak ayni sey degildir. Cunku mesela $y$'yi sabit
tutarsam ve $u$'yu degistirirsem, $v$ de degismelidir (ki biz bunu
istemiyoruz) $y = u+v$ ifadesindeki toplaminin sabit kalmasi icin. Ya da $v$ 
sabit ise ve $u$'yu degistiriyorsam, $y$ degisecektir. 

Yani hos, guzel kismi turev notasyonumuz neyin degistigini acikca
gostermesine ragmen, neyin sabit tutuldugunu gostermedigi icin yanilgilara
yol acabiliyor. Bunu aklimizda tutmamiz lazim. Ornekteki kismi turevler
birbiriyle ayni degil cunku 

$\partial f/\partial x$, $u=x$'i degistir, ve $y$'yi sabit tut

$\partial f/\partial u$, $u=x$'i degistir, ve $v = y-x$'i sabit tut

anlamina geliyor. 

Daha acik bir notasyon soyle olabilir

\[ 
\bigg( \frac{\partial f}{\partial x}  \bigg)_y = \textrm { y sabit}
 \]

\[ 
\bigg( \frac{\partial f}{\partial u}  \bigg)_v = \textrm { v sabit}
 \]

Ornege donersek

\[ 
\underbrace{
\bigg( \frac{\partial f}{\partial x}  \bigg)_y 
}_{1} 
\ne 
\underbrace{
\bigg( \frac{\partial f}{\partial x}  \bigg)_v = 
\bigg( \frac{\partial f}{\partial u}  \bigg)_v 
}_{2}
 \]

Ornek

\includegraphics[height=2cm]{14_1.png}

\[ A = \frac{1}{2}absin(\theta) \]

Alan, $a,b,\theta$'nin fonksiyonu. 

Farz edin ki size $a,b,\theta$ arasinda bir iliski oldugunu soyledim. 

\includegraphics[height=2cm]{14_2.png}

Diyelim ki ucgen aslinda bir dik ucgen, bunu cebirsel olarak soylemenin
yolu da alttaki kisitlama ifadesi

\[ a = bcos(\theta) \]

Incelemek istedigimiz alanin $\theta$'ya olan baglantisi, yani, mesela
$A$'nin degisiminin $\theta$'nin degisimine orani nedir? Bunu hesaplamanin
3 yontemi olabilir

1) $a,b,\theta$'yi bagimsiz kabul et, o zaman

\[ \frac{\partial A}{\partial \theta} = 
\bigg( \frac{\partial A}{\partial \theta} \bigg)_{a,b}  \]

Tabii $a,b$ sabitken $\theta$ degissin demek, ucgenin dikliginin ihlali
demektir, cunku hem kenarlar sabit, hem aci degissin diyoruz, ama o zaman
dik aci degismek zorundadir. Her neyse, kismi turevleri hesaplayalim. 

\[ \frac{\partial A}{\partial \theta} = \frac{1}{2}ab cos(\theta) \]

Simdiye kadar kisitlama ifadelerimi kullanmadim. 

2) $a$'yi sabit tutalim, $b$ degisebilsin, ki boylece dik aci yerinde
kalabilsin. 

\[ b = b(a,\theta) = \frac{a}{cos(\theta)} \]

\[ \bigg( \frac{\partial A}{\partial \theta} \bigg)_{a}  \]

3) $b$'yi sabit tutalim, $a = a(b,\theta)$ degissin, ki boylece dik aci yerinde
kalabilsin. 

\[ \bigg( \frac{\partial A}{\partial \theta} \bigg)_{b}  \]














\end{document}
