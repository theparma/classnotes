\documentclass[12pt,fleqn]{article}
\setlength{\parindent}{0pt}
\usepackage{graphicx}
\usepackage{listings}
\usepackage[latin5]{inputenc}
\setlength{\parskip}{8pt}
\setlength{\parsep}{0pt}
\setlength{\headsep}{0pt}
\setlength{\topskip}{0pt}
\setlength{\topmargin}{0pt}
\setlength{\topsep}{0pt}
\setlength{\partopsep}{0pt}
\setlength{\mathindent}{0cm}

\begin{document}
Cok Degiskenli Calculus - Ders 12

Zincirleme Kanunu hatirlayalim

\[ \frac{dw}{dt}  = w_x \frac{dx}{dt} + 
w_y \frac{dy}{dt} + 
w_z \frac{dz}{dt}  \]

Bu formul, kismi turevler uzerinden, $w$'daki degisimin $x,y,z$'deki
degisime ne kadar ``hassas'' ne kadar ``bagli'' oldugnu gosteriyor.

Simdi usttekini daha azaltilmis, ozetli (compact, concise) bir formda soyle
yazacagim. 

\[ = \nabla w \cdot  \frac{d\vec{r}}{dt} \]

Gradyan vektoru tum kismi turevlerin bir araya konmus halidir. 

\[ \nabla w = <w_x, w_y, w_z> \]

Tabii ki bunu soyleyince ustteki gradyan'in $x,y,z$'ye bagli oldugunu da
soyluyoruz, mesela $w$'nun belli bir nokta $x,y,z$'da gradyanini
alabilirsiniz, o zaman her degisik $x,y,z$ noktasinda farkli bir vektor
elde edersiniz, ki bu vektorlerin tamamina ileride ``vektor alani (vector
field)'' ismini verecegiz.







\end{document}
