\documentclass[12pt,fleqn]{article}
\setlength{\parindent}{0pt}
\usepackage{graphicx}
\usepackage{cancel}
\usepackage{listings}
\usepackage[latin5]{inputenc}
\usepackage{color}
\setlength{\parskip}{8pt}
\setlength{\parsep}{0pt}
\setlength{\headsep}{0pt}
\setlength{\topskip}{0pt}
\setlength{\topmargin}{0pt}
\setlength{\topsep}{0pt}
\setlength{\partopsep}{0pt}
\setlength{\mathindent}{0cm}

\begin{document}
Cok Degiskenli Calculus - Ders 20

Cizgi entegrallerini is hesabinda gormustuk. $\vec{F}$ tarafindan $C$
egrisi uzerinde yapilan isi

\includegraphics[height=2cm]{20_1.png}

\[ \int_C \vec{F} \cdot d\vec{r} =  \int_C \vec{F} \cdot \vec{T} ds \]

olarak gormustuk. Esitligin sagi birim teget vektoru kullanarak ayni hesabi
gosteriyor, $ds$ ise egri uzunlugu $s$'ten ortaya cikiyor. Diger bir form

\[ = \int_C M dx + N dy \]

ki $\vec{F} = <M,N>$ olmak uzere. 

Ornek 

Soyle bir vektor alani veriyorum

\[ \vec{F} = <y,x> \]

Bu alanin neye benzedigi cok bariz degil, ama bu alanin bir bilgisayar
cizimi altta

\includegraphics[height=5cm]{20_2.png}

Diyelim ki bu vektor alaninda orijinden baslayarak $c_1,c_2,c_3$
vektorlerini takip ederek hareket ettigimde yapilan isi hesaplamak
istiyorum. $c_1$ duz, $c_2$ birim cember uzerinde bir parca, $0 \le \theta
\le \pi / 4$ 
olmak uzere, ve $c_3$ tekrar duz. Yani 

\[ c = c_1 + c_2 + c_3 \]

O zaman is hesap entegralinin uc parcasi olacak. Her parca $i$ icin 

\[ \int_{C_i} y dx + x dy\]

gerekiyor. 

1) Yatay x ekseninde, $(0,0)$'dan $(1,0)$'a. 

\[ y = 0, \ dy = 0 \]

\[ \int_{c_1} y dx + x dy = 0 \ dx + 0 = 0\]

Cizgi entegrali cok basit yani. Bu sifir sonucunu baska sekilde de
gorebilirdik, vektor alanina bakarsak x eksenine her zaman dik oldugunu
gorururuz. O zaman $\vec{F}\cdot \vec{T}$ hep sifir sonucunu verecektir. 

2) $c_2$ bolumu. 

$x,y$'yi tek degisken baglaminda nasil temsil edecegimizi bulmamiz
gerekiyor. Eger bir cember uzerinde hareket ediyorsak, bu tek degisken aci
olabilir. 

\[ x = cos(\theta) \]

\[ y = sin(\theta) \]

\[ 0 \le \theta \le \pi / 4 \]

\includegraphics[height=3cm]{20_3.png}

Turevleri alirsak

\[ dx = -sin\theta \ d\theta\]

\[ dy = cos\theta \ d\theta \]

Entegral

\[ \int_{c_2} y dx + x dy = 
\int_0^{\pi/4} sin\theta (-sin\theta \ d\theta)  + 
cos\theta \ cos\theta \ d\theta
\]

\[ = \int_0^{\pi/4} cos^2\theta - sin^2\theta d\theta \]

\[ = \int_0^{\pi/4} cos(2\theta) d\theta \]

\[ = \frac{1}{2}sin2\theta \bigg|_0^{\pi/4} \]

\[ = \frac{1}{2} \]

3) $c_3$ bolumu

\[ \int_{c_2} y dx + x dy \]

\includegraphics[height=4cm]{20_4.png}

Basladigimiz noktayi biliyoruz, geriye dogru orijine gelecegiz. Bu cizgiyi
parametrize etmek zor degil. 

\[ x = \frac{1}{\sqrt{2}} - \frac{1}{\sqrt{2}} t \]

\[ y = \frac{1}{\sqrt{2}} - \frac{1}{\sqrt{2}} t \]

\[ 0 \le t \le 1 \]

Ama ustteki dogru olsa da, gereginden fazla cetrefil oldu. 

Daha kolay bir yontem ``orijinden'' ileri dogru bir yon dusunmek, ve sonra
``bunun tersi olsun'' diyerek istedigimiz gidisati elde etmek. Yani

\[ x = t \]

\[ y = t \]

ki $t$, 0 ile $1/\sqrt{2}$ arasinda. Bu bize $(-c_3)$'u verecek, yani
$c_3$'un tersini. O zaman entegrali su sekilde gorebiliriz

\[ \int_{-c_3} = - \int_{c_3}  \]

Buradaki numara 0'dan baslamanin cebirsel temsili kolaylastirmis
olmasi. Devam edelim

\[ dx = dt \]

\[ dy = dt \]

\[ \int_{-c_3} y dx + x dy  = 
\int_{0}^{\frac{1}{\sqrt{2}}} t \ dt + t \ dt = 
\int_{0}^{\frac{1}{\sqrt{2}}} 2t \ dt = 
t^2 \bigg|_{0}^{\frac{1}{\sqrt{2}}} =
\frac{1}{2}
\]

Usttekinin tersine ihtiyacimiz olduguna gore 

\[ \int_{c_3} y dx + x dy  = -\frac{1}{2}
\]


Ya da iki ustteki entegralin sinirlarini tam ters yonde de alabilirdik,
0'dan $1/\sqrt{2}$'a gitmek yerine, $1/\sqrt{2}$'dan 0'a gidebilirdik, o da
ayni sonucu verirdi. 

Nihayet, yapilan tum is, tum entegrallerin toplami olacagina gore

\[ \int_C = \int_{c_1} + \int_{c_2} + \int_{c_3}  \]

\[ = 0 + \frac{1}{2} - \frac{1}{2} = 0\]

Peki cizgisel entegralleri hesaplamaktan kurtulabilir miyiz? 

Simdi, gordugumuzde vurgulamamis olsakDiyelim ki vektor alani $\vec{F}$ bir
fonksiyonun gradyani, yani

\[ \vec{F}  = \nabla f\]

Bir gradyan alanimiz var yani, ve bu durumda $f(x,y)$'ye bir potansiyel
alani (potential field) diyebiliriz. Bu isim fizikle alakali dogal olarak,
$f$ fonksiyonu $x,y$ noktasinda ne kadar enerji, potansiyel,
vs. depolandigini gosterir genellikle, ve bu noktadaki gradyan kuvveti
verir. Daha dogrusu gradyanin negatifi, fizikciler gradyani eksi ile
carparlar, yani matematikciler ile aralarinda boyle bir fark
vardir. Aklimizda tutalim. Biz eksi olmayan yontemi kullanacagiz.

Cizgizel Entegraller Icin Calculus'un Temel Teorisi

\[ \int_{C} \nabla f \cdot d\vec{r} = 
f(P_1) - f(P_0)
 \]

\includegraphics[height=3cm]{20_5.png}

Bu cok faydali bir formul, ama sadece vektor alani bir gradyan ise, ve
$f$'i biliyorsak ise yarar. Ileriki bir derste bir vektor alaninin bir
gradyan olup olmadigini nasil anlayacagimizi gorecegiz, ve eger bir gradyan
ise, potansiyel fonksiyonunu geri elde etmenin tekniklerini gorecegiz. 


















\end{document}
