\documentclass[12pt,fleqn]{article}
\setlength{\parindent}{0pt}
\usepackage{graphicx}
\usepackage{listings}
\usepackage[latin5]{inputenc}
\setlength{\parskip}{8pt}
\setlength{\parsep}{0pt}
\setlength{\headsep}{0pt}
\setlength{\topskip}{0pt}
\setlength{\topmargin}{0pt}
\setlength{\topsep}{0pt}
\setlength{\partopsep}{0pt}
\setlength{\mathindent}{0cm}

\begin{document}
Cok Degiskenli Calculus - Ders 11

Onceki derste cok degiskenli fonksiyonlari min, maks uzerinden
inceledik. Bu derste bu tur fonksiyonlarin herhangi bir yondeki
varyasyonunu nasil hesaplayacagimizi gorecegiz. Bunu yapabilmek icin daha
fazla kavramsal araclara ihtiyacimiz var. 

Bugunku konumuz diferansiyeller. 

Diferansiyeller

Tek degiskenli Calculus'tan dolayli (implicit) diferansiyel almayi
biliyoruz herhalde. Mesela elimizde

\[ y = f(x) \]

var. Dolayli turevler ile $x$ uzerindeki sonsuz kucuk bir degisimi $y$
uzerindeki sonsuz kucuk bir degisime baglayabiliyoruz. 

\[ dy = f'(x) dx \]

Ornek

\[ y = sin^{-1}(x) \]

Bu formulun turevini bulmak icin, soyle bir zincir takip
edebiliriz. Usttekine tersten bakalim

\[ x = sin(y) \]

O zaman

\[ dx = cos(y)dy \]

\[ 
\frac{dy}{dx} = \frac{1}{cos(y)} = \frac{1}{\sqrt{1-x^2}}
 \]

Esitligin sag tarafini nasil elde ettik? Hatirlayalim

\[ cos^2(y) + sin^2(y) = 1 \]

\[ cos (y) = \sqrt{1 - sin^2y} \]

$sin^2(y)$ nedir? 

\[ y = sin^{-1}(x) \]

\[ sin(y) = sin(sin^{-1}(x)) \]

Ters sinus fonksiyonunun sinusu alinirsa, geriye sadece $x$ kalir. 

\[ sin(y) = x \]

O zaman

\[ cos (y) = \sqrt{1 - x^2} \]

Iste bu tur $dy,dx$ iceren formulleri kullanacagiz, bu derste cok
degiskenli ortamda bunu yapacagiz. 

Tam Diferansiyeller

Eger $f(x,y,z)$ varsa, 

\[ df = f_xdx + f_ydy + f_zdz \]

Diger notasyonla

\[ df = \frac{\partial f}{\partial x}dx + \frac{\partial f}{\partial y}dy + 
\frac{\partial f}{\partial z}dz \]

Bu formulun hesapladigi nedir? Elde edilen bir sayi, matris, vektor
degildir. Bu degisik turden bir nesne ve bu nesneleri manipule etmenin,
kullanmanin kendine has kurallari var. Onlari nasil kullanacagimizi
ogrenmemiz gerekiyor. 

Onlari nasil irdelemek gerekir? Bunu cevaplayabilmek icin onlari nasil
``gor\textbf{me}memiz'' gerektigini anlamamiz lazim. 





\end{document}
