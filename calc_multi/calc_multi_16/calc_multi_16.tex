\documentclass[12pt,fleqn]{article}
\setlength{\parindent}{0pt}
\usepackage{graphicx}
\usepackage{cancel}
\usepackage{listings}
\usepackage[latin5]{inputenc}
\usepackage{color}
\setlength{\parskip}{8pt}
\setlength{\parsep}{0pt}
\setlength{\headsep}{0pt}
\setlength{\topskip}{0pt}
\setlength{\topmargin}{0pt}
\setlength{\topsep}{0pt}
\setlength{\partopsep}{0pt}
\setlength{\mathindent}{0cm}
\usepackage{showkeys}
\renewcommand*\showkeyslabelformat[1]{(#1)}

\begin{document}
Ders 16

Konumuz cift entegraller (double integrals). Bildigimiz gibi su sekilde bir
entegral aldiginizda

\[ \int_a^bf(x)dx = \textit{f'in [a,b] arasindaki bolumdeki alani} \]

anlamina gelir. 

\includegraphics[height=2cm]{16_1.png}














\end{document}
