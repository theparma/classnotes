\documentclass[12pt,fleqn]{article}\usepackage{../common}
\begin{document}
Ders 18

Cift Entegrallerde Degisken Degisimi 

Ornek 

Bir elipsin alanini bulmak istedigimizi dusunelim. 

\includegraphics[height=2cm]{18_1.png}

\[ \bigg(\frac{x}{a}\bigg)^2 + \bigg(\frac{y}{b}\bigg)^2 = 1 \]

Formul cember formulune benziyor, tek fark $x,y$ kordinatlari farkli
sekillerde tekrar olceklenmisler (rescale). Elipsin alanini hesaplayalim. 

Diyelim ki 

\[ \int \int dx \ dy \]

olarak basladik. Ve

\[ \bigg(\frac{x}{a}\bigg)^2 + \bigg(\frac{y}{b}\bigg)^2 < 1 \]

olmak uzere $x,y$ uzerinden entegral alacagiz. Sinirlari ayarlamak,
vs. gibi islere hemen girisebiliriz, fakat bu is arap sacina donebilir. Bu
isi yapmanin en iyi yolu da degildir. 

Bu sekil bir cember olsaydi hemen kutupsal kordinata gecebilirdik, fakat
ustteki durumda bunu hemen yapamayiz. Fakat elips kenarlarindan bir basilmis
cemberdir, o zaman cemberi $a,b$ ile tekrar olceklersek, elips problemimizi
bir cember problemine indirgeyebiliriz. 

\[ \frac{x}{a} = u, \frac{y}{b} = v \]

O zaman alan soyle tanimlanabilir 

\[ \int \int_{u^2 + v^2 < 1} dx \ dy \]

Ama hala $dx,dy$ ile ne yapacagimiza karar vermedik. 

\[ du = \frac{1}{a}dx,dv = \frac{1}{b}dy \]

\[ du \ dv = \frac{1}{ab}dx \ dy \]

\[ dx \ dy = ab \ du \ dv \]

Entegrale sokarsak

\[ = ab \int \int_{u^2 + v^2 < 1} du \ dv \]

\[ = ab \cdot \textit{birim diskin alani} \]

\[ = \pi ab \]

Ustteki ornekte $u,v$ ile $x,y$ arasindaki iliski oldukca basitti. Eger
iliski daha cetrefil ise neler yapmamiz gerektigini alttaki ornekte
gorecegiz. Genel olarak yapmaya calistigimiz olcekleme faktorunun (scaling
factor) ne oldugunu bulmak, ki $dx \ dy$ ve $du \ dv$ arasindaki gecis
mumkun olsun.

Ornek 

\[ u = 3x - 2y \]

\[ v = x + y \]

Niye ustteki gibi $u,v$ kullanilmis? Belki entegre edilen fonksiyonu, belki
de sinirlari basitlestirmek istiyoruz ($x,y$ icin $A$, $u,v$ versiyonu icin
$A'$ kullanalim). 

$dA = dx \ dy$, ya da $dA' = du \ dv$ 

Fakat bir problem var. Alan hesabinda entegralin ufak alan parcalarini
topladigini soylemistik. 

\includegraphics[height=2cm]{18_2.png}

Problem soyle, ustteki dikdortgen, bu ornegin donusum formullerine gore
$u,v$ baglaminda alttaki gibi bir sekle donusecek, yani paralelogram
olacak. Parallelogram olacagini biliyoruz cunku degisim formulleri
lineer. 

\includegraphics[height=2cm]{18_3.png}

Bu tur degisimler matrisler uzerinden gosterilebilir bu arada, cunku
matrislerin, bir transformasyon matrisinin resimlere ne yapabilecegini
biliyoruz, yatirip, yassilastirma yapabiliyorlar mesela, acilar,
buyuklukler tamamen degisiyor vs. 

Sonsuz kucuklukte olmasa da, neler olacagini daha iyi gormek icin her
kenari 1 uzunlugunda, alani 1 olan kareye bakalim

\includegraphics[height=3cm]{18_4.png}

ve bu karenin her noktasini $u,v$ formulleriyle donusturursek

\includegraphics[height=3cm]{18_5.png}

elde edilir. Hakikaten bu bir paralelogram. Alan hesaplamak icin
determinant hesabi yapariz

\[ A' = 
\left|\begin{array}{rr}
3 & 1 \\
-2 & 1
\end{array}\right| = 5
 \]

$x,y$ formundaki 1 buyuklugundeki alan, 5 katina cikti. Yani 

\[ dA = 5 \ dA' \]

\[ du \ dv = 5 dx \ dy \]

O zaman entegrasyon sirasinda 

\[ \int \int ... \ dx \ dy = \int \int .. \ \frac{1}{5} du \ dv \]

haline gelmeli. 

Genel Durum 

\[ u = u(x,y) \]

\[ v = v(x,y)  \]

\[ \Delta u \approx u_x \Delta x + u_y \Delta y  \]

\[ \Delta v \approx v_x \Delta x + v_y \Delta y  \]

Matris formunda 

\[ 
\left[\begin{array}{r}
\Delta u \\
\Delta v 
\end{array}\right] \approx 
\left[\begin{array}{rr}
u_x & u_y \\
v_x & v_y
\end{array}\right] 
\left[\begin{array}{r}
\Delta x \\
\Delta y 
\end{array}\right] 
 \]

Demek ki kenarlari $\Delta x,\Delta y$ olan dikdortgen transform
edildiginde, kenarlari $\Delta u,\Delta v$ olan paralelogramin sekli kismi
turevlere bagli, cunku ustte transform eden matrisin icinde kismi turevler
var, ve kismi turevlerin degerleri belli $x,y$ noktalarinda hesaplandigina
gore, transformasyon da belli $x,y$ noktalarina da bagli. Eger cebirsel
olarak turetseydik, alan buyuklugu olceklenmesinin (ustte 5 olan) ustteki
transform matrisinin determinanti oldugunu gorurduk. 

\[<\Delta x,0> \to <\Delta u, \Delta v> \approx <u_x \Delta x, v_x\Delta x> \]

\[ <0, \Delta y> \to <\Delta u,\Delta v> \approx <u_y\Delta y,v_y \Delta y> \]

Eger ustteki iki formulde sag taraftaki vektorlerin determinantini alirsak,
ki o vektorler paralelogramin kenarlaridir, o zaman 

\[ Alan' =  det(..)\Delta x \Delta y  \]

bulurduk. 

Degiskenin degisiminin Jacobian'i denen bir kavramdan bahsedelim simdi: 

\[ J = \frac{\partial (u,v)}{\partial(x,y)} \]

Bu cok garip bir notasyon. $\partial$ isareti kullaniyorum ama bu cercevede
kismi turev aldigim anlaminda degil, $du \ dv$ ve $dx \ dy$ arasindaki
orani hesapladigimi bana hatirlatmasi icin. 

\[ J =  \left|\begin{array}{rr}
u_x & u_y \\
v_x & v_y
\end{array}\right|
\]

O zaman 

\[ du \ dv = 
|J| dx \ dy = 
\bigg|\frac{\partial (u,v)}{\partial(x,y)}\bigg| dx \ dy
 \]

$|J|$ ifadesi Jacobian'in yani determinant hesabinin mutlak degeri (absolute
value) anlaminda. Eger $J$ sonucu mesela -10 cikarsa, biz 10 kullanacagiz. 

Ornek 

Kutupsal forma gecerken transformasyonun $r \ dr \ d\theta$ gerektirdigini
bir ornek uzerinden gormustuk. Simdi bu yeni metotu kullanarak ayni sonucu
bulmaya calisalim. 

\[  x= r \cos \theta \]

\[  y= r \sin\theta \]

Degisken degisiminin Jacoban'i

\[ J = \frac{\partial (x,y)}{\partial(r,\theta)}  =
\left|\begin{array}{rr}
x_r & x_\theta \\
y_r & y_\theta
\end{array}\right|   =
\left|\begin{array}{rr}
\cos\theta & -r\sin\theta\\
\sin\theta & r\cos\theta
\end{array}\right|
\]

En sagdaki determinanti hesaplayinca

\[ = r\cos^2\theta + r\sin^2\theta \]

\[ = r \]

O zaman 

\[ dx \ dy = |r| dr \ d\theta \]

$r$ her zaman pozitif olduguna gore tam deger isaretine gerek yok

\[ = r \ dr \ d\theta\]

Yorum 

$u,v$ orneginde hedef $J$ hesabinin bolum kismindaydi, simdi hedef
$r,\theta$ bolen kisminda. Bu problem olur mu? 

Olmaz, cunku 

\[ J = \frac{\partial (u,v)}{\partial(x,y)} \cdot
\frac{\partial (x,y)}{\partial(u,v)} = 1
 \]

yani $u,v \to x,y$ yonunde transformu yapan Jacobian ile $x,y \to u,v$
yonunde transform yapan Jacobian birbirinin tersi. Bu sebeple transformu
yaparken bu yonlerden hangisini hesapladiginiz onemli degil, en kolayi
hangisiyse onu hesaplayin, sonra eger ters yon gerekiyorsa $1 / sonuc$ ile
istediginiz sonucu elde edin. Zaten dikkat edilirse, $u,v$ transformunda
$|J|$, $dx,dy$ yaninda cikti, ustteki son ornekte $r$, $dr,d\theta$
yaninda. 

Ornek

\[ \int_0^1 \int_0^1 x^2y dx \ dy  \]

Aslinda bu hesabi oldugu gibi yapmak cok kolay. Fakat biz isi
zorlastirarak, degisken degisimi

\[ u =x  \]

\[ v = xy \]

sonrasi hesabi yapmaya ugrasacagiz. Bir degisken degisimi entegre edilen,
ya da sinir ifadelerini basitlestirmek icin yapilir, ama ustteki degisim
bunlari hicbirini yapmiyor. Sadece ornek amacli bu zor yolu sectik zaten. 

Once neyin entegre edildigini bulalim. 

1) Alan ogesini bulalim, Jacobian'i kullanalim. 

\[ \frac{\partial (u,v)}{\partial(x,y)}  = 
\left|\begin{array}{rr}
1 & 0 \\
y & x
\end{array}\right| = 
x
 \]

Yani

\[ du \ dv = x dx \ dy \]

Tabii $x$'in tam degeri olacak, ama tanimladigimiz sinir icinde zaten $x$
pozitif. 

2) Entegre edilen formulun $u,v$ iceren hali 

\[ x^2y dx \ dy =  x^2y  \frac{1}{x} du \ dv = 
xy du \ dv = 
v \ du \ dv 
\]
O zaman 

\[ \int \int _{???} v \ du \ dv \]

Yeni sinirlari bulalim. Ustteki entegrale bakalim, ic entegral $u$
degisiyor, $v$ sabit kaliyor demektir. O zaman $v=xy$ sabit demektir, her
sabit $xy$ icin alttaki resimdeki kirmizi cizgilerden biri dusunulebilir

\includegraphics[height=3cm]{18_6.png}

Ic entegralin sinirlarini bulmak icin su soruyu cevaplamak gerekir, sabit
bir $v$ icin (cunku ic entegralde o sabit), $u$ degerleri hangi iki uc
deger arasinda hareket eder, ve entegrasyon bolgeme neresinden girip,
neresinden cikarim? 

Cevap icin tek bir kirmizi cizgiye bakarim, onun uzerinde giderken bolgeye
bir taraftan girip, oteki taraftan cikiyor olurum. 

Bolgenin en ust sinirini ele alalim, $y=1$. 

\[ y = \frac{v}{u}  \]

\[ u = v \]

O zaman $u$, $v$ ile ayni, $y=1$ iken hem $u,v$ 1'e esit.

\includegraphics[height=3cm]{18_7.png}

Sinirin en ustunden basliyorum, $u$, $v$ ile ayni, kirmizi cizgiyi takip
ediyorum, ediyorum, asagi dogru inerken $u$ buyuyor. Sag kenardan disari
cikiyorum. Sag kenarda $x=1$, o zaman $u = 1$. Demek ki sinir

\[ \int \int _v^1 v \ du \ dv \]

Dis Entegral 

$v$'nin en az, en fazla degerleri alttaki resimde sari ile cizili

\includegraphics[height=4cm]{18_8.png}

\[ \int_0^1 \int _v^1 v \ du \ dv \]

Bu bizi bayagi zorladi. Ama eger bu yontem takip edilecekse, bazi
tavsiyeler, $u,v$ baglaminda kesitlerin ne oldugunu bulmak, sonra her
birinin ayri ayri uzerinde kalip bolgeye nereden girilip cikildigina
bakmak. 

Ise yaramazsa, o zaman tum bolge $u,v$ olarak cizilmeye
ugrasilabilir. Teker teker 

\includegraphics[height=3cm]{18_9.png}

\includegraphics[height=3cm]{18_10.png}

\includegraphics[height=3cm]{18_11.png}

$u,v$ baglaminda bir ucgen elde edildi, ve bu ucgen uygun sekilde kesilerek
cift entegral hesaplanabilir. 

Ornek

$e^{-x^2}$ Nasil Entegre Edilir? 

\[ \int_{-\infty}^{\infty} e^{-x^2} dx
\mlabel{1} 
\]

ifadesi ozellikle olasilik matematiginde cokca gorulen bir ifadedir. Bu
hesabi yapmak icin kutupsal kordinatlar kullanacagiz. 

Simdi ustteki ifadeyle alakali su ifadeye bakalim. 

\[ \int_{-\infty}^{\infty} \int_{-\infty}^{\infty} e^{-x^2-y^2} dx dy \]

Iddia ediyorum ki bu son ifade (1)'in sadece karesi, yani (1)'in kendisiyle
carpimi. Niye boyle? Cunku $e$ ifadelerini carpim olarak gosterirsek

\[ \int_{-\infty}^{\infty} 
\underbrace{\int_{-\infty}^{\infty} e^{-x^2} e^{-y^2} dx}
dy \]

cift entegral icinde isaretlenen blokta yer alan $e^{-y^2}$ $x$'ten
bagimsiz, o zaman bloktaki entegralin disina alinabilir. Yani soyle olabilir

\[ \int_{-\infty}^{\infty} 
e^{-y^2} \int_{-\infty}^{\infty} e^{-x^2}  dx
dy \]

Devam edelim: ustteki ic entegral (1) ifadesi degil mi? Evet. Simdi bir
ilginc durum daha ortaya cikti, 

\[ \int_{-\infty}^{\infty}  e^{-y^2} 
\underbrace{\int_{-\infty}^{\infty} e^{-x^2}  dx}
dy \]

simdi de isaretlenen blok $y$ entegraline gore sabit, o da ikinci
entegralin disina cikarilabilir! (1) yerine $I$ kullanirsak 

\[ I \int_{-\infty}^{\infty}  e^{-y^2} dy \]

Icinde $y$ iceren entegral nedir? O da $I$'dir! Niye, cunku bu ifade 
(1)'in icinde $y$ olan versiyonundan ibaret. O zaman 

\[ I \cdot I = I^2 \]

Tum bu taklalari niye attik peki? Cunku cift entegralli ifadenin
entegralini almak daha kolay, eger onu hesaplarsak, sonucun karekokunu
aldigimiz anda $I$'yi bulmus olacagiz. 

O zaman ifadeyi hesaplayalim, 

\[ I^2 = \int_{-\infty}^{\infty} \int_{-\infty}^{\infty} e^{-x^2-y^2} dx dy \]

Basta soyledigimiz gibi kullanacagimiz numara kutupsal forma
gecmek. Entegralin sinirlarina bakalim, tum $x$ ve tum $y$ ekseni uzerinden
entegral aliyoruz. Kutupsal formda bu $r$'nin 0'dan sonsuza ve $\theta$'nin
0'dan $2\pi$'a gitmesi anlamina geliyor. 

\[ r^2 = x^2 + y^2 \]

Peki $e^{-x^2-y^2}$ kutupsal formda nedir? 

\[ e^{-x^2-y^2} = e^{-(x^2+y^2)} = e^{-r^2} \]

Entegrali yazalim

\[ \int_{0}^{\infty} \int_{0}^{2\pi} e^{-r^2} r d\theta dr 
\mlabel{2}
\]

Niye entegral sirasinda $\theta$'yi once yazdim? Cunku entegral icindeki
ifadede $\theta$'ya bagli hicbir terim yok, o zaman ic entegral bana sadece
$2\pi$ verir. Geriye kalanlar

\[=  2\pi \int_{0}^{\infty} e^{-r^2} r dr \]

Bu entegral cok daha kolay. Yerine koyma (subtitution) teknigi ile bu
problemi cozebiliriz. 

\[ u = -r^2 \]

\[ du =  -2r \ dr\]

\[=  2\pi \int_{0}^{\infty} e^u \frac{-1}{2} du \]

\[=  -\pi \int_{0}^{\infty} e^u  du \]

\[=  -\pi  e^u  \bigg|_{0}^{\infty} = -\pi  e^{-r^2}  \bigg|_{0}^{\infty} \]

\[ = \pi \]

Bu sonuc $I^2$. Eger $I$ degerini istiyorsak, karekok almaliyiz, yani
aradigimiz sonuc $\sqrt{\pi}$. 

Tek degiskenli bir problemi aldik ve cift degiskenli problem haline
getirdik. Isleri kolaylastiran (2) denklemindeki $r$ degiskeni oldu, onun
sayesinde yerine gecirme islemi cok kolaylasti, ve sonuca ulastik. 

Kaynaklar

\url{http://www.youtube.com/watch?v=fWOGfzC3IeY}

\end{document}
