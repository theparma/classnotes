\documentclass[12pt,fleqn]{article}\usepackage{../common}
\begin{document}
Veri Analizi - Ders 1

Gaussian Kontrolu

Diyelim ki Gaussian dagilimina sahip oldugunu dusundugumuz  $\{ x_i\}$
verilerimiz var. Bu verilerin Gaussian dagilimina uyup uymadigini nasil
kontrol edecegiz? Normal bir dagilimi soyle temsil edebiliriz, 

\[ y_i = \Phi\bigg(\frac{ x_i - \mu}{\sigma}\bigg) \]

Burada $\Phi$ standart Gaussian'i temsil ediyor (detaylar icin {\em Istatistik
Ders 1}). 


\end{document}
