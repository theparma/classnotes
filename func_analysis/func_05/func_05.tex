\documentclass[12pt,fleqn]{article}\usepackage{../common}
\begin{document}
Ders 5

Teori 

Bir kume $F$ kapalidir (closed), eger $F$ icindeki her yaklasiksal dizinin
limiti yine $F$ icindeyse. Ispat atlandi. 

Tanim 

Vektoru uzayi $X$'ten reel (ya da kompleks) skalar uzayina yapilan
transformasyona $X$ uzerinde tanimli bir {\em fonksiyonel} denir. 

Dikkat fonksiyon degil, fonksiyonel. Fonksiyonelleri diger daha genel
transformasyonlardan ayirtetmek icin onlara notasyon olarak kucuk harfler
verilir, mesela $f,g$ gibi. 

Norm edilmis uzayda $f(x) = ||x||$ bir fonksiyonel ornegidir. Yani norm
operatorunun kendisi de bir fonksiyoneldir. Reel degerli fonksiyoneller
optimizasyon teorisi acisindan cok onemlidir normal olarak cunku
optimizasyonun amaci bir fonksiyoneli minimize (ya da maksimize) edecek bir
vektoru bulmaktir. 

$l_p$ ve $L_p$ Uzaylari 

Simdi derslerin geri kalaninda cok kullanacagimiz, faydali olacak bazi
klasik norm edilmis uzaylari gorelim. 

Tanim 

$0 < p < \infty$ olacak sekilde $p$ bir reel sayi olsun. $l_p$ uzayi $\{
\xi_1,\xi_2,...\xi_n\}$ 
skalar dizisidir, ki bu dizi su sarta uymalidir,

\[ \sum_{i=1}^{\infty} |\xi_i|^p < \infty \]

$p$ sayisi tanimlanan uzaya gore degisir, yani $l_3$ olabilir, bir digeri
$l_5$, vs. Bu uzayin normu nedir? Dikkat, ustteki bir norm degil, uzayi
tanimlamak icin kullandigimiz sartlardan biri. Norm, 

\[ ||x||_p = \bigg( \sum_{i=1}^\infty |\xi_i|^p \bigg)^{1/p}  \]






\end{document}
