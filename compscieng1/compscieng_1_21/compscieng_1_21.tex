\documentclass[12pt,fleqn]{article}\usepackage{../common}
\begin{document}
Ders 21

Sinir Kosullari

Sinirli Elementler metodunda sabit kosullara verilen bir diger isim
Dirichlet, ya da vazgecilmez (essential) sarttir, mesela $u(0) = 0$ gibi,
serbest kosullara dogal ya da Neuman sarti denir, bu demektir ki hicbir
sart konulmasi gerekli degildir. 

Soru, eger sart 

\[ u' = A \]

olsaydi, yani sifir yerine $A$ olsaydi, ne olurdu? Cevap icin baslangica
donmek lazim, hatirlayalim, zayif formda (weak form) test fonksiyonu ile
carpiyoruz, sonra parcalarla entegre ediyoruz,

\[ \int (-cu')' vdx = \int (cu')v' dx - [cu'v]' \]














\end{document}
