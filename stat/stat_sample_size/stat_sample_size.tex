\documentclass[12pt,fleqn]{article}\usepackage{../common}
\begin{document}
Orneklem Buyuklugu

Bir arastirmaci $n$ bagimsiz deney baz alinarak elde edilen binom
parametresi $p$'yi tahmin etmek istiyor, fakat kac tane $n$ kullanmasi
gerektigini bilmiyor. Tabii ki daha buyuk $n$ degerleri daha iyi sonuclar
verecektir, her deneyin bir masrafi vardir. Bu iki gereklilik nasil birbiri
ile uzlastirilir?

Yeterli olacak en az kesinligi, duyarliligi (precision) bulmak icin Z
transformasyonu kullanilabilir belki. Diyelim ki $p$ icin maksimum olurluk
tahmini olan $X/n$'in en azindan $100(1-\alpha)\%$ olasilikta $p$'nin $d$
kadar yakininda olmasini istiyoruz. O zaman alttaki denklemi tatmin eden en
ufak $n$'i buldugumuz anda problemimizi cozduk demektir, 

$$ P\bigg( -d \g \frac{X}{n} - p \le d \bigg)  = 1-\alpha$$











\end{document}
