\documentclass[12pt,fleqn]{article}\usepackage{../common}
\begin{document}
Orneklem Buyuklugu

Bir arastirmaci $n$ bagimsiz deney baz alinarak elde edilen binom
parametresi $p$'yi tahmin etmek istiyor, fakat kac tane $n$ kullanmasi
gerektigini bilmiyor. Tabii ki daha buyuk $n$ degerleri daha iyi sonuclar
verecektir, her deneyin bir masrafi vardir. Bu iki gereklilik nasil birbiri
ile uzlastirilir?

Yeterli olacak en az kesinligi, duyarliligi (precision) bulmak icin Z
transformasyonu kullanilabilir belki. Diyelim ki $p$ icin maksimum olurluk
tahmini olan $X/n$'in en azindan $100(1-\alpha)\%$ olasilikta $p$'nin $d$
kadar yakininda olmasini istiyoruz. O zaman alttaki denklemi tatmin eden en
ufak $n$'i buldugumuz anda problemimizi cozduk demektir, 

$$ P\bigg( -d \le \frac{X}{n} - p \le d \bigg)  = 1-\alpha$$

Tahmin edici $X/n$'nin kendisi de bir rasgele degiskendir. Bu degisken
normal olarak dagilmistir, cunku $X$ Binom olarak dagilmis ise, bu dagilim
ayri Bernoulli'lerin toplamina esittir. Toplamin aritmetik ortalamasi ise
Merkezi Limit Kanunu'na gore normal'e yaklasir. O zaman, standardize etmek
icin $X/n$'i standart sapmaya bolmek gerekir. 

$$ Var(X/n) = \frac{1}{n^2}Var(X) = \frac{1}{n^2}np(1-p)=
\frac{1}{n}p(1-p) 
$$

cunku Binom dagilimlar icin $Var(X) = np(1-p)$. Standart sapma ustteki
ifadenin karekoku, yani 

$$ Std(X/n) = \sqrt{p(1-p)/n}
$$


$$ P\bigg( 
\frac{-d}{\sqrt{p(1-p)/n}} \le 
\frac{\frac{X}{n} - p }{\sqrt{p(1-p)/n}}\le 
\frac{d}{\sqrt{p(1-p)/n}} 
\bigg)  = 
1-\alpha$$


$$ P\bigg( 
\frac{-d}{\sqrt{p(1-p)/n}} \le 
Z
\frac{d}{\sqrt{p(1-p)/n}} 
\bigg)  = 
1-\alpha$$



\end{document}
