\documentclass[12pt,fleqn]{article}\usepackage{../common}
\begin{document}
Oge Kumeleri Bulmak ve Ikisel Matris Ayristirmasi (Binary Matrix Factorization)

Veri madenciligi denince pek cok kisinin aklina gelen ilk ornek, aslinda,
s�k bulunan ��e kumeleri (frequent itemsets) ornegidir: "filanca ulkeden
sitemize gelen musterilerin ayni zamanda vs ozelliklerinin oldugunu da
kesfettik" gibi. 

Benzer bir ornek, ki bu alan oge kumelerinin aslinda en onemli cikis
sebeplerinden birisidir, alisveris sepeti analizidir. Musterinin her
alisverisinde sepetinde belli mallar vardir, ve bu mallarin hangilerinin
ayni anda, ayni sepette oldugu analiz edilmeye ugrasilir. Eger surekli
ekmek ve recel ayni anda aliniyorsa, bu bilgi kullanilarak belki mallarin
daha iyi konumlandirilmasi yapilacaktir, vs. S�k bulunan oge kumeleri
teknikleri bazen degisik adlar altinda da gecebiliyor, mesela alaka
madenciligi (association mining) gibi. Algoritma olarak kullanilan pek cok
teknik var, APriori iyi bilinenlerden, FPGrowth ondan daha hizli calisan ve
daha tercih edilen bir teknik.

Bir diger alternatif ikisel matris ayristirmasi (binary matrix
factorization -BMF-) kullanmaktir [3]. Aynen SVD'de oldugu gibi BMF de bir
matrisi ayristirir, fakat uc matris yerine iki matrise ayristirir ve hem
sonuc matrisi hem de ayristirilan matrisler sadece 0 ya da 1 degerini
tasiyabilirler. Yani bu ayristirma sonuc matrislerinin ikisel olmasini
mecbur tutar, negatif olmayan matris ayristirmasinin (non-negative matrix
factorization) sonuc matrisinin pozitif degerler tasimasini mecbur kilmasi
gibi. Bunlar birer kisitlama (constraint) ve bu sonuc o kisitlamalara gore
ortaya cikiyor.

Ayristirma oncesi hangi kerte (rank) $k$ degerine gecmek istedigimizi biz
belirtiriz. BMF'nin oge kumeleri madenciligi icin faydasi surada: oge
kumeleri ararken baktigimiz ogeler kategorik seylerdir, alisveris sepeti
orneginde mesela ekmek, recel gibi. Kategorik ogeleri daha once one-hot
kodlamasi (encoding) ile 1/0 degerleri tasiyan yeni kolonlara
gecirebildigimizi gormustuk. Yani tamamen kategorik degerler tasiyan
veriler tamamen 1/0 tasiyacak sekilde tekrar kodlanabilir, yani ikisel
matris haline getirilebilir. Bu ikisel matrisi ayristirdigimiz zaman ve
kendileri de ikisel olan iki yeni matris elde ettigimizde ise bir anlamda
boyut indirgemesi yapmis oluruz, yani sanki ana matrisi ``ozetleriz''. Iste
bu ozet, ozellikle carpilan ``baz'' matris, oge kumelerinin hangileri
oldugu hakkinda ipuclari iceriyor olabilir.

Bir ornek uzerinde gorelim, mesela altta Alice (A), Bob Marley (B) ve Prens
Charles (C) verileri var. Bu kisiler icin saci uzun mu (long-haired), �nl�
m� (well-known) ve bay m� (male) verileri var. 

\includegraphics[height=5cm]{abc.png}

Bu matris uzerinde ikisel ayristirma yaparsak, $k=2$

\includegraphics[height=2cm]{abc_res.png}

Sonuca bakalim, ozellikle sol taraftaki carpilan ``baz'' matrisi. {\em
  Matris Carpimi} yazisindan hareketle, bu yazidaki kolon kombinasyon
bakisini kullanalim, o zaman soldaki baz matrisin dikey, kolon bazli
olarak, bir ozet oldugunu gorebiliyoruz. Cunku carpan sag taraf bu
kolonlari alip onlari belli sekillerde ``kombine ederek'' nihai (orijinal)
matrisi ortaya cikartabilmeli. Bu sebeple soldaki carpilan matris bir ozet
olmali / baz olusturmali, ve bunun yan etkisi olarak kolonlardaki
degerlerde belli bir kalip / oruntu (pattern) olmali. O zaman her baz
kolonunda birbiriyle alakali olan ogeler ayni anda 1 degeri tasiyor
olacaktir.

Sonuca gore uzun sacli ve unlu olmak (1. kolon) arasinda baglanti varmis ,
ayrica erkek olmak ve unlu olmak (2. kolon) arasinda da baglanti varmis :)
Veriye gore boyle en azindan.. Bu sonucu orijinal matrise bakarak ta
kontrol edebiliriz.

Ayristirma Kodlamasi 

BMF ozel bir hesaptir ve Numpy / Scipy icinde mevcut degildir. Bunun icin
ozel bir kutuphane cagrisi kullanabiliriz. Nimfa denen paket icinde [4]
gerekli kodlar var. Kurduktan sonra ustteki ornegi soyle cozebiliriz;

\begin{minted}{python}
import nimfa
import pandas as pd
import scipy.sparse as sp

def __fact_factor(X):
    return X.todense() if sp.isspmatrix(X) else X

A = np.array([[1., 1., 0],
              [1., 1., 1.],
              [0, 1., 1.]])

fctr = nimfa.mf(A,
              seed = "nndsvd", 
              rank = 2, 
              method = "bmf", 
              max_iter = 40, 
              initialize_only = True,
              lambda_w = 1.1,
              lambda_h = 1.1)

res = nimfa.mf_run(fctr)

threshold = 0.2
res1 = __fact_factor(res.basis())
res2 = __fact_factor(res.coef())
res1 = np.abs(np.round(res1 - 0.5 + threshold))
res2 = np.abs(np.round(res2 - 0.5 + threshold))
res1 = pd.DataFrame(res1, index=['long-haired','well-known','male'])
res2 = pd.DataFrame(res2, columns=['A','B','C'])
print res1
print '\n'
print res2
\end{minted}

\begin{verbatim}
             0  1
long-haired  1  0
well-known   1  1
male         0  1


   A  B  C
0  1  0  0
1  0  1  1
\end{verbatim}

Sonuc neredeyse tipatip ayni; sadece carpan matriste [0,B] kordinati 1
degil, fakat bize lazim olan baz matris ayni cikti. BMF hakkinda bazi ek
bilgiler: [2]'ye gore en az hatali BMF hesaplamak NP-hard zorlugunda, yani
3SAT gibi, ya da Seyahat Eden Satis Elemani (Traveling Salesman) problemi
gibi ki bu problem kombinatoryel (combinatorial) optimizasyon problemi,
yani cozum icin tum olasiliklar denendigi ve kisayolun mevcut olmadigi
cesitten problem. Fakat, yaklasiksal BMF metotlari oldukca hizli, ayrica
seyreklik cok fark yaratiyor (yani seyreklik iyi) ki kategorik veriler
gercek dunyada cogunlukla seyrek olarak goruluyor. Eldeki 2000 tane mal
cesidi icinden bir sepette ancak 5-10 tane urun oluyor mesela, tum 2000
tane mali bir sepete koymak mumkun degil.

Simdi baska bir ornek gorelim. 

\begin{minted}{python}
data = [
['outlook=sunny', 'temparature=hot', 'humidity=high', 'windy=false', 'play=no'],
['outlook=sunny', 'temparature=hot', 'humidity=high', 'windy=true', 'play=no'],
['outlook=overcast', 'temparature=hot', 'humidity=high', 'windy=false', 'play=yes'],
['outlook=rainy', 'temparature=mild', 'humidity=high', 'windy=false', 'play=yes'],
['outlook=rainy', 'temparature=cool', 'humidity=normal', 'windy=false', 'play=yes'],
['outlook=rainy', 'temparature=cool', 'humidity=normal', 'windy=true', 'play=no'],
['outlook=overcast', 'temparature=cool', 'humidity=normal', 'windy=true', 'play=yes'],
['outlook=sunny', 'temparature=mild', 'humidity=high', 'windy=false', 'play=no'],
['outlook=sunny', 'temparature=cool', 'humidity=normal', 'windy=false', 'play=yes'],
['outlook=rainy', 'temparature=mild', 'humidity=normal', 'windy=false', 'play=yes'],
['outlook=sunny', 'temparature=mild', 'humidity=normal', 'windy=true', 'play=yes'],
['outlook=overcast', 'temparature=mild', 'humidity=high', 'windy=true', 'play=yes'],
['outlook=overcast', 'temparature=hot', 'humidity=normal', 'windy=false', 'play=yes'],
['outlook=rainy', 'temparature=mild', 'humidity=high', 'windy=true', 'play=no']
]
\end{minted}

Hava ile alakali bazi veriler [1] bunlar; bu veriler tahmin (outlook),
sicaklik (temparature), nem (humidity), ruzgar (windy), disarida oyun
oynayan var mi (play). Mesela ilk satirda tahmin gunesli, isi sicak, nem
yuksek, ruzgar yok ve oyun oynayan yok. Bu sekilde bir suru satir. Biz bu
veride bir kalip olup olmadigina bakacagiz.

FPGrowth

Oge kumeleri bulmak icin BMF haricinde bir yontem FPGrowth yontemidir
[1,2]. Bu yontem once her ogeden (tek basina) kac tane oldugunu sayar,
belli bir esik degeri \verb!minsup! altinda olanlari atar, sonucu
siralar. Bu liste bir yapisina isaret eden bir baslik yapisi haline
gelir. Agacin kendisini olusturmak icin veri satirlari teker teker islenir,
her satirdaki her oge icin baslik yapisindaki en fazla degeri tasiyan oge
once olmak uzere tepeden baslanip alta dogru uzayan bir agac yapisi
olusturulur. Agactaki her dugum altindaki dugumun sayisal toplamini
tasir. Madencilik icin alttan baslanarak yukari dogru cikilir (amac en uste
ulasmak) ve bu sirada ogeler \verb!minsup! altinda ise, atilirlar. Sonucta
ulasilan ve atilmayan yollar bir oge kumesini temsil ederler. [2]'deki kodu
[1]'den aldigimiz ustteki veriye uygularsak, sonuc soyle:

\begin{minted}{python}
import fp
items = fp.fpgrowth(data, minsup=6)
for x in items:
    if len(x) > 1: print x
\end{minted}

\begin{verbatim}
<fp.node instance at 0x5017ef0>
   Null Set   1
     play=yes   9
       humidity=high   1
         windy=true   1
           temparature=mild   1
       windy=false   6
         humidity=high   2
           temparature=mild   1
         humidity=normal   4
           temparature=mild   1
       humidity=normal   2
         windy=true   2
           temparature=mild   1
     humidity=high   2
       windy=true   2
         temparature=mild   1
     windy=false   2
       humidity=high   2
         temparature=mild   1
     humidity=normal   1
       windy=true   1
   Null Set   1
     play=yes   6
   Null Set   1
     play=yes   6
set(['play=yes', 'humidity=normal'])
set(['play=yes', 'windy=false'])
\end{verbatim}

Bulunan sonuclar iki tane (tek ogeli sonuclar da var ama onlari
eledik). Bunlar hakikaten veri icindeki kaliplari temsil ediyorlar. Fena
degil. 

Kiyas icin BMF uzerinden madencilik yapalim. Once one-hot kodlamasi
yapalim, ve ornek icin bir veri satirini ekrana basalim,

\begin{minted}{python}
from sklearn.feature_extraction import DictVectorizer
import pandas as pd, re

def one_hot_dataframe(data, cols, replace=False):
    vec = DictVectorizer()
    mkdict = lambda row: dict((col, row[col]) for col in cols)
    tmp = data[cols].apply(mkdict, axis=1)
    vecData = pd.DataFrame(vec.fit_transform(tmp).toarray())
    vecData.columns = vec.get_feature_names()
    vecData.index = data.index
    if replace is True:
        data = data.drop(cols, axis=1)
        data = data.join(vecData)
    return (data, vecData, vec)

cols = ['outlook','temparature','humidity','windy','play']
df = pd.DataFrame(data,columns=cols)
# kolon ismini veriden cikart, cunku tekrar geri koyulacak
# fpgrowth icin veri icinde olmasi lazim
df = df.applymap(lambda x: re.sub('.*?=','',x))
df2, _, _ = one_hot_dataframe(df, cols, replace=True)
# tek ornek ekrana bas
print df2.ix[0]
\end{minted}

\begin{verbatim}
humidity=high       1
humidity=normal     0
outlook=overcast    0
outlook=rainy       0
outlook=sunny       1
play=no             1
play=yes            0
temparature=cool    0
temparature=hot     1
temparature=mild    0
windy=false         1
windy=true          0
Name: 0, dtype: float64
\end{verbatim}

Simdi BMF isletelim, $k=4$

\begin{minted}{python}
import nimfa
import scipy.sparse as sp

def __fact_factor(X):
    return X.todense() if sp.isspmatrix(X) else X

fctr = nimfa.mf(np.array(df2).T, seed = "nndsvd", 
              rank = 4, method = "bmf", 
              max_iter = 40, initialize_only = True,
              lambda_w = 1.1, lambda_h = 1.1)

res = nimfa.mf_run(fctr)

threshold = 0.2
res1 = __fact_factor(res.basis())
res2 = __fact_factor(res.coef())
res1 = np.abs(np.round(res1 - 0.5 + threshold))
res2=  np.abs(np.round(res2 - 0.5 + threshold))
res1 = pd.DataFrame(res1,index=df2.columns)
print res1
\end{minted}

\begin{verbatim}
                  0  1  2  3
humidity=high     1  0  0  1
humidity=normal   0  1  0  0
outlook=overcast  0  0  1  0
outlook=rainy     1  0  0  0
outlook=sunny     0  0  0  1
play=no           0  0  0  1
play=yes          0  1  1  0
temparature=cool  0  0  0  0
temparature=hot   0  0  0  0
temparature=mild  1  0  0  0
windy=false       0  0  1  0
windy=true        1  0  0  0
\end{verbatim}

Bu sonuclari kategoriksel hale cevirip tekrar ekrana basalim,

\begin{minted}{python}
for i in range(4):
    print np.array(df2.columns)[res1.ix[:,i] == 1]
\end{minted}

\begin{verbatim}
['humidity=high' 'outlook=rainy' 'temparature=mild' 'windy=true']
['humidity=normal' 'play=yes']
['outlook=overcast' 'play=yes' 'windy=false']
['humidity=high' 'outlook=sunny' 'play=no']
\end{verbatim}

1. sonuc atlanabilir, buradaki ``kalabalik'' orada bir kalip olmadigina
dair bir isaret. Ayristirma sonucu bu tur kolonlar ortaya cikabilir, diger
kolonlardaki kaliplar butunu temsil etmeye {\em tam} yetmemisse, arta kalan
her turlu gereklilik bir yerlere tikilabiliyor, bu normal. 2. sonuc
FPGrowth sonucunda var, guzel. 3. sonuc ta neredeyse ayni, sadece ek olarak
outlook=overcast var. Fakat, 3. sonuc aslinda onemli bir kalip iceriyor
olabilir, yani kalmasi daha iyi olur. Ayrica tavsiyemiz BMF sonuclari
uzerinde her sonuc icin ogelerini teker teker eleyerek kalanlarin kuvvetini
kontrol etmek, yani alt yeni sonuclara bakmak. Bu sekilde 3. uzerinde eleme
yaparken outlook=overcast elenince ana kalibi bulurduk. 

4. sonuc ise cok onemli bir kalip ve FPGrowth bunu tamamen kacirmis!

Sebep FPGrowth'un cozume lokal olarak erismeye calisiyor olmasi, kiyasla
BMF butune (global) bakiyor [3]. Bu ne demektir? Bir ayristirmanin ne
oldugunu dusunursek, bir matrisi olusturan carpimi ayristiriyoruz ve bu
ayristirma olduktan sonra iki matris elde ediyoruz. Bu iki matris ozgundur
(unique). Yani belli bir ikisel matrisi olusturan carpim sadece tek bir
sekilde olabilir. Buradan hareketle diyebiliriz ki bu ayristirma butunu
goze alarak yapilmalidir, sagi, solu tutan ama kosesi tutmayan bir
ayristirma olmaz. Bu sebeptendir ki ayristirma cozumunden belli bir
kapsayicilik bekleyebiliriz.

FPGrowth ise olaya yerel bakiyor; agac olustururken degisik bir sira takip
edilirse mesela degisik agaclar ortaya cikabilir. Ayrica her onemli iliski
muhakkak ozgun bir dal yapisinda olmayabilir. Madencilik algoritmasi alt
dallardan baslar ve yukariya dogru cikar, fakat bu her zaman iyi bir yontem
midir?

Kodlama Notlari

Su kod \verb!np.round(num - 0.5 + threshold)! kullanimi yuvarlama
(rounding) yapiyor, cunku Nimfa 1 degeri yerine 0.9, 0.8 gibi degerler
uretebiliyor, ayrica 0.1 gibi degerler de oluyor. Biz bildigimiz yuvarlama
\verb!.5!  sonrasi uzerini 1 yapmak yerine belli bir esik degeri
(threshold) uzerinden yuvarlama yaptik. Yani esik=0.2 ise 0.7 alta
yuvarlanir ve 0 olur, 0.9 esik ustunde oldugu icin uste yuvarlanir 1 olur.

BMF icin kerte $k$ kullanici tarafindan secilmeli, ama bu durum SVD, ya da
k-Means gibi diger yapay ogrenim metotlarindan farkli degildir. Bu
oynanmasi gereken, kesfedilmesi gereken bir deger. 

Kaynaklar

[1] Ian H. Witten, Eibe Frank, Mark A. Hall, {\em Data Mining Practical Machine Learning Tools and Techniques}

[2] Harrington, P., {\em Machine Learning in Action}

[3] \url{http://www.mpi-inf.mpg.de/~pmiettin/slides/BooleanMatrixFactorizationsForDataMining_Antwerp_slides.pdf}

[4] \url{http://nimfa.biolab.si/#installation}

\end{document}
