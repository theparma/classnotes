\documentclass[12pt,fleqn]{article}\usepackage{../common}
\begin{document}
Ders 2

Ornek 

$X \sim N(3,5)$ ise $P(X > 1)$ nedir? Cevap:

\[ P(X>1) = 1 - P(X < 1) = 1 - P( Z < \frac{ 1 - 3}{\sqrt{5 }}) = 
1 - \Phi(-0.8944) = .81
 \]

Soru tam $P(a  < X < b)$, sadece $b$ oldugu icin yukaridaki form ortaya
cikti. 

Ornek 

Simdi oyle bir $q$ bul ki $P(X < q) = .2$ olsun. Yani $\Phi^{-1}(.2)$'yi
bul. Yine $X \sim N(3,5)$. 

Cevap 

Demek ki tablodan $.2$ degerine tekabul eden esik degerini bulup, ustteki
formul uzerinden geriye tercume etmemiz gerekiyor. Normal tablosunda
$\Phi(-0.8416) = .2$, 

\[ .2 = P(X<q) = P( Z < \frac{ q - \mu}{\sigma}) = \Phi(\frac{ q - \mu}{\sigma})
\]

O zaman 

\[ -0.8416 = \frac{q - \mu}{\sigma} = \frac{ q - 3}{\sqrt{ 5}} \]

\[ q = 3 - 0.8416 \sqrt{ 5} = 1.1181 \]

$t$ (Student's t)ve Cauchy Dagilimi 

$X$, $v$ derece bagimsizlikta $t$ dagilimina sahiptir, ki bu $X \sim t_v$
diye yazilir eger 

\[ f(x) = 
\frac{ \Gamma(v+1)/2)} {\sqrt{v\pi}\Gamma(v/2)}
\bigg(1 + \frac{ x^2}{v}\bigg)^{-(v+1)/2}
 \]

$t$ dagilimi Normal dagilima benzer ama daha kuyrugu daha kalindir. Aslinda
Normal dagilimi $t$ dagiliminin $v = \infty$ oldugu hale tekabul
eder. Cauchy dagilimi da $t$'nin ozel bir halidir, $v = 1$ halidir. Bu
durumda yogunluk fonksiyonu

\[ f(x)  = \frac{ 1}{\pi(1+ x^2)} \]

Bu formul hakikaten bir yogunluk mudur? Kontrol icin entegralini alalim, 

\[ \int _{ -\infty}^{\infty} f(x) dx = 
\frac{ 1}{\pi} \int _{ -\infty}^{\infty} \frac{ dx}{1 + x^2} 
 \]

Cogunlukla entegre edilen yerde  ``1 arti ya da eksi bir seyin karesi''
turunde  bir ifade gorulurse, yerine gecirme (subsitution) islemi
trigonometrik  olarak  yapilir. 

\[  x = \tan \theta, \theta = \arctan x \]

\[ 1 + x^2 = 1 + \tan^2\theta = \sec^2\theta\]

\[ dx / d\theta = \sec^2\theta \]

O zaman 

\[ =
\frac{ 1}{\pi} \int _{ -\infty}^{\infty} \frac{ dx}{1 + x^2}   =
\frac{ 1}{\pi} \int _{ -\infty}^{\infty}  \frac{ 1}{\sec^2\theta}\sec^2\theta d\theta = 
\frac{ 1}{\pi} \int _{ -\infty}^{\infty}  1 \ d\theta = 
 \]

\[ = 
\frac{ 1}{\pi} \theta | _{ -\infty}^{\infty}   = 
\frac{ 1}{\pi} [\arctan(\infty) - \arctan(-\infty)]
 \]

\[ =
\frac{ 1}{\pi} [\frac{ \pi}{2} - (-\frac{ \pi}{2}) ] = 1
 \]


$\chi^2$ Dagilimi

$X$'in $p$ derece serbestlige sahip bir $\chi^2$ dagilima sahip ise $X \sim
\chi^2_p$ olarak gosterilir, yogunluk 

\[ f(x) = \frac{ 1}{\Gamma(p/2) 2^{p/2}} x^{(p/2) - 1} e^{-x/2 }, \ x > 0 \]

Eger $Z_1, .. , Z_p$ bagimsiz standart Normal rasgele degiskenler ise,
$\sum _{ i=1}^{p} Z_p \sim \chi^2_p$ esitligi dogrudur. 

Iki Degiskenli Dagilimlar 

Tanim

Surekli ortamda $(X,Y)$ rasgele degiskenleri icin yogunluk fonksiyonu
$f(x,y)$ tanimlanabilir eger i) $f(x,y) > 0, \ \forall (x,y)$ ise, ve ii)
$\int _{ -\infty}^{\infty} \int _{ -\infty}^{\infty} f(x,y) dx dy = 1$ ise ve her kume $A \subset \mathbb{R} \times \mathbb{R}$ icin 
$P((X,Y) \in A) = \int
\int_A f(x,y) dx dy$. Hem ayriksal hem surekli durumda 
birlesik (joint) CDF $F_{X,Y}(x,y) = P (X \le x, Y \le y)$ diye
gosterilir. 

Bu tanimda $A$ kumesi olarak tanimlanan kavram uygulamalarda bir olaya
(event) tekabul eder. Mesela

Ornek

$(X,Y)$'in birim kare uzerinde birbicimli (uniform) olsun. O zaman 

\[ 
f(x,y) =
\left\{ \begin{array}{ll}
1 & \textit{eger} \ 0 \le x \le 1, 0 \le y \le 1 \ ise\\
0 & \textit{diger durumlarda}
\end{array} \right.
 \]

$P(X < 1/2, Y < 1/2)$'yi bul. 

Cevap

Burada verilen $A = \{ X < 1/2, Y < 1/2\}$ bir altkumedir ve bir
olaydir. Olaylari boyle tanimlamamis miydik? Orneklem uzayinin bir
altkumesi olay degil midir? O zaman $f$'i verilen altkume uzerinden entegre
edersek, sonuca ulasmis oluruz. 

Ornek 

Eger dagilim kare olmayan bir bolge uzerinden tanimliysa hesaplar biraz
daha zorlasabilir. $(X,Y)$ yogunlugu 

\[ 
f(x,y) = 
\left\{ \begin{array}{ll}
cx^2y & eger \ x^2 \le y \le 1 \\
0 & digerleri
\end{array} \right.
 \]

Niye $c$ bilinmiyor? Belki problemin modellemesi sirasinda bu bilinmez
olarak ortaya cikmistir. Olabilir. Bu degeri hesaplayabiliriz, cunku
$f(x,y)$ yogunluk olmali, ve yogunluk olmanin sarti $f(x,y)$ entegre
edilince sonucun 1 olmasi. 

Once bir ek bilgi uretelim, eger $x^2 \le 1$ ise, o zaman $-1 \le x \le
1$ 
demektir. Bu lazim cunku entegrale sinir degeri olarak verilecek. 

\[ 1 = \int  \int f(x,y) dy dx = c \int _{ -1}^{1} \int _{ x^2}^{1}x^2y  \]

\[=  c \int _{ -1}^{1} x^2 \int _{ x^2}^{1} y dy dx = 
\int _{ -1}^{1} x^2 (\frac{ 1}{2} - \frac{ x^4}{2} )dx = 1
 \]

\[=  c \int _{ -1}^{1} x^2 (\frac{ 1 - x^4}{2} ) dx = 1 \]

\[ = \frac{ c}{2} \int _{ -1}^{1} x^2 - x^6 dx  = 1\]

Devam edersek $c = 21/4$ buluruz. 

Simdi, diyelim ki bizden $P(X \ge Y)$'yi hesaplamamiz isteniyor. Bu hangi
$A$ bolgesine tekabul eder? Elimizdekiler

\[ -1 \le x \le 1, \  x^2 \le y, \ y \le 1   \]

Simdi bunlara bir de $y \le x$ eklememiz lazim. Yani ortadaki esitsizlige
bir oge daha eklenir.

\[ -1 \le x \le 1 \]

\[  x^2 \le y \le x \]

\[  y \le 1   \]

$x^2 \le y$'yi hayal etmek icin $x^2 = y$'yi dusunelim, bu bir parabol
olarak cizilebilir, ve parabolun ustunde kalanlar otomatik olarak $x^2 \le
y$ 
olur, bu temel irdelemelerden biri. 


\includegraphics[height=4cm]{2_1.png}

Ayni sekilde $y \le x$ icin $y = x$'i dusunelim, ki bu 45 derece aciyla
cizilmis duz bir cizgi. Cizginin alti $y \le x$ olur. Bu iki bolgenin
kesisimi yukaridaki resimdeki golgeli kisim. 

Ek bir bolge sarti $0 \le x \le 1$. Bu sart resimde bariz goruluyor, ama
cebirsel olarak bakarsak $y \ge x^2$ oldugunu biliyoruz, o zaman $y \ge 0$
cunku $x^2$ muhakkak bir pozitif sayi olmali. Diger yandan $x \ge y$
verilmis, tum bunlari yanyana koyarsak $x \ge 0$ sarti ortaya cikar. 

Artik $P(X \ge Y)$ hesabi icin haziriz, 

\[ P(X \ge Y) = 
\frac{ 21}{4} \int_{ 0}^{1} \int _{ x^2}^{x} x^2y dy dx = 
\frac{ 21}{4} \int_{ 0}^{1} x^2 \bigg[ \int _{ x^2}^{x} y dy \bigg] dx 
 \]

\[ = \frac{ 21}{4} \int _{ 0}^{1} x^2 \frac{ x^2 - x^4}{2} dx = \frac{ 3}{20} \]


``Hafizasiz'' Dagilim, Ustel (Exponential) Dagilim

Ustel dagilimin hafizasiz oldugu soylenir. Bunun ne anlama geldigini
anlatmaya ugrasalim. Diyelim ki rasgele degisken $X$ bir aletin omrunu
temsil ediyor, yani bir $p(x)$ fonksiyonuna bir zaman ``sordugumuz'' zaman
bize dondurulen olasilik, o aletin $x$ zamani kadar daha islemesinin
olasiligi. Eger $p(2) = 0.2$ ise, aletin 2 yil daha yasamasinin olasiligi
0.2. 

Bu hafizasizligi, olasilik matematigi ile nasil temsil ederiz?

\[ P( X>s+t \ | X>t ) =  P(X>s) , \ \forall s, \ t \ge 0 \]

Yani oyle bir dagilim var ki elimizde, $X>t$ bilgisi veriliyor, ama (kalan)
zamani hala $P(X>s)$ olasiligi veriyor. Yani $t$ kadar zaman gectigi 
bilgisi hicbir seyi degistirmiyor. Ne kadar zaman gecmis olursa olsun,
direk $s$ ile gidip ayni olasilik hesabini yapiyoruz. 

Sartsal (conditional) formulunu uygularsak ustteki soyle olur

\[  \frac{P( X>s+t,  X>t )}{P(X>t)} = P(X>s)  \]

ya da

\[  P( X>s+t,  X>t ) = P(X>s)P(X>t) \]

Bu son denklemin tatmin olmasi icin $X$ ne sekilde dagilmis olmalidir?
Ustteki denklem sadece $X$ dagilim fonksiyonu ustel (exponential) olursa
mumkundur, cunku sadece o zaman

\[ e^{-\lambda(s+t)}  = e^{-\lambda s} e^{-\lambda t}\]

gibi bir iliski kurulabilir. 

Ornek

Diyelim ki bir bankadaki bekleme zamani ortalam 10 dakika ve ustel olarak
dagilmis. Bir musterinin i) bu bankada 15 dakika beklemesinin ihtimali
nedir? ii) Bu musterinin 10 dakika bekledikten sonra toplam olarak 15
dakika (ya da daha fazla) beklemesinin olasiligi nedir? 

Cevap

i) Eger $X$ musterinin bankada bekledigi zamani temsil ediyorsa

\[ P(X>15) = e^{-15 \cdot 1/10} = e^{-3/2} \approx 0.223 \]

ii) Sorunun bu kismi musteri 10 dakika gecirdikten sonra 5 dakika daha
gecirmesinin olasiligini soruyor. Fakat ustel dagilim ``hafizasiz'' oldugu
icin kalan zamani alip yine direk ayni fonksiyona geciyoruz, 

\[ P(X>5> = e^{-5 \cdot 1/10} = e^{-1/2} \approx 0.60\]

Kaynak

Introduction to Probability Models, Sheldon Ross, 8th Edition, sf. 273

\end{document}
