\documentclass[12pt,fleqn]{article}\usepackage{../common}
\begin{document}
Kovaryans ve Korelasyon

Bugun ``kovaryans gunu'', bu teknigi kullanarak nihayet bir toplamin
varyansini bulabilecegiz, varyans lineer degildir (kiyasla beklenti
-expectation- lineerdir). Bu lineer olmama durumu bizi korkutmayacak tabii,
sadece yanlis bir sekilde lineerlik uygulamak yerine probleme farkli bir
sekilde yaklasmayi ogrenecegiz. 

Diger bir acidan, hatta bu ana kullanimlardan biri, kovaryans iki rasgele
degiskeni ayni anda analiz etmemize yarayacak. 

Tanim

$$ Cov(X,Y) = E((X-E(X))(Y-E(Y)))  $$

Yani herhangi iki rasgele degiskeni $X,Y$'in kovaryansi $X$'ten
ortalamasi cikartilmis, $Y$'ten ortalamasi cikartilmis halinin carpilmasi
ve tum bu carpimlarin ortalamasinin alinmasidir. 

Tanim boyle. Simdi bu tanima biraz bakip onun hakkinda sezgi / anlayis
gelistirmeye ugrasalim. Tanim baska bir sekilde degil, bu sekilde yapilmis? 

Ilk once esitligin sag tarafindaki bir carpim, bir sey carpi bir baska
sey. Bu seylerden biri $X$ ile digeri $Y$ ile alakali, onlari carparak ve
carpimin bir ozelliginden faydalanarak sunu elde ettik; arti deger carpi
arti yine arti degerdir, eski carpi arti eksidir, eksi carpi eksi
artidir. Carpimlardan birinin buyuklugu $X$'in ortalamasina bagli olan bir
diger, ayni durum $Y$'yi iceren carpim icin gecerli. 




















\end{document}
