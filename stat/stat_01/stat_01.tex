\documentclass[12pt,fleqn]{article}
\setlength{\parindent}{0pt}
\usepackage{graphicx}
\usepackage{cancel}
\usepackage{listings}
\usepackage[latin5]{inputenc}
\usepackage{color}
\setlength{\parskip}{8pt}
\setlength{\parsep}{0pt}
\setlength{\headsep}{0pt}
\setlength{\topskip}{0pt}
\setlength{\topmargin}{0pt}
\setlength{\topsep}{0pt}
\setlength{\partopsep}{0pt}
\setlength{\mathindent}{0cm}

\begin{document}
Istatistik - Ders 1

Bu notlar makine ogrenimi, veri madenciligi gibi konularda gerekli olasilik
ve istatistik bilgisini paylasmak icin hazirlaniyor. Notlarda olasilik ve
istatistik ayni anda anlatilacak, ve uygulamalara agirlik verilecek. 

Orneklem Uzayi (Sample Space)

Orneklem uzayi $\Omega$ bir deneyin mumkun tum olasiliksal sonuclarin
(outcome) kumesidir. Eger deneyimiz ardi ardina iki kere yazi (T) tura (H)
atip sonucu kaydetmek ise, bu deneyin mumkun tum sonuclari soyledir

\[\Omega = \{HH,HT,TH,TT\} \]

Sonuclar ve Olaylar (Outcomes and Events)

$\Omega$ icindeki her nokta bir sonuctur (outcome). Olaylar $\Omega$'nin
herhangi bir alt kumesidir ve sonuclardan olusurlar. Mesela ustteki
yazi-tura deneyinde ``iki atisin icinden ilk atisin her zaman H gelmesi
olayi'' boyle bir alt kumedir, bu olaya $A$ diyelim, $A =
\{HH,HT\}$.

Ya da bir deneyin sonucu $\omega$ fiziksel bir olcum , diyelin ki sicaklik
olcumu. Sicaklik $\pm$, reel bir sayi olduguna gore, $\Omega = (-\infty,
+\infty)$, ve
sicaklik olcumunun 10'dan buyuk ama 23'ten kucuk ya da esit
olma ``olayi'' $A = (10,23]$. Koseli parantez kullanildi cunku sinir
degerini dahil ediyoruz. 

Ornek 

10 kere yazi-tura at. $A$ = ``en az bir tura gelme'' olayi olsun. $T_j$ ise
$j$'inci yazi-tura atisinda yazi gelme olayi olsun. $P(A)$ nedir? 

Bunun hesabi icin en kolayi, hic tura gelmeme, yani tamamen yazi gelme
olasiligini, $A^c$'yi hesaplamak, ve onu 1'den cikartmaktir. $^c$ sembolu
``tamamlayici (complement)'' kelimesinden geliyor.

\[ P(A) = 1 - P(A^c) \]

\[ = 1 - P(\textit{hepsi yazi}) \]

\[ = 1-P(T_1)P(T_2)...P(T_{10}) \]

\[ = 1 - \bigg(\frac{1}{2}\bigg)^{10} \approx .999 \]


Rasgele Degiskenler (Random Variables)

Bir rasgele degisken $X$ bir eslemedir, ki bu esleme $X: \Omega \to \Re$
her sonuc ile bir reel sayi arasindaki eslemedir. 

Olasilik derslerinde bir noktadan sonra artik ornekleme uzayindan
bahsedilmez, ama bu kavramin arkalarda bir yerde her zaman devrede oldugunu
hic aklimizdan cikartmayalim. 

Ornek

10 kere yazi-tura attik diyelim. VE yine diyelim ki $X(\omega)$ rasgele
degiskeni her $\omega$ siralamasinda (sequence) olan tura sayisi. Iste bir
esleme. Mesela eger $\omega = HHTHHTHHTT$ ise $X(\omega) = 6$. Tura sayisi
eslemesi $\omega$ sonucunu 6 sayisina esledi. 

Ornek 

$\Omega = \{ (x,y); x^2+y^2 \le 1 \}$, yani kume birim cember ve icindeki
reel sayilar (unit disc). Diyelim ki bu kumeden rasgele secim
yapiyoruz. Tipik bir sonuc $\omega = (x,y)$'dir. Tipik rasgele degiskenler
ise $X(\omega) = x$, $Y(\omega) = y$, $Z(\omega) = x+y$ olabilir. Goruldugu
gibi bir sonuc ile reel sayi arasinda esleme var. $X$ rasgele degiskeni
bir sonucu $x$'e eslemis, yani $(x,y)$ icinden sadece $x$'i cekip
cikartmis. Benzer sekilde $Y,Z$ degiskenleri var. 

Toplamsal Dagilim Fonksiyonu (Cumulative Distribution Function -CDF-)

Tanim

$X$ rasgele degiskeninin CDF'i $F_X: \Re \to [0,1]$ tanimi

\[ F_X(x) = P(X \ge x) \]

Eger $X$ ayriksal ise, yani sayilabilir bir kume $\{x_1,x_2,...\}$ icinden
degerler aliyorsa olasilik fonksiyonu (probability function), ya da
olasilik kutle fonksiyonu (probability mass function -PMF-) 

\[ f_X(x) = P(X = x) \]

Bazen $f_X$, ve $F_X$ yerine sadece $f$ ve $F$ yazariz. 

Tanim

Eger $X$ surekli (continuous) ise, yani tum $x$'ler icin $f_X(x) > 0$,
$\int_{-\infty}^{+\infty}f(x) dx = 1$ olacak sekilde bir $f_X$ mevcut ise, o zaman her $a \le b$ icin

\[ P(a<X<b) = \int_{a}^{b}f_X(x)dx \]

Bu durumda $f_X$ olasilik yogunluk fonksiyonudur (probability density function
-PDF-). 

\[ F_X = \int_{\infty}^{x}f_X(t)dt \]

Ayrica $F_X(x)$'in turevi alinabildigi her $x$ noktasinda  $f_X(x) = F'_X(x)$
demektir. 

Dikkat! Eger $X$ surekli ise o zaman $P(X = x) = 0$ degerindedir. $f(x)$
fonksiyonunu $P(X=x)$ olarak gormek hatalidir. Bu sadece ayriksal rasgele
degiskeninler icin isler. Surekli durumda olasilik hesabi icin belli iki
nokta arasinda entegral hesabi yapmamiz gereklidir. Ek olarak PDF 1'den
buyuk olabilir, ama PMF olamaz. PDF'in 1'den buyuk olabilmesi entegrali
bozmaz mi? Unutmayalim, entegral hesabi yapiyoruz, noktasal degerlerin 1
olmasi tum 1'lerin toplandigi anlamina gelmez. Bakiniz Entegralleri Nasil
Dusunelim yazimiz. 






















\end{document}
