\documentclass[12pt,fleqn]{article}
\setlength{\parindent}{0pt}
\usepackage{graphicx}
\usepackage{cancel}
\usepackage{listings}
\usepackage[latin5]{inputenc}
\usepackage{color}
\setlength{\parskip}{8pt}
\setlength{\parsep}{0pt}
\setlength{\headsep}{0pt}
\setlength{\topskip}{0pt}
\setlength{\topmargin}{0pt}
\setlength{\topsep}{0pt}
\setlength{\partopsep}{0pt}
\setlength{\mathindent}{0cm}

\begin{document}
Istatistik - Ders 1

Bu notlar makine ogrenimi, veri madenciligi gibi konularda gerekli olasilik
ve istatistik bilgisini paylasmak icin hazirlaniyor. Notlarda olasilik ve
istatistik ayni anda anlatilacak, ve uygulamalara agirlik verilecek. 

Orneklem Uzayi (Sample Space)

Orneklem uzayi $\Omega$ bir deneyin mumkun tum olasiliksal sonuclarin
(outcome) kumesidir. Eger deneyimiz ardi ardina iki kere yazi (T) tura (H)
atip sonucu kaydetmek ise, bu deneyin mumkun tum sonuclari soyledir

\[\Omega = \{HH,HT,TH,TT\} \]

Sonuclar ve Olaylar (Outcomes and Events)

$\Omega$ icindeki her nokta bir sonuctur (outcome). Olaylar $\Omega$'nin
herhangi bir alt kumesidir. Mesela ustteki yazi tura deneyinde ``iki atisin
icinden ilk atisin her zaman H gelmesi olayi'' boyle bir alt kume
olusturur, bu olaya $A$ diyelim, ve bu kume $A = \{HH,HT\}$ olacaktir. 

Ya da bir deneyin sonucu $\omega$ fiziksel bir olcum , diyelin ki sicaklik
olcumu. Sicaklik $\pm$, reel bir sayi olduguna gore, $\Omega = (-\infty,
+\infty)$, ve
sicaklik olcumunun 10'dan buyuk ama 23'ten kucuk ya da esit
olma ``olayi'' $A = (10,23]$. Koseli parantez kullanildi cunku sinir
degerini dahil ediyoruz. 

Rasgele Degiskenler (Random Variables)

Bir rasgele degisken $X$ bir eslemedir, ki bu esleme $X: \Omega \to \Re$
seklindedir, yani bir sonucu bir reel sayi esler. 




















\end{document}
