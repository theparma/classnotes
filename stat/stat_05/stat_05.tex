\documentclass[12pt,fleqn]{article}\usepackage{../common}
\begin{document}
Ders 5

Ortalama (Mean) ve Medyan (Median)

Ozet Istatistikleri 

Genellikle istatistik kitaplari hemen ortalama (mean), medyan (median) ve
baglantili ozet istatistiklerinden (summary statistics) bahsederek ise
girerler. Bu istatistikleri dikkatle kullanmak gerekir, cunku her turlu
veri, her yerde gecerli degildirler. Mesela ortalama sadece tek merkezi bir
tepesi olan (unimodal) dagilimlar icin gecerlidir. Eger bu temel varsayim
gecerli degilse, ortalama kullanarak yapilan hesaplar bizi yanlis yollara
goturur. Ayrica bir dagilimi simetrik olup olmadigi da ortalama ya da
medyan kullanilip kullanilmamasi kararinda onemlidir. Eger simetrik, tek
tepeli bir dagilim var ise, ortalama ve medyan birbirine yakin
olacaktir. Fakat veri baska turde bir dagilim ise, o zaman bu iki olcut
birbirinden cok farkli olabilir.

Once ortalama ve standart sapmayi (standart deviation) gorelim.

\[ m  = \frac{ 1}{n}\sum_i x_i \]

Standart sapma veri noktalarin ``ortalamadan farkinin ortalamasini''
verir. Tabii bazen noktalar ortalamanin altinda, bazen ustunde olacaktir,
bizi bu negatiflik, pozitiflik ilgilendirmez, biz sadece farkla
alakaliyiz. O yuzden her sapmanin karesini aliriz, bunlari toplayip nokta
sayisina boleriz .

\[ s^2 = \frac{ 1}{n} \sum_i (x_i - m)^2 \]

Eger $m$ tanimini ustte yerine koyarsak, 

\[ = \frac{ 1}{n} \sum_i x_i^2 + \frac{ 1}{n} \sum_i m^2 - \frac{ 2}{n} \sum_i x_im  \]

\[ = \frac{ 1}{n} \sum_i x_i^2 + \frac{ m^2n}{n} - \frac{ 2mn}{n}m \]

\[ = \frac{ 1}{n} \sum_i x_i^2 +  m^2 - 2m^2 \]

\[ = \frac{ 1}{n} \sum_i x_i^2 - m^2 \]

Bu olcuye varyans (variance) denir ve teorik olarak ortalamadan daha onemli
oldugu soylenebilir. Fakat dagilimin yayilma olcusu olarak biz bu olcuyu
oldugu gibi degil, onun karesini kullanacagiz (ki standart sapma buna
deniyor aslinda). Niye? Cunku o zaman veri noktalarinin ve yayilma olcusunun
birimleri birbiri ile ayni olacak. Eger veri setimiz bir alisveris
sepetindeki malzemelerin lira cinsinden degerleri olsaydi, varyans bize
sonucu ``karekok lira'' olarak verecekti ve bunun pek anlami olmayacakti. 

Medyan ve Yuzdelikler (Percentile)

Ustteki hesaplar sayilari toplayip, bolmek uzerinden yapildi. Medyan ve
diger yuzdeliklerin hesabi (ki medyan 50. yuzdelige tekabul eder) icin
eldeki tum degerleri ``siraya dizmemiz'' ve sonra 50. yuzdelik icin {\em
  ortadakine} bakmamiz gerekiyor. Mesela eger ilk 5. yuzdeligi ariyorsak ve
elimizde 80 tane deger var ise, bastan 4. sayiya / vektor hucresine /
ogeye bakmamiz gerekiyor. Eger 100 eleman var ise, 5. sayiya bakmamiz
gerekiyor, vs. 

Bu siraya dizme islemi kritik. Kiyasla ortalama hesabi hangi sirada olursa
olsun, sayilari birbirine topluyor ve sonra boluyor. Zaten ortalama ve
sapmanin istatistikte daha cok kullanilmasinin tarihi sebebi de aslinda bu;
bilgisayar oncesi cagda sayilari siralamak (sorting) zor bir isti. Bu
sebele hangi sirada olursa olsun, toplayip, bolerek hesaplanabilecek
ozetler daha makbuldu. Fakat artik siralama islemi kolay, ve veri setleri
her zaman tek tepeli, simetrik olmayabiliyor. 

Ornek veri seti olarak unlu \verb!dellstore2! tabanindaki satis miktarlari
kullanirsak, 

\begin{verbatim}
In [1]: from pylab import *

In [2]: import numpy as np

In [3]: data = loadtxt("dell.csv")

In [4]: np.mean(data)
Out[4]: 213.94889916666668

In [5]: np.median(data)
Out[5]: 214.06

In [6]: np.std(data)
Out[6]: 125.1184819538924

In [7]: np.mean(data)+2*np.std(data)
Out[7]: 464.1858630744515

In [8]: np.percentile(data, 95)
Out[8]: 410.41150000000005
\end{verbatim}

Goruldugu gibi uc nokta hesabi icin ortalamadan iki sapma otesini
kullanirsak, \verb!464.18!, fakat 95. yuzdeligi kullanirsak \verb!410.41!
elde ediyoruz. Niye? Sebep ortalamanin kendisi hesaplanirken cok uc
degerlerin toplama dahil edilmis olmasi ve bu durum, ortalamanin kendisini
daha buyuk seviyeye dogru itiyor. Yuzdelik hesabi ise sadece sayilari
siralayip belli bazi elemanlari otomatik olarak uc nokta olarak addediyor.

Box Whisker Grafikleri

Tek boyutlu bir verinin dagilimini gormek icin Box ve Whisker grafikleri
faydali araclardir; medyan (median), dagilimin genisligini ve siradisi
noktalari (outliers) acik sekilde gosterirler. Isim nereden geliyor? Box
yani kutu, dagilimin agirliginin nerede oldugunu gosterir, medyanin
sagindada ve solunda olmak uzere iki ceyregin arasindaki kisimdir, kutu
olarak resmedilir. Whiskers kedilerin biyiklarina verilen isimdir, zaten
grafikte birazcik biyik gibi duruyorlar. Bu uzantilar medyan noktasindan
her iki yana kutunun iki kati kadar uzatilir sonra verideki "ondan az olan
en buyuk" noktaya kadar geri cekilir. Tum bunlarin disinda kalan veri ise
teker teker nokta olarak grafikte basilir. Bunlar siradisi (outlier)
olduklari icin daha az olacaklari tahmin edilir.

BW grafikleri iki veriyi dagilimsal olarak karsilastirmak icin
birebirdir. Mesela Larsen and Marx adli arastirmacilar cok az veri iceren
Quintus Curtius Snodgrass veri setinin degisik oldugunu ispatlamak icin bir
suru hesap yapmislardir, bir suru matematiksel isleme girmislerdir, fakat
basit bir BW grafigi iki setin farkliligini hemen gosterir.

BW grafikleri iki veriyi dagilimsal olarak karsilastirmak icin
birebirdir. Mesela Larsen and Marx adli arastirmacilar cok az veri iceren
Quintus Curtius Snodgrass veri setinin degisik oldugunu ispatlamak icin bir
suru hesap yapmislardir, bir suru matematiksel isleme girmislerdir, fakat
basit bir BW grafigi iki setin farkliligini hemen gosterir.

Python uzerinde basit bir BW grafigi 

\lstinputlisting[language=Python]{box1.py}

Bir diger ornek Glass veri seti uzerinde

\lstinputlisting[language=Python]{box2.py}

\includegraphics[height=8cm]{05_02.png}

Guven Araligi (Confidence Intervals)

Bu kavram istatistikte tartisilan konulardan biri. Bayes ve Frenkans�� 
(Frequentist) istatistik arasindaki felsefi farklardan biri burada ortaya
cikiyor. Frekansci tanim soyledir:

\begin{quote}
Bir parametre $\theta$ icin $1-\alpha$ seviyesinde bir $C_n=(a,b)$ guven araligi
tanimlanabilir -- bu aralik $a=a(X_1,..,X_n)$ ve $b=b(X_1,..,X_n)$ adli iki
fonksiyon uzerinden tanimlanabilir. Bu fonksiyonlar veri uzerinde isleyen, 
{\em verinin} fonksiyonlaridir, ve sonucta

\[ \mathbb{P}_\theta (\theta \in C_n) \ge 1-\alpha, \ \ \forall \theta \in \Theta \]

Yani $(a,b)$ araligi $1-\alpha$ olasiliginda $\theta$'yi icine alir /
hapseder. Daha detayli olarak deney arka arkaya pek cok kez
tekrarlandiginda parametrenin tahmininin $1-\alpha$ oraninda tanimlanan
araliga dusecegi soylenir. $1-\alpha$ sayisina guven araliginin kapsami
(coverage) ismi de verilir. Genellikle insanlar yuzde 95 guven araligini
kullanirlar, ve bu yuzdeye tekabul eden $\alpha = 0.05$ rakami kullanilir.

Uzerine basarak belirtmek gerekir ki $C_n$ rasgele (random) bir degerdir,
ama $\theta$ sabittir, cunku $C_n$ verinin bir fonksiyonudur, ve veriden,
yani bir orneklemden gelecegi icin o da rasgele olmalidir.

Eger $\theta$ bir vektor ise o zaman bir aralik yerine bir guven kumesi
kullanilir (mesela bir kure, ya da elips). 
\end{quote}

Fakat frekansci yaklasimda aralik fonksiyonlari $a,b$ ile guven araligi
arasindaki baglanti net degildir. Hangi fonksiyon secimi hangi $\alpha$'ya
sebebiye vermektedir? Bu durum net oldugu durumlarda bile teorik olarak
saglamligi suphelidir, ayrica hesabin sozel olarak ortaya konmasinda bazi
eksikler vardir. ``Deney arka arkaya pek cok kez tekrarlandiginda
parametrenin tahmini, $1-\alpha$ guven araliga dusecektir'' ibaresi
mesela; ``deney tekrari'' her durumda gecerli olmayabilir. Meteoroloji
``yarin yuzde 80 ihtimali ile yagmur yagacak'' diyorsa, o hesap sartlarinin
bir daha ortaya cikmasinin olasiligi cok dusuktur, Kaos Teorisi bize en
azindan bunu soyluyor.

Wiki sayfasinda [1] tartismanin boyutlari gorulebilir.

Son onyillarda ortaya cikan yaklasim ise Bayes Teorisini devreye
sokmak. Bir guven araligi tanimlamanin en saglam yolu bu hesabi bir
dagilimi baz alarak yapmak. Eger sonuc olarak bir tekil sayi degil, bir
dagilim elde edersek bu dagilim uzerinde guvenlik hesaplarini yapmak cok
kolay hale gelir. Mesela sonuc (sonsal dagilim) bir Gaussian dagilim ise,
bu dagilimin yuzde 95 agirliginin nerede oldugu, ve nasil hesaplandigi
bellidir. 

Bayes Teorisi

\[ P(A|B)  = \frac{ P(B|A)P(A)}{P(B)} \]

Veri analizi baglaminda diyelim ki deneyler yaparak tahmini olarak
hesaplamak (estimate) istedigimiz bir parametre var, bu bir protonun
kutlesi ya da bir ameliyat sonrasi hayatta kalma orani olabilir. Bu
durumlarda iki ayri ``olaydan'' bahsetmemiz gerekir, B olayi spesifik bazi
olcumlerin elde edilmesi ``olayidir'', mesela olcum uc sayidan olusuyorsa,
biz bir olcumde spesifik olarak $\{0.2,4,5.4\}$ degerlerini elde
etmisiz. Ikinci olay bilmedigimiz parametrenin belli bir degere sahip
olmasi olacak. O zaman Bayes Teorisinin su sekilde tekrar yazabiliriz, 

\[ P(parametre | veri ) \propto P(data | parametre)P(parametre) \]

$\propto$ isareti orantili olmak (proportional to) anlamina geliyor. Boleni
attik cunku o bir sabit (tamamen veriye bagli, tahmini hesaplamak
istedigimiz parametreye bagli degil). Tabii bu durumda sol ve sag taraf
birbirine esit olmaz, o yuzden esitlik yerine orantili olmak isaretini
kullandik. Bu cercevede ``belli bir numerik sabit cercevesinde birbirine
esit (equal within a numeric constant)'' gibi cumleler de gorulebilir. 

Ornek

Diyelim ki bir bozuk para ile 10 kere yazi-tura attik, ve sonuc altta

T H H H H T T H H H

Bu veriye bakarak paranin hileli olup olmadigini anlamaya
calisacagiz. Bayes ifadesini bu veriye gore yazalim,

\[ P(p | \{ \textrm{T H H H H T T H H H} \} \propto 
P(\{ \textrm{T H H H H T T H H H} | p) P(p) \}
\]

$P(p)$ ifadesi ne anlama gelir? Aslinda bu ifadeyi $P([Dagilim] = p)$
olarak gormek daha iyi, artik $p$ parametresini bir dagilimdan gelen bir
tekil deger olarak gordugumuze gore, o dagilimin belli bir $p$'ye esit
oldugu zamani modelliyoruz burada. Her halukarda $P(p)$ dagilimini, yani
onsel (prior) olasiligi bilmiyoruz, hesaptan once her degerin mumkun
oldugunu biliyoruz, o zaman bu onsel dagilimi duz (flat) olarak aliriz,
yani $P(p) = 1$. 

$P(\{\textrm{T H H H H T T H H H} | p)$ ifadesi goz korkutucu olabilir, ama
buradaki her ogenin bagimsiz ozdesce dagilmis (independent identically
distributed) oldugunu gorursek, ama bu ifadeyi ayri ayri
$P(\{\textrm{T}|p)$ ve $P(\{\textrm{H}|p)$ carpimlari olarak gorebiliriz. $P(\{\textrm{T}|p) = p$ ve 
$P(\{\textrm{H}|p)=1-p$ oldugunu biliyoruz. O zaman 

\[ P(p | \{ \textrm{7 Tura, 3 Yazi} \} \propto
p^7(1-p)^3
\]

Grafiklersek, 

\includegraphics[height=8cm]{05_01.png}

Boylece $p$ icin bir sonsal (posterior) dagilim elde ettik. Artik bu
dagilimin yuzde 95 agirliginin nerede oldugunu rahatca gorebiliriz /
hesaplayabiliriz. Dagilimin tepe noktasinin $p=0.7$ civarinda oldugu
goruluyor. Bir dagilimla daha fazlasini yapmak ta mumkun, mesela bu
fonksiyonu $p$'ye bagli baska bir fonksiyona karsi entegre etmek mumkun,
mesela beklentiyi bu sekilde hesaplayabiliriz. 

Onsel dagilimin her noktaya esit agirlik veren birornek (uniform) secilmis
olmasi, yani problemi cozmeye sifir bilgiden baslamis olmamiz, yontemin bir
zayifligi olarak gorulmemeli. Yontemin kuvveti elimizdeki bilgiyle baslayip
onu net bir sekilde veri ve olurluk uzerinden sonsal tek dagilima
goturebilmesi. Baslangic ve sonuc arasindaki baglanti gayet net. Fazlasi da
var; ilgilendigimiz alani (domain) ogrendikce, basta hic bilmedigimiz onsel
dagilimi daha net, bilgili bir sekilde secebiliriz ve bu sonsal dagilimi da
daha olmasi gereken modele daha yaklastirabilir. 

Gaussian Kontrolu

Diyelim ki Gaussian dagilimina sahip oldugunu dusundugumuz $\{ x_i\}$
verilerimiz var. Bu verilerin Gaussian dagilimina uyup uymadigini nasil
kontrol edecegiz? Normal bir dagilimin her veri noktasi icin soyle temsil
edebiliriz,

\[ y_i = \Phi\bigg(\frac{ x_i - \mu}{\sigma}\bigg) \]

Burada $\Phi$ standart Gaussian'i temsil ediyor (detaylar icin {\em
  Istatistik Ders 1}) ve CDF fonksiyonuna tekabul ediyor. CDF fonksiyonunun
ayni zamanda ceyregi (quantile) hesapladigi soylenir, aslinda CDF son
derece detayli bir olasilik degeri verir fakat evet, dolayli yoldan
noktanin hangi ceyrek icine dustugu de gorulecektir.

Simdi bir numara yapalim, iki tarafa ters Gaussian formulunu uygulayalim,
yani $\Phi ^{-1} $.

\[ \Phi ^{-1}(y_i) = \Phi ^{-1}\bigg(\Phi\bigg(\frac{ x_i - \mu}{\sigma}\bigg)\bigg) \]

\[ \Phi ^{-1}(y_i) = \frac{ x_i - \mu}{\sigma}\]

\[  x_i = \Phi^{-1}(y_i) \sigma + \mu  \]

Bu demektir ki elimizdeki verileri $\Phi^{-1}(y_i)$ bazinda grafiklersek,
bu noktalar egimi $\sigma$, baslangici (intercept) $\mu$ olan bir duz cizgi
olmalidir. Eger kabaca noktalar duz cizgi olusturmuyorsa, verimizin 
Gaussian dagilima sahip olmadigina karar verebiliriz. 

Ustte tarif edilen grafik,  olasilik grafigi (probability plot) olarak
bilinir. 

Ters Gaussian teorik fonksiyonunu burada vermeyecegiz, Scipy
\verb!scipy.stats.invgauss! hesaplar icin kullanilabilir. Fakat $y_i$'nin
kendisi nereden geliyor? Eger $y_i$, CDF'in bir sonucu ise, pur veriye
bakarak bir CDF degeri de hesaplayabilmemiz gerekir. Bunu yapmak icin bir
baska numara lazim. 

1. Eldeki sayilari artan sekilde siralayin

2. Her veri noktasina bir derece (rank) atayin (siralama sonrasi hangi
seviyede oldugu yeterli, 1'den baslayarak). 

3. Ceyrek degeri $y_i$ bu sira / $n+1$, $n$ eldeki verinin buyuklugu. 

Bu teknik niye isliyor? $x$'in CDF'i $x_i < x$ sartina uyan $x_i$'lerin
orani degil midir? Yani bir siralama soz konusu ve ustteki teknik te bu
siralamayi biz elle yapmis olduk, ve bu siralamadan gereken bilgiyi aldik. 


Kaynaklar

[1] http://en.wikipedia.org/wiki/Confidence\_interval

[2] Janert, P., Data Analysis with Open Source Tools

\end{document}
