\documentclass[12pt,fleqn]{article}\usepackage{../common}
\begin{document}
Ders 4

Beklenti (Expectation) 

Bu deger dagilim $f(x)$'in tek sayilik bir ozetidir. Yani beklenti hesabina
bir taraftan bir dagilim fonksiyonu girer, diger taraftan tek bir sayi
disari cikar. Surekli dagilim fonksiyonlari icin $E(X)$

\[  E(X) = \int x f(x) dx\]

ayriksal durumda

\[ E(X) = \sum_x xf(x) \]

olarak hesaplanir. Hesabin, her $x$ degerini onun olasiligi ile carpip
topladigina dikkat. Bu tur bir hesap dogal olarak tum $x$'lerin
ortalamasini verecektir, ve dolayli olarak dagilimin ortalamasini
hesaplayacaktir. Ortalama $\mu_x$ olarak ta gosterilebilir.

Notasyonel basitlik icin ustteki toplam / entegral yerine 

\[ = \int x \ dF(x) \]

diyecegiz, bu notasyonel bir kullanim sadece, unutmayalim, reel analizde
$\int x \ dF(x)$'in ozel bir anlami var (hoca tam diferansiyel $dF$'den
bahsediyor). 

Ornek 

$X \sim Unif(-1,3)$ olsun. $E(X) = \int xdF(x) = \int x f_X(x)dx = \frac{
  1}{4} \int _{ -1}^{3} x dx = 1$. 














\end{document}
