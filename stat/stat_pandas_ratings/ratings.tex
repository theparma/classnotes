\documentclass[12pt,fleqn]{article}\usepackage{../common}
\begin{document}
Pivotlama

Ornek olarak film isimleri ve o filmlere verilmis begeni notlarini
tasiyan bir veri tabanini isleyecegiz. Verimiz uc ayri dosyaya
yayilmis halde. Uc tabloyu alttaki sekilde, \verb!merge! komutu
ile birlestiriyoruz - Pandas otomatik olarak ortak kolon ismini
bulacak ve onun uzerinden birlestirimi yapacak.

\begin{minted}[fontsize=\footnotesize]{python}
import pandas as pd
unames = ['user_id', 'gender', 'age', 'occupation', 'zip']
users = pd.read_table('users.dat', sep='::', header=None,names=unames)
rnames = ['user_id', 'movie_id', 'rating', 'timestamp']
ratings = pd.read_table('ratings.dat', sep='::', header=None,names=rnames)
mnames = ['movie_id', 'title', 'genres']
movies = pd.read_table('movies.dat', sep='::', header=None,names=mnames)
data = pd.merge(pd.merge(ratings, users), movies)
print data
\end{minted}

\begin{verbatim}
<class 'pandas.core.frame.DataFrame'>
Int64Index: 1000209 entries, 0 to 1000208
Data columns (total 10 columns):
user_id       1000209  non-null values
movie_id      1000209  non-null values
rating        1000209  non-null values
timestamp     1000209  non-null values
gender        1000209  non-null values
age           1000209  non-null values
occupation    1000209  non-null values
zip           1000209  non-null values
title         1000209  non-null values
genres        1000209  non-null values
dtypes: int64(6), object(4)
\end{verbatim}

Eger erkeklerin en cok sevdigi ama kadinlarin en az sevdigi (ve
hanimlar icin tam tersi olan) filmleri bulmak istiyorsak, bu islemi
nasil yapariz? Bu islemi Pandas ile yapmak icin ilginc bir takla
atacagiz. "Bir grubun en cok digerinin en az" sorusu, onlarin bir
filme verdigi ortalama notun farkinin en buyuk olmasi demektir. Bunu
dusunebilmek onemli.

Ikinci olarak bu islemin kodlamasi icin ne gerekir? Bir cikartma
islemi lazim. Ideal olarak bir kolonu (ya da satiri) digerinden
cikartmak - bu tur toptan islemler zaten Pandas ile cok hizli.

Fakat verimiz halen o formatta degil.  Her satir, tek bir film, tek
bir kisi (cinsiyet) ve tek bir not icin kaydedilmis. Bizim
ilgilendigimiz analiz icin biz film bazinda icin cinsiyet verisini
{\em yanyana, degisik kolonlarda} gormeliyiz.

Peki nasil? Cevap pivotlamak.

Pivotlamak bir kolonu (hatta birkac kolonu) alip onu x ekseni yapmak,
ayni sekilde bir (veya birkac) kolonu y ekseni yapmak anlamina
gelir. Yani bir kolon uzerindeki tum degerler okunur, ve kordinatmis
gibi o eksene yayilir. Ayni sekilde diger kordinat halledilir. Daha
sonra bu iki kordinattaki kesisim degerleri icin bir ucuncu numerik
kolon secilir (ve onun uzerinden ek bir numerik islem de
tanimlanabilir), ve boylece pivotlama gerceklesmis olur.

Bizim pivot icin cinsiyet kolona yayilacak, film ismi satira
yayilacak. Kesisim ise not ortalamasi (rating mean) olacak.

\begin{minted}[fontsize=\footnotesize]{python}
mean_ratings = data.pivot_table('rating', rows='title', cols='gender',
                                aggfunc='mean')
print mean_ratings[:5]
\end{minted}

\begin{verbatim}
gender                                F         M
title                                            
$1,000,000 Duck (1971)         3.375000  2.761905
'Night Mother (1986)           3.388889  3.352941
'Til There Was You (1997)      2.675676  2.733333
'burbs, The (1989)             2.793478  2.962085
...And Justice for All (1979)  3.828571  3.689024
\end{verbatim}

Daha fazla ilerlemeden ufak bir ek islem daha yapalim, 250'den daha az
not almis olan filmleri eleyelim.

\begin{minted}[fontsize=\footnotesize]{python}
ratings_by_title = data.groupby('title').size()
active_titles = ratings_by_title.index[ratings_by_title >= 250]
print active_titles[:10]
\end{minted}

\begin{verbatim}
Index([u''burbs, The (1989)', u'10 Things I Hate About You (1999)', u'101 Dalmatians (1961)', u'101 Dalmatians (1996)', u'12 Angry Men (1957)', u'13th Warrior, The (1999)', u'2 Days in the Valley (1996)', u'20,000 Leagues Under the Sea (1954)', u'2001: A Space Odyssey (1968)', u'2010 (1984)'], dtype=object)
\end{verbatim}

Yapilan harekete dikkat: \verb!ratings_by_title.index! uzerinde bir boolean
filtreleme yaptik, yani \verb![True, False..., True]!  gibi bir
filtreleyiciyi \verb!Index! {\em objesi} uzerinde kullandik. Bu niye
isledi? Cunku \verb!.index! cagrisi da sonucta bir dizindir, ve dizinler
uzerinde istenen boolean filtrelemesi yapilabilir (her iki taraf ta ayni
boyutta oldugu surece).

Devam edelim, simdi ortalama notlari ustteki yeni Index'e gore
azaltalim (ve \verb!.ix! kullanacagiz, cunku Index objesi
satirlar uzerinde islem yapar ve \verb!.ix! cagrisi satirlara
erismek icin kullanilir), ve hanimlarin en cok sevdigi filmlere
bakalim,

\begin{minted}[fontsize=\footnotesize]{python}
mean_ratings = mean_ratings.ix[active_titles]
top_female_ratings = mean_ratings.sort_index(by='F', ascending=False)
print top_female_ratings[:4]
\end{minted}

\begin{verbatim}
gender                                                         F         M
title                                                                     
Close Shave, A (1995)                                   4.644444  4.473795
Wrong Trousers, The (1993)                              4.588235  4.478261
Sunset Blvd. (a.k.a. Sunset Boulevard) (1950)           4.572650  4.464589
Wallace & Gromit: The Best of Aardman Animation (1996)  4.563107  4.385075
\end{verbatim}

Baylara pek tanidik gelmeyen bir liste. Simdi erkekler ve hanimlar
begeni farkini hesaplayalim ve en buyuk farklar en ustte olacak
sekilde siralama (sort) yapalim,

\begin{minted}[fontsize=\footnotesize]{python}
mean_ratings['diff'] = mean_ratings['M'] - mean_ratings['F']
sorted_by_diff = mean_ratings.sort_index(by='diff')
print sorted_by_diff[:6] 
\end{minted}

\begin{verbatim}
gender                            F         M      diff
title                                                  
Dirty Dancing (1987)       3.790378  2.959596 -0.830782
Jumpin' Jack Flash (1986)  3.254717  2.578358 -0.676359
Grease (1978)              3.975265  3.367041 -0.608224
Little Women (1994)        3.870588  3.321739 -0.548849
Steel Magnolias (1989)     3.901734  3.365957 -0.535777
Anastasia (1997)           3.800000  3.281609 -0.518391
\end{verbatim}

{\em Dirty Dancing}, {\em Grease} gibi romantik filmler ustte cikti. Simdi
listeyi ters cevirelim ve en alta bakalim, orada baylarin en cok
hanimlarin en az sevdigi filmler olmali,

\begin{minted}[fontsize=\footnotesize]{python}
print sorted_by_diff[::-1][:15]
\end{minted}

\begin{verbatim}
gender                                         F         M      diff
title                                                               
Good, The Bad and The Ugly, The (1966)  3.494949  4.221300  0.726351
Kentucky Fried Movie, The (1977)        2.878788  3.555147  0.676359
Dumb & Dumber (1994)                    2.697987  3.336595  0.638608
Longest Day, The (1962)                 3.411765  4.031447  0.619682
Cable Guy, The (1996)                   2.250000  2.863787  0.613787
Evil Dead II (Dead By Dawn) (1987)      3.297297  3.909283  0.611985
Hidden, The (1987)                      3.137931  3.745098  0.607167
Rocky III (1982)                        2.361702  2.943503  0.581801
Caddyshack (1980)                       3.396135  3.969737  0.573602
For a Few Dollars More (1965)           3.409091  3.953795  0.544704
Porky's (1981)                          2.296875  2.836364  0.539489
Animal House (1978)                     3.628906  4.167192  0.538286
Exorcist, The (1973)                    3.537634  4.067239  0.529605
Fright Night (1985)                     2.973684  3.500000  0.526316
Barb Wire (1996)                        1.585366  2.100386  0.515020
\end{verbatim}

Burada da {\em Good, The Bad and The Ugly} gibi kovboy filmleri, ve buna
benzer vurdulu kirdili filmler ya da enseye tokat turunden {\em Aptal ve Daha
Aptal (Dumb \& Dumber)} gibi filmler cikti. Ilginc bir analiz oldu. :)

Burada takip edilen mantiga, ve onun nasil Pandas islemlerina
cevirildigine dikkat. "X grubunun en cok ama Y grubunun en az" turunde
bir sorgu bir aritmetik fark hesabina cevrildi ve bir grup icin onemli
olan kalemlerin en ustte, digeri icin en onemli olanin en altta
olacagi akil edildi (en altta eksi degerler vardi tabii ki, bunun
sebebini iyi dusunelim) ve sonuca varildi.

Yapay Ogrenim engin bir alandir, ama regresyon, siniflama gibi
islemlerden once hala yapilabilecek ilginc ve onemli, ustteki gibi
veri analizler var.

Kaynak

McKinney, W., Python for Data Analysis



\end{document}
