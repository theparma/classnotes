\documentclass[12pt,fleqn]{article}\usepackage{../common}
\begin{document}
Lineer Cebir - Ders 1

Ilk dersimize hosgeldiniz, ben Gilbert Strang. Lineer cebirin cozmeye
calistigi en temel problem bir lineer denklem sistemini cozmektir. Bu
baglamda mesela en genel durum ``bilinmeyen ve denklem sayisinin birbirine
esit oldugu'' durumdur, ki bu ``guzel durum'' olarak nitelenebilir. 

Devam edersek, dersin kavramlarini anlamak icin ``satir bakisi''na
basvuracagiz, bu durumda her denkleme teker teker bakiyoruz gibi olacak,
dersimizde pek cok kez kullanacagimiz $A$ matrisimizde satirlarin her denklemin
degiskenlerine tekabul ettigi dusunulur cogunlukla. 

Kolon bakis acisi belki daha once gormediginiz bir aci olacak, bu durumda
her kolon ayri ayri islenecek. 

Cebirsel bakis acisi ise tum matrisi, bu durumda $A$, ayni anda ele alir. 

Guzel durumdan baslayalim - iki bilinmeyen, iki denklem.

$$ 2x - y = 0 $$

$$ -x + 2y = 3 $$










\end{document}
