\documentclass[12pt,fleqn]{article}\usepackage{../common}
\begin{document}
Ders 21

Bu ozdeger/vektorler hakkindaki ilk dersimiz. Bu degerler ozel
buyukluklerdir, ozel sayilardir, ve onlari niye istedigimizi, niye
hesapladigimizi gorecegiz.

Ozvektor nedir? 

Elimde bir $A$ matrisi var. Bir matris ne yapar? Vektorler uzerinde, mesela
$x$ vektoru, etkide bulunabilir, onlari degistirebilir. Sanki $A$'yi bir
fonksiyon gibi de gorebiliriz, $x$ vektoru $A$'ya ``giriyor'' ve
``disariya'' bir $Ax$ cikiyor. Calculus'ta oldugu gibi $f()$'e bir tek sayi
$x$ veriliyor, $f(x)$ geri disari cikiyor [hakikaten de $x$ ve $Ax$ ayni
boyutta yani giris cikis analojisi cok uygun]. Lineer Cebirde daha cok
boyut var, giren ve cikan vektorler.

Bu derste ozellikle ilgilendigim vektorler ise disari ciktigi zaman girdigi
haliyle {\em ayni yonu gosteren} vektorler. Dikkat, ``ayni'' olan vektorler
degil cikinca ayni ``yonu'' gosteren vektorler. Bu tipik bir durum olmazdi
degil mi? Cogunlukla $A$'yi bir uyguladik mi disari cikan vektor tamamen
baska bir yonu gosterir. Bizim ilgilendigimiz durumda oyle olmayacak, bu
durumda $Ax$, $x$'e paralel olacak. Iste bu vektorler ozvektorler olacak. 

Paralel ne demektir? Formulle daha rahat belirtilir,

$$ Ax = \lambda x $$

$\lambda$, yani ozdeger, bir skalardir. Iki tarafta $x$'in olmasi
paralellige isaret ediyor, sadece buyukluk ($\lambda$ uzerinden) degisik
olabiliyor. Tabii buyukluk derken $\lambda$ eksi degerde olabilecegi icin
vektorun ters yonde olmasina da izin vermis oluyoruz. $\lambda$ sifir da
olabilir, hatta hayali sayi bile olabilir.

Ozdeger sifir uzerinde biraz daha duralim. Bu durumda $Ax = 0 \cdot x$ elde
ederiz yani $Ax = 0$. Bu ne demektir? $x$'lerin $A$'nin null uzayinda
(nullspace) olmasi... Eger $A$ tekil (singular) ise, ki $Ax = 0$ bu demek
zaten demek ki oyle bir $x$ olabiliyor ki $Ax = 0$ olabiliyor, o zaman $x$
sifir olmayan bir vektordur, ve $\lambda = 0$ bir ozdeger olmalidir.

Bir projeksiyon matrisine bakalim, mesela $P$. Elimizde bir satih (plane)
var, ve bu sathin uzerinde yansitma yapan bir $P$ var. 

\includegraphics[height=4cm]{21_1.png}

$b$, $P$'nin bir ozvektoru mudur? Degildir. Cunku $b$ ve $Pb$ ayni yonu
gostermiyorlar. 

Peki, bu resme gore, yansitma sonrasi ayni yonde olacak bir vektor var
midir? Varsa nerededir? Cevap, eger $x$ ustteki duzlemin tam uzerinde ise
$P$ yansitmasi sonrasi ayni yonde kalir. Tabii yansitma tekrar kendisini
verir, yani vektor hic degismemis olur. $Px = x$, ki $\lambda = 1$.

Baska bir ozvektor var mi? Olmasini umuyorum cunku 3 boyuttayim ve bu
demektir ki 2 tane daha birbirinden bagimsiz ozvektor bulabilmeliyim, ki
nihayetinde ozvektorlerin ikisi duzlem uzerinde [duzlem iki boyutlu bir sey
olduguna gore), o zaman ucuncusu duzlem disinda olacak. Duzlem disinda olan
ozvektor dik olan ozvektor olmali.

\includegraphics[height=4cm]{21_2.png}

Bu durumda $Px = 0$, ve $\lambda = 0$, cunku dikligin bir diger tanimi
carpim sonrasi sonucun sifir olmasi. 

Bir diger ornek. Su permutasyon matrisine bakalim. 

$$ 
A = \left[\begin{array}{cc}
0 & 1 \\
1 & 0
\end{array}\right]
 $$

 Bu matrisi hangi vektor ile carparsam ayni yonde bir vektor elde ederim?
 Permutasyon matrisi tanim itibariyle permutasyon yapar, yani ogelerin
 yerini degistirir. Iki boyut baglaminda bir vektorun iki ogesinin yerini
 degistirecektir. Peki hangi vektorun ogeleri yer degistirirse yine kendisi
 olur? Cevap basit, $x = [1 \ 1]$. 

$$ 
x = \left[\begin{array}{c}
1 \\ 1
\end{array}\right], 
Ax = \left[\begin{array}{c}
1 \\ 1
\end{array}\right], 
\lambda = 1, Ax = x
 $$

Bir tane daha ozdeger/vektor lazim. Bu diger ozdeger $\lambda = -1$
olmali. Peki nasil bir vektor olmali ki ogeleri yer degistirince ters yonu
gostersin? 

$$ 
x = \left[\begin{array}{r}
-1 \\ 1
\end{array}\right], 
Ax = \left[\begin{array}{r}
1 \\ -1
\end{array}\right], 
\lambda = -1, Ax = -x
 $$



14:00












\end{document}
