
\documentclass[12pt,fleqn]{article}\usepackage{../common}
\begin{document}
Lineer Cebir - Ders 7

Vektor uzaylarindan, ozellikle sifir uzayindan (nullspace), ve kolon
uzayindan bahsettik, simdi bu uzaylarin icindeki vektorleri nasil
bulacagimizi, nasil hesaplayacagimizi gorecegiz. Yani onceki derste
gordugumuz tanimlari bu derste algoritmaya donusturecegiz. $Ax=0$'i cozen
algoritma nedir, mesela. Ornek uzerinde gorelim,

$$ 
A = 
\left[\begin{array}{cccc}
1 & 2 & 2 & 2  \\
2 & 4 & 6 & 8 \\
3 & 6 & 8 & 10
\end{array}\right]
 $$

Ilk bakista gozume carpan 2. kolon 1. kolonun bir kati. Ya da 2. kolon
1. ile ``ayni yonde'', bu iki kolon ``bagimsiz degil''. Tabii bu bilgileri
cozum sirasinda da algoritmanin bir yan etkisi olarak kesfetmeyi
bekleriz. Satirlara bakiyorum, 1. ve 2. toplami 3. ile ayni, yani 3. satir
bagimsiz degil. Tum bunlar eliminasyonun yan urunleri olarak bulunmalilar.

Ana algoritmamiz eliminasyon olacak, ama onun dikdortgensel kosula adapte
edilmis hali, pivotta sifir var ise durmadan cozume devam ediyoruz, vs.

Eliminasyon sirasinda yapilan islemler sifir uzayini degistirmez. Degil mi?
Bu onemli. Bir denklem sisteminde bir denklemin (satirin) bir katini bir
diger denklemden cikartiyorsam bu nihai cozumu degistirmez, cunku denklem
sistemi bir butun olarak degismemistir. Satir uzayi degismez, ama dikkat,
kolon uzayi degisir. Eliminasyonun neyi degistirdigi neyi degistirmedigini
bilmek lazim. 

Ilk pivot'tan baslayalim (yesil ile isaretli)

$$ 
\left[\begin{array}{cccc}
\textcolor{green}{1} & 2 & 2 & 2  \\
2 & 4 & 6 & 8 \\
3 & 6 & 8 & 10
\end{array}\right]
 $$

Pivot satirini 2 ile carpip 2. satirdan cikartiyoruz. Sonra pivot satirini
3 ile carpip 3. satirdan cikartiyoruz. Sonuc

$$ 
\left[\begin{array}{cccc}
1 & 2 & 2 & 2  \\
0 & 0 & 2 & 4 \\
0 & 0 & 2 & 4
\end{array}\right]
 $$

Simdi sonraki pivot'u ariyoruz, normal durumda bu pivot

$$ 
\left[\begin{array}{rrrr}
1 & 2 & 2 & 2  \\
0 & \textcolor{green}{0} & 2 & 4 \\
0 & 0 & 2 & 4
\end{array}\right]
 $$

olurdu. Ama orada sifir var, o zaman bir alttaki satira bakiyoruz, ki
umuyoruz ki satir degis-tokusu yaparak o noktaya sifir olmayan bir deger
gelsin. Ama mustakbel pivot'un altindaki hucre de sifir degerini tasiyor!
Bu bir seyin isareti aslinda.. Neyin? Bu baktigimiz kolonun kendinden once
gelen kolonlarin bir kombinasyonu oldugunun isareti. Fakat bunun uzerinde
fazla durmaya gerek yok, algoritmik olarak durup dusunmeye gerek yok, eger
bir pivot'u kullanamiyorsak hemen yana geceriz, yani

$$ 
\left[\begin{array}{rrrr}
1 & 2 & 2 & 2  \\
0 & 0 & \textcolor{green}{2} & 4 \\
0 & 0 & 2 & 4
\end{array}\right]
 $$

Simdi 2. satiri 3. satirdan cikartmak yeterli. Sonuc altta. Bu matrise $U$
diyebiliriz, gerci tam ustucgensel (uppertriangular) sayilmaz, cunku
sifir dengesi tam degil, ama hafiften ustucgensel. Tum pivot'lari
gosterirsek, 

$$ 
\left[\begin{array}{rrrr}
\textcolor{green}{1} & 2 & 2 & 2  \\
0 & 0 & \textcolor{green}{2} & 4 \\
0 & 0 & 0 & 0
\end{array}\right]
 $$

Iki tane pivot var; ve bu iki sayisi bu matris hakkinda onemli bir bulguya
isaret ediyor, bu matrisin kertesi (rank) 2. Kerte bir matrisin pivot
sayisidir. 



























\end{document}
