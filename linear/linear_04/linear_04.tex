\documentclass[12pt,fleqn]{article}\usepackage{../common}
\begin{document}
Ders 4

Bu derste yapacaklarimiz sunlar; bir $AB$ carpiminin tersini (inverse)
nasil alirim? Biliyoruz ki 

\[ AA ^{-1}  = I = A ^{-1} A \]

Soru su ki 

\[ AB... = I \]

noktali kisma ne gelirse sonuc birim (identity) matris olur? Sonra bu
bilgiyi bir baska carpim uzerinde kullanacagiz, bu carpim eliminasyon
matrislerini bir sira carpim olarak gorecek, ve bu sekilde Gaussian
eliminasyon islemine degisik bir bakis getirmis olacagiz. 

$AB$'den birim matrise nasil erisirim? 

\[ ABB ^{-1} A ^{-1} = I \]

Ya da

\[ B ^{-1}A ^{-1} AB  = I\]

Simdi tersini alma isleminin devrigini alma ile nasil isleyecegine
bakalim. Diyelim ki 

\[ AA ^{-1} = I \]

Bunun tersini alirsam ne olur ?

\[ (A ^{-1})^T A^T = I \]

Devrik alinca siralama degisiyor bildigimiz gibi, ve birim matrisin devrigi
yine kendisi. 

Fakat ustteki ifade bir seyi daha soyluyor: $A^T$'yi ne carparsa sonuc $I$
gelir? Cevap $A^T$'nin tersi! Yani $(A ^{-1})^T$ ve  $(A^T) ^{-1}$
ifadelerinin ayni sey oldugunu soylemis oluyoruz. Bu nasil kullanilabilir?
Eger $A^T$'nin tersini hesaplamamiz gerekiyorsa, ve $A$'nin tersini bir
sekilde biliyorsam, onun devrigini almam yeterli. Diger bir deyisle,
tersini alma ve devrigini alma islemleri herhangi bir sirada yapilabilir. 















\end{document}
