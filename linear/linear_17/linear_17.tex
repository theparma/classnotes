\documentclass[12pt,fleqn]{article}\usepackage{../common}
\begin{document}
Ders 17

Bugun dikeylik / ortogonallik (orthogonality) ders dizisinin
sonuncusundayiz. Ortogonal vektorleri, iki tanesini, gorduk, ortogonal
altuzaylari gorduk, ki bunlar satir uzayi ve null uzayi idi, bugun
ortogonal baz ve ortogonal matrisi gorecegiz. 

Simdi ortonormal (orthonormal) kelimesinden bahsetmek istiyorum. Bu arada
bu derte $q$ harfini ortogonal vektorleri temsil etmek icin kullanacagim. 

Ortonormal vektorler 

\[ 
q_i^Tq_j = \left\{ \begin{array}{lll}
0 & eger & i \ne j \ ise\\
1 & eger & i = j \ ise
\end{array} \right.
 \]

Yani $q$ vektorleri diger her $q$'ya (kendisi haricinde) ortogonal. Her
biri bir digerine 90 derece dik vektorlerden olusan bir baz olmasi dogal
bir sey. $q$ vektorleri birim, bu sebeple kendisiyle noktasal carpimi
1. Ortonorma kelimesindeki ``normal'' buradan geliyor, normalize edilmis
vektorlerimiz var. 

Ortonormal baza sahip olmak hesaplari basitlestirir, cogu hesabi
iyilestirir, sayisal lineer cebirin cogunlugu ortonormal vektorlerle is
yapmak etrafinda kurulmustur, cunku onlar asiri buyumezler, asiri
kuculmezler, kontrol altinda is yapmak mumkun olur. 

Bu $q$'leri $Q$ icine koyacagiz. Dersin ikinci kisminda eger ortonormal bir
$A$ matrisim var ise, onu nasil ortonormal yaparim, onu gorecegiz. Simdi
ustteki 1 ve 0 iceren formulu matris olarak yazmak istiyorum. 

\[ 
Q = 
\left[\begin{array}{rrr}
\uparrow &  & \uparrow \\
q_1 & ... &  q_n \\
\downarrow &  & \downarrow 
\end{array}\right]
 \]

O zaman 

\[ 
Q^TQ = 
\left[\begin{array}{rrr}
\leftarrow & q_1^T & \rightarrow \\
& &  \\
\leftarrow & q_n^T & \rightarrow 
\end{array}\right]
\left[\begin{array}{rrr}
\uparrow &  & \uparrow \\
q_1 & ... &  q_n \\
\downarrow &  & \downarrow 
\end{array}\right] = 
I
 \]

Eger $Q$ kare ise, $Q^TQ = I$  bize $Q^T = Q^{-1}$ oldugunu soyler. 

Ornek 

Her permutasyon matrisi 

\[ 
Q = 
\left[\begin{array}{rrr}
0 & 0 & 1 \\
1 & 0 & 0 \\
0 & 1 & 0 
\end{array}\right]
 \]

kendi devrigi ile carpilinca, yani 

\[ 
Q^T = 
\left[\begin{array}{rrr}
0 & 1 & 0 \\
0 & 0 & 1 \\
1 & 0 & 0 
\end{array}\right]
 \]

ile, sonuc $I$ olacaktir. Bir diger ornek 

\[ Q = 
\left[\begin{array}{cc}
cos \theta & -sin \theta \\
sin \theta & cos \theta \\
\end{array}\right]
 \]


Ornek

\[ Q = 
\left[\begin{array}{cc}
1 & 1 \\
1 & -1 \\
\end{array}\right]
 \]

ortogonal matris degildir (simdilik). Her kolonun uzunlugu $\sqrt{2}$, o
zaman tum matrisi $\sqrt{2}$ bolerim,

\[ Q = \frac{1}{\sqrt{2}}
\left[\begin{array}{cc}
1 & 1 \\
1 & -1 \\
\end{array}\right]
 \]

Ornek 

\[ Q = \frac{ 1}{2}
\left[\begin{array}{rrrr}
1 & 1& 1& 1 \\
1 & -1& 1& -1 \\
1 & 1& -1& -1 \\
1 & -1& -1& 1 
\end{array}\right]
 \]

Ornek 

\[ Q = \frac{ 1}{3}
\left[\begin{array}{rr}
1 & -2 \\
2 & -1 \\
2 & 2 
\end{array}\right]
 \]

Bir kolon daha eklersem 

\[ Q = \frac{ 1}{3}
\left[\begin{array}{rrr}
1 & -2 & 2\\
2 & -1 & -2 \\
2 & 2 & 1
\end{array}\right]
 \]

Ortonormal vektorler ve ozelde Gram-Schmidt ile ugrasirken gorecegiz,
surekli normal vektorlerle ugrastigimiz icin surekli uzunluga bolmeniz
gerekir, ve bu surekli bir karekoku hesabin icine ceker. Ustteki ornek
temiz, karekok direk sayi olarak elde edildi. 

Ortogonal matrisler niye iyidir? Onlar hangi hesaplari basitlestirirler?
$Q$ ne icin kullanilir? 

Diyelim ki bir diger matrisi alip $Q$'nun kolon uzayina yansitmak
(projection) istiyorum. Yansitma formulunu, matrisi $Q$ icin yazarsak,

\[ P = Q (Q^TQ)^{-1}Q^T \]

Formulde $Q$ olmasinin bir avantaji, $Q^TQ = I$, o zaman 

\[ P = Q Q^T \]

Bu yansitma matrisinin ozelliklerini / ogelerini (properties) kontrol
edelim. Iki tane ozellik olmasi gerekiyor. Eger kolonlar ortonormal ise, ve
yansitma matrisi kare ise, o zaman kolon uzayi nedir? Tum uzaydir!
Ortonormal kolonlar o uzayi yaratmak / kapsamak (span) icin
yeterlidir. Bu uzaydaki her turlu vektoru bu ortonormal vektorlerin bir
kombinasyonu uzerinen uretebilirsiniz. Peki, tum uzaya yansitmak ne
demektir? Nasil bir $P$ tum uzaya yansitir? Birim (identity) matrisi. Tum
uzaya yansitmak hicbir seyi degistirmemek demektir aslinda, ve bu
degismezligi yapacak tek yansitma matrisi $I$. 

\[ P = Q Q^T = I \textit{ eger I kare ise }\]

Bir diger ozellik yansitma matrislerinin simetrik olmasi. Devrigi ile
sagdan carpilmak bir matrisi zaten simetrik yapar, sart $QQ^T$ icin
gecerlidir. Bu ozelligin sebebi yansitip, sonra tekrar yansitinca, ikinci
yansitmanin degisim yaratmamasi. Yani 

\[ (QQ^T)(QQ^T) = QQ^T \]

olmali. Kontrol edelim,

\[ Q(Q^TQ)Q^T = QIQ^T = QQ^T\]

$Q$'nun ozellikleri bize su sekilde de yarar. Normal formul nedir? 

\[ A^TA\hat{x} = A^Tb \]

$A$ yerine $Q$ ise

\[ Q^TQ\hat{x} = Q^Tb \]

$Q^TQ = I$ olduguna gore, 

\[ \hat{x} = Q^Tb \]

Isler iyice basitlesti yani. $A$'li versiyonda sol taraftaki ic carpimlari
(inner product) hesaplamak gerekecekti, cebirsel cozum icin ugrasilacakti,
vs. Bunlar $Q$ ile yok. $\hat{x}$'in $i$'inci elemani $q_i^Tb$, yani 

\[ \hat{x}_i = q_i^Tb \]

Ustteki formulun matematigin en onemli formullerinden biri oldugu
soylenebilir. Yani elimizde ortonormal baz var ise, yansimaninin $i$'inci
ogesi yukaridaki gibidir. 

















\end{document}
