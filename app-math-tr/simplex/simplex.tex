\documentclass[12pt,fleqn]{article}\usepackage{../common}
\begin{document}
Lineer Optimizasyon ve Simplex

Simplex algoritmasi lineer optimizasyon alaninda kullanilan bir
algoritma. Simplex, George Dantzig tarafindan icat edildi, ve 2. Dunya
Savasi sirasinda Berlin'e Hava Ikmali (Berlin Airlift) adli yardim
operasyonunda yogun bir sekilde kullanildi. Elde s�n�rl� miktarda ucak,
kargo kapasitesi vardi ve diger bazi k�s�tlamalar (constraints) da goz
onune alinarak, durum bir lineer programa verildi ve optimal seferler
planlandi.

Problem soyleydi:

24 Haziran 1948'te Sovyetler Birligi Dogu Almanya'dan Berlin'e giden tum
kara ve deniz yollarini tikadi. Bu yuzden Berlin'de yasayan 2. milyon
insana yiyecek, giyim, vb. esyalari nakil edebilmek icin Amerikali ve
Ingiliz ucaklarindan olusan dev bir nakliyat operasyonu
planlandi. 

Bir Amerikali ucagin kargo kapasitesi 30,000 $\textrm{feet}^3$, Ingiliz
ucaginin kargo kapasitesi 20,0000 $\textrm{feet}^3$ idi. Sovyetlerin
engellemelerini etkili bir sekilde asabilmek icin muttefik gucler
tasidiklari yuku maksimize etmek zorundaydilar. Diger k�s�tlamalar
soyleydi: En fazla 44 ucak kullanilabilecekti. Daha buyuk Amerikan
ucaklarini ucurmak icin 16 kisilik bir ekip gerekiyordu, Ingiliz ucaklari
icin 8 kisi gerekiyordu. Kullanilabilecek elde olan ekipler toplam 512
kisiydi. Amerikan ucaginin her ucusu \$9000, Ingiliz ucagin her ucusu
\$5000 idi. Ve nihayetinde haftalik masraf toplam olarak \$300,000'i
gecemeyecekti. 

Bu bir lineer optimizasyon problemidir. Cozmek icin su sekilde belirtmek
gerekir:

\[ \textrm{maksimize et  } 30000x + 20000y, \textrm{  oyle ki} \] 

\[ x + y \le 44 \]

\[ 16x + 8y \le 512 \]

\[ 9000x + 5000y \le 300000 \]

sartlari gecerli olsun. 

Ekteki program \verb!lp.py! ile bu problemi cozebiliriz. 

\lstinputlisting[language=Python]{berlin.py}

Sonucu basinca 

\begin{lstlisting}[language=Python]
1080000.0
[ 20.  24.]
\end{lstlisting}

ekrana gelecek. Yani hesap (cost) adi verilen hedef fonksiyonu kargo
buyuklugunun 1080000.0 oldugu noktada maksimize oldu (haftada en fazla bu
kadar kargo tasinabilecek), ve bu optimal nokta icin $x=20$, $y=24$
olmali. Demek ki optimal bir Berlin ikmal operasyonu icin 20 Amerikali, ve
24 Ingiliz ucagi kullanmak gerekiyor.

Bazi ek bilgiler: ustteki problemin belirten kitaplar, makalelerde ``44
ucak kullanimindan'' bahsediliyor, fakat eldeki ucak mi, oyleyse gunde,
haftada, ayda ne kadar havalandirilabileceklerinden bahsedilmiyor. Buyuk
bir ihtimalle 44 bir hafta icinde havada olabilecek ucak sayisi, bir nevi
ucus koridoru, ya da seyahati. 

Dantzig hakkinda da ilginc hikayelerden bir sudur: Doktorasini yaptigi
sirada ogrenciyken bir istatistik dersine gec girer. Hoca, tahtaya bazi
problemler yazmistir, Dantzig bu problemleri odev problemi olarak not
eder. Birkac hafta sonra hocayi evinde bulur, "hocam, bu odev problemleri
cok agirmis, gunlerce ugrastim, ama cozdum" diyerek odev cozumlerini
verir. Hocasi o problemlerin odev degil, istatistikte simdiye kadar
cozulemeyen problemler oldugunu o zaman soyler. :) Dantzig farkinda olmadan
birkac hafta icinde aslinda ciddi bir tez arastirmasi yapmistir.

Aslinda bu hikayede psikolojik bir boyut ta var. Dantzig problemi ``bir odev
olarak verildigi icin cozmesi beklendigini'' dusundugu icin mi cozmustur?
Belki de. Bu hikaye Manuel Blum'un doktora hakkinda soylediklerini
cagristiriyor (bkz doktora yazisi).

Kaynaklar

http://projects.scipy.org/scipy/attachment/ticket/1252/lp.py


\end{document}

