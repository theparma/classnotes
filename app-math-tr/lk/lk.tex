\documentclass[12pt,fleqn]{article}
\setlength{\parindent}{0pt}
\usepackage{graphicx}
\usepackage{cancel}
\usepackage{listings}
\usepackage[latin5]{inputenc}
\setlength{\parskip}{8pt}
\setlength{\parsep}{0pt}
\setlength{\headsep}{0pt}
\setlength{\topskip}{0pt}
\setlength{\topmargin}{0pt}
\setlength{\topsep}{0pt}
\setlength{\partopsep}{0pt}
\setlength{\mathindent}{0cm}

\begin{document}
Piksel Takibi, Optik Akis, Lucas Kanade Algoritmasi

Hareket halindeki bir kameranin aldigi goruntulerdeki herhangi bir pikseli
nasil takip ederiz? 

Matematiksel olarak temsil etmek gerekirse, zamana gore degisen 2 boyutlu
goruntuyu bir fonksiyon olarak dusunelim, ki bu fonksiyonun degerleri
ayriksal olarak, imajin ta kendisi. Bir $I(x(t),y(t),t)$ fonksiyonu piksel
degerlerini veriyor. Bu fonksiyonda $x,y$ ekran kordinatlarina tekabul
ediyor, $t$ ise zaman, 1,2,.. gibi degerleri var, mesela $I(100,200,1)$,
bize 1. fotograftaki $x=100,y=200$ kordinatlarindaki piksel degerini
verecek. 

$x,y$ degiskenleri parametrize edildi, bir noktayi takip etmek istiyoruz
cunku, ve $t$'ye gore bu takip edilen noktanin $x,y$ kordinatlari belli bir
gidisat yonunde degisiyor.

O zaman su faraziyeyi yaparak problemimizi kolaylastirabiliriz. Diyelim ki
takip edilen bir nokta, goruldugu her karede ayni piksel rengindedir. Bu
cok siradisi bir faraziye degil, resim karelerinden bir araba geciyor, ve
bu arabanin uzerindeki piksellerin renkleri, en azindan iki kare arasinda
degismiyor. Isik seviyesi, golgede olma, vs. gibi durumlarda biraz
degisebilir, fakat basitlestirme amaciyla bu faraziye ise yarar.

\includegraphics[height=3cm]{disp2.png}

Bir diger faraziye, kameralar hareket ettiklerinde alinan iki goruntu
arasindaki tum piksellerin yer degisimi genellikle ayni yonde olmasidir. Bu
degisim yonunu $<u,v>$ vektoru olarak gorebiliriz, ve bu degiskenler iki
goruntu arasindaki degisimde tum pikseller icin ayni olacaktir. Kamarayi
saga hareket ettiriyoruz, ve goruntudeki tum pikseller sola dogru
gidiyorlar. 

\includegraphics[height=3cm]{disp.png}

Tum bunlari modelimizde nasil kullaniriz? 

Takip edilen nokta her karede ayni renkte ise, su ifade dogru demektir 

\[ I(x(t),y(t),t) = \textrm{ sabit } \]

Eger bu fonksiyonun zamana gore turevini alirsak

\[ \frac{d \ I(x(t),y(t),t)}{dt} = 0\]

sonucu gelir. Esitligin sagi sifir, cunku bir sabitin turevini aldik. Sol
tarafa Zincirleme Kanununu uygularsak, 

\[ \frac{\partial I}{\partial x}\frac{dx}{dt} +
\frac{\partial I}{\partial y}\frac{dy}{dt} +
\frac{\partial I}{\partial t} = 0
\]

Bu formulde $dx/dt$ ve $dy/dt$, hareket halindeki (zaman gecerken) noktanin
sonsuz kucuklukteki ne kadar yer degimi. Ayriksal baglamda arka arkaya iki
kare icindeki yer degisimi. O zaman,

\[ \frac{dx}{dt}, \frac{dy}{dt} = u, v \]

Alttakiler ise mesafesel (spatial) gradyanlardir, bunlarin nasil
hesaplanacagini cok iyi biliyoruz! 

\[ 
\frac{\partial I}{\partial x}, \frac{\partial I}{\partial y}
 \]

Alttaki ise resim karelerinin zamana gore turevi

\[ 
\frac{\partial I}{\partial t}
 \]

Daha derli toplu olarak gostermek gerekirse ana formul nihai olarak soyle

\[ 
\nabla I \cdot <u, v> = -I_t
 \]


\lstinputlisting[language=Python]{deriv.py}

\lstinputlisting[language=Python]{lk.py}




\end{document}
