\documentclass[12pt,fleqn]{article}\usepackage{../common}
\begin{document}

\inputminted[fontsize=\footnotesize]{python}{rbmp.py}

\inputminted[fontsize=\footnotesize]{python}{test_rbmkfold.py}

\begin{minted}[fontsize=\footnotesize]{python}
! python test_rbmkfold.py
\end{minted}

\begin{verbatim}
0.989898989899
\end{verbatim}

Daha cetrefil bir veri seti MNIST veri setine [2] bakalim. Veri 28x28
boyutunda ikisel veri olarak kodlanmis rakamlarin el yazisindan alinmis
resimlerini icerir. Veriyi aldiktan sonra egitim / test kisimlarinin ilk
1000 tanesi uzerinde algoritmamizi kullanirsak, tek komsulu KNN (yani 1-NN)
yuzde 85.4 basari sonucunu verir. Alttaki parametreler uzerinden RBM yuzde
86 olacaktir.

\inputminted[fontsize=\footnotesize]{python}{test_mnist.py}

[1] \url{http://videolectures.net/icml09_tieleman_ufw/}

[2] \url{http://www.iro.umontreal.ca/~lisa/deep/data/mnist/mnist.pkl.gz}

\end{document}
