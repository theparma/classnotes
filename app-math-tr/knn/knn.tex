\documentclass[12pt,fleqn]{article}\usepackage{../common}
\begin{document}
En Yakin k-Komsu (k-Nearest Neighbor)

Yapay Ogrenim alaninda ornek bazli ogrenen algoritmalardan bilinen
kNN, egitim verinin kendisini siniflama (classification) amacli olarak
kullanir, yeni bir model ortaya cikartmaz. Algoritma soyle isler:
etiketleri bilinen egitim verisi alinir ve bir kenarda tutulur. Yeni
bir veri noktasi xqgorulunce bu veriye geri donulur ve o noktaya ``en
yakin'' k tane nokta bulunur. Daha sonra bu noktalarin etiketlerine
bakilir ve cogunlugun etiketi ne ise, o etiket yeni noktanin etiketi
olarak kabul edilir. Mesela elde \verb!1! kategorisi altinda
\verb![2 2]!, \verb!2! kategorisi altinda \verb![5 5]!
var ise, yeni nokta \verb![3, 3]! icin yakinlik
acisindan \verb![2 2]! bulunmali ve etiket olarak \verb!1!
sonucu dondurulmelidir.

Ustte tarif edilen basit bir ihtiyac, yontem gibi gorulebilir. Fakat yapay
ogrenim ve yapay zeka cok boyutlarda oruntu tanima (pattern recognition)
ile ugrasir, ve milyonlarca satirlik veri, onlarca boyut (ustteki ornekte
2, fakat cogunlukla cok daha fazla boyut vardir) isler hakikaten
zorlasabilir. Mesela goruntu tanimada veri \verb!M x N! boyutundaki
dijital imajlar (duzlestirilince $M \cdot N$ boyutunda), ve onlarin
icindeki resimlerin kime ait oldugu etiket bilgisi olabilir. kNN bu tur
multimedya, cok boyutlu veri ortaminda basarili sekilde
calisabilmektedir. Ayrica en yakin k komsunun icerigi tarifsel bilgi
cikarimi (knowledge extraction) amaciyla da kullanilabilir [2].

``En yakin'' sozu bir kordinat sistemi anlamina geliyor, ve kNN, aynen
k-Means ve diger pek cok kordinatsal ogrenme yontemi gibi eldeki cok
boyutlu veri noktalarinin elemanlarini bir kordinat sistemindeymis gibi
gorur. Kiyasla mesela APriori gibi bir algoritma metin bazli veriyle oldugu
gibi calisabilirdi.

Peki arama baglaminda, bir veri obegi icinden en yakin noktalari bulmanin
en basit yolu nedir? Listeyi bastan sonra taramak (kaba kuvvet yontemi
-brute force-) listedeki her nokta ile yeni nokta arasindaki mesafeyi teker
teker hesaplayip en yakin k taneyi icinden secerdi, bu bir yontemdir.. Bu
basit algoritmanin yuku $O(N)$'dir. Eger tek bir nokta ariyor olsaydik,
kabul edilebilir olabilirdi. Fakat genellikle bir siniflayici (classifier)
algoritmasinin surekli islemesi, mesela bir online site icin gunde
milyonlarca kez bazi kararlari almasi gerekebilir. Bu durumda ve $N$'in cok
buyuk oldugu sartlarda, ustteki hiz bile yeterli olmayacaktir.

Arama islemini daha hizli yapmanin yollari var. Akilli arama algoritmalari
kullanarak egitim verilerini bir agac yapisi uzerinden tarayip erisim
hizini $O(\log N)$'e indirmek mumkundur.

K�re Aga�lar� (Ball Tree, BT) 

Bir noktanin diger noktalara yakin olup olmadiginin hesabinda yapilmasi
gereken en pahali islem nedir? Mesafe hesabidir. BT algoritmasinin puf
noktasi bu hesabi yapmadan, noktalara degil, noktalari kapsayan
"kurelere" bakarak hiz kazandirmasidir. Noktalarin her biri yerine o
noktalari temsil eden kurenin mihenk noktasina (pivot -bu nokta kure
icindeki noktalarin ortalamasal olarak merkezi de olabilir, herhangi bir
baska nokta da-) bakilir, ve oraya olan mesafeye gore bir kure altindaki
noktalara olabilecek en az ve en fazla uzaklik hemen anlasilmis olur.

Not: Kure kavrami uc boyutta anlamli tabii ki, iki boyutta bir cemberden
bahsetmek lazim, daha yuksek boyutlarda ise merkezi ve capi olan bir
``hiper yuzeyden'' bahsetmek lazim. Tarifi kolaylastirdigi icin cember ve
kure tanimlarini kullaniyoruz.

Mesela elimizde alttaki gibi noktalar var ve kureyi olusturduk. 

\includegraphics[height=4cm]{knn0.png}

Bu kureyi kullanarak kure disindaki herhangi bir nokta $q$'nun kuredeki
"diger tum noktalar $x$'e" olabilecegi en az mesafenin ne olacagini
ucgensel esitsizlik ile anlayabiliriz.

Ucgensel esitsizlik 

$$ |x-y| \le |x-z| + |z-y| $$

$||$ operatoru norm operatoru anlamina gelir ve uzaklik hesabinin
genellestirilmis halidir. Konu hakkinda daha fazla detay icin
{\em Fonksinel Analiz} ders notlarimiza bakabilirsiniz. Kisaca soylenmek
istenen iki nokta arasinda direk gitmek yerine yolu uzatirsak, mesafe
artacagidir. Tabii uzaklik, yol, nokta gibi kavramlar tamamen soyut
matematiksel ortamda da isleyecek sekilde ayarlanmistir. Mesela mesafe
(norm) kavramini degistirebiliriz, Oklitsel yerine Manhattan mesafesi
kullaniriz, fakat bu kavram bir norm oldugu ve belirttigimiz uzayda
gecerli oldugu icin ucgensel esitsizlik uzerine kurulmus tum diger
kurallar gecerli olur.

\includegraphics[height=6cm]{tri1.png}

Simdi diyelim ki disaridaki bir $q$ noktasindan bir kure icindeki diger tum
$x$ noktalarina olan mesafe hakkinda bir seyler soylemek istiyoruz. Ustteki
sekilden bir ucgensel esitsizlik cikartabiliriz,

$$ |x-c| + |x-q| \ge |q-c|  $$

Bunun dogru bir ifade oldugunu biliyoruz. Peki simdi yaricapi bu ise dahil
edelim, cunku yaricap hesabi bir kere yapilip kure seviyesinde depolanacak
ve bir daha hesaplanmasi gerekmeyecek, yani algoritmayi hizlandiracak bir
sey olabilir bu, o zaman eger $|x-c|$ yerine yaricapi kullanirsak,
esitsizlik hala gecerli olur, sol taraf zaten buyuktu, simdi daha da buyuk
olacak, 

$$ radius + |x-q| \ge |q-c|  $$

Bunu nasil boyle kesin bilebiliyoruz? Cunku BT algoritmasi radius'u
$|x-c|$'ten kesinlikle daha buyuk olacak sekilde secer). Simdi yaricapi
saga gecirelim,

$$ |x-q| \ge |q-c| - radius $$

Boylece guzel bir tanim elde ettik. Yeni noktanin kuredeki herhangi
bir nokta $x$'e olan uzakligi, yeni noktanin mihenke olan uzakliginin
yaricapi cikartilmis halinden {\em muhakkak} fazladir. Yani bu cikartma
isleminden ele gecen rakam yeni noktanin $x$'e uzakligina bir "alt
sinir (lower bound)" olarak kabul edilebilir. Diger tum mesafeler bu
rakamdan daha buyuk olacaktir. Ne elde ettik? Sadece bir yeni nokta,
mihenk ve yaricap kullanarak kuredeki "diger tum noktalar hakkinda"
bir irdeleme yapmamiz mumkun olacak. Bu noktalara teker teker bakmamiz
gerekmeyecek. Bunun nasil ise yaradigini algoritma detaylarinda
gorecegiz.

Benzer sekilde 

\includegraphics[height=4cm]{tri2.png}

Bu ne diyor? 

$$ |q-c| + |x-c| \ge |q-x| $$

$|x-c|$ yerine yaricap kullanirsak, sol taraf buyuyecegi icin buyukluk hala
buyukluk olarak kalir, 

$$ |q-c| + radius \ge |q-x| $$

Ve yine daha genel ve hizli hesaplanan bir kural elde ettik (onceki
ifadeye benzemesi icin yer duzenlemesi yapalim)

$$ |q-x| \le |q-c| + radius $$

Bu ifade ne diyor? Yeni noktanin mihenke olan uzakligina yaricap
``eklenirse'' bu uzakliktan, buyuklukten daha buyuk bir yeni nokta / kure
 mesafesi olamaz, kuredeki hangi nokta olursa olsun. Bu esitsizlik te bize
 bir ust sinir (upper bound) vermis oldu. 

Algoritma


\begin{algorithm}[t]
\begin{pseudocode}
\codename $\code{ball\_knn}\left(PS^{in},node\right)$\\
\codeline -- Eger alttaki sart gecerli ise node icindeki bir noktanin daha once \\
\codeline -- kesfedilmis $k$ en yakin komsudan daha yakin olmasi imkansizdir\\
\codeline $\code{if } D^{node}_{minp} \ge D_{sofar}$ \\
\codeline \> $\code{return } PS_{in}$ degismemis halde;\\
\codeline $\code{else if } node $ bir cocuk noktasi ise \\
\codeline \> $PS_{out} = PS_{in}$;\\
\codeline \> $\code{for } \forall x \in points(node)$\\
\codeline \> \> $\code{if } \left( |x-q| < D_{sofar} \right)$; -- basit lineer arama yap\\
\codeline \> \> $x$'i $PS_{out}$'a ekle;\\
\codeline \> \> $\code{if } |PS^{out}| == k+1$;\\
\codeline \> \> \> en uzak olan komsuyu $PS^{out}$'tan cikart;\\
\codeline \> \> \> $D_{sofar}$'i guncelle;\\
\codeline \\
\codeline -- eger uc nokta degil ise iki cocuk dugumden daha yakin olanini \\
\codeline -- incele, sonra daha uzakta olanina bak. buyuk bir ihtimalle   \\
\codeline -- arama devam ettirilirse bu arama kendiliginden kesilecektir  \\
\codeline $\code{else }$\\
\codeline \> $node_1 = node$'un $q$'ya en yakin cocugu;\\
\codeline \> $node_2 = node$'un $q$'dan en uzak cocugu;\\
\codeline \> $PS^{temp} = \code{ball\_knn}(PS^{in},node_1)$;\\
\codeline \> $PS^{out} = \code{ball\_knn}(PS^{temp},node_2);$
\end{pseudocode}
\end{algorithm}

\includegraphics[height=7cm]{alg.png}

Kure Agaclari (BT) metotu once kureleri, agaclari olusturmalidir. Bu
kureler hiyerarsik sekilde planlanir, tum noktalarin icinde oldugu bir "en
ust kure" vardir her kurenin iki tane cocuk kuresi olabilir. Belli bir
(disaridan tanimlanan) minimum $r_{min}$ veri noktasina gelinceye kadar
sadece noktalari geometrik olarak kapsamakla gorevli kureler olusturulur,
kureler noktalari sahiplenmezler. Fakat bu $r_{min}$ sayisina erisince
(artik oldukca alttaki) kurelerin uzerine noktalar konacaktir.

Once tek kurenin olusturulusuna bakalim. Bir kure olusumu icin eldeki veri
icinden herhangi bir tanesi mihenk olarak kabul edilebilir. Daha sonra bu
mihenkten diger tum noktalara olan uzaklik olculur, ve en fazla, en buyuk
olan uzaklik yaricap olarak kabul edilir (her seyi kapsayabilmesi icin).

Not: Bu arada "tum diger noktalara bakilmasi" dedik, bundan kacinmaya
calismiyor muyduk?  Fakat dikkat, "kure olusturulmasi" evresindeyiz, k
tane yakin nokta arama evresinde degiliz. Yapmaya calistigimiz aramalari
hizlandirmak - egitim / kure olusturmasi bir kez yapilacak ve bu egitilmis
kureler bir kenarda tutulacak ve surekli aramalar icin ardi ardina
kullanilacaklar.

Kureyi olusturmanin algoritmasi soyledir: verilen noktalar icinde herhangi
birisi mihenk olarak secilir. Sonra bu noktadan en uzakta olan nokta $f_1$,
sonra $f_1$'den en uzakta olan nokta $f_2$ secilir. Sonra tum noktalara
teker teker bakilir ve $f_1$'e yakin olanlar bir gruba, $f_2$'ye yakin
olanlar bir gruba ayrilir. 

\begin{minted}[fontsize=\footnotesize]{python}
import itertools

def dist(vect,x):
    return np.fromiter(itertools.imap
                       (np.linalg.norm, vect-x),dtype=np.float)

def norm(x,y): return np.linalg.norm(x-y)

points = np.array([[3.,3.],[2.,2.]])
q = [1.,1.]
print 'diff', points-q
print 'dist', dist(points,q)
\end{minted}

\begin{verbatim}
diff [[ 2.  2.]
 [ 1.  1.]]
dist [ 2.82842712  1.41421356]
\end{verbatim}

\begin{minted}[fontsize=\footnotesize]{python}
# k-nearest neighbor Ball Tree algorithm in Python
import pprint

__rmin__ = 2

# node: [pivot, radius, points, [child1,child2]]
def new_node(): return  [None,None,None,[None,None]]

def zero_if_neg(x):
    if x < 0: return 0
    else: return x

def form_tree(points,node,all_points,plot_tree=False):    
    pivot = points[0]
    radius = np.max(dist(points,pivot))
    if plot_tree: plot_circles(pivot, radius, points, all_points)
    node[0] = pivot
    node[1] = radius
    if len(points) <= __rmin__:
        node[2] = points
        return
    idx = np.argmax(dist(points,pivot))
    furthest = points[idx,:]
    idx = np.argmax(dist(points,furthest))
    furthest2 = points[idx,:]
    dist1=dist(points,furthest)
    dist2=dist(points,furthest2)
    diffs = dist1-dist2
    p1 = points[diffs <= 0]
    p2 = points[diffs > 0]
    node[3][0] = new_node() # left child
    node[3][1] = new_node() # right child
    form_tree(p1,node[3][0],all_points)
    form_tree(p2,node[3][1],all_points)

# knn: [min_so_far, [points]]
def search_tree(new_point, knn_matches, node, k):
    pivot = node[0]
    radius = node[1]
    node_points = node[2]
    children = node[3]

    # calculate min distance between new point and pivot
    # it is direct distance minus the radius
    min_dist_new_pt_node = norm(pivot,new_point) - radius
    
    # if the new pt is inside the circle, its potential minimum
    # distance to a random point inside is zero (hence
    # zero_if_neg). we can only say so much without looking at all
    # points (and if we did, that would defeat the purpose of this
    # algorithm)
    min_dist_new_pt_node = zero_if_neg(min_dist_new_pt_node)
    
    knn_matches_out = None
    
    # min is greater than so far
    if min_dist_new_pt_node >= knn_matches[0]:
        # nothing to do
        return knn_matches
    elif node_points != None: # if node is a leaf
        print knn_matches_out
        knn_matches_out = knn_matches[:] # copy it
        for p in node_points: # linear scan
            if norm(new_point,p) < radius:
                knn_matches_out[1].append([list(p)])
                if len(knn_matches_out[1]) == k+1:
                    tmp = [norm(new_point,x) \
                               for x in knn_matches_out[1]]
                    del knn_matches_out[1][np.argmax(tmp)]
                    knn_matches_out[0] = np.min(tmp)

    else:
        dist_child_1 = norm(children[0][0],new_point)
        dist_child_2 = norm(children[1][0],new_point)
        node1 = None; node2 = None
        if dist_child_1 < dist_child_2:
            node1 = children[0]
            node2 = children[1]
        else:
            node1 = children[1]
            node2 = children[0]

        knn_tmp = search_tree(new_point, knn_matches, node1, k)
        knn_matches_out = search_tree(new_point, knn_tmp, node2, k)
            
    return knn_matches_out
                   
points = np.array([[3.,4.],[5.,5.],[9.,2.],[3.2,5.],[7.,5.],
                 [8.,9.],[7.,6.],[8,4],[6,2]])
tree = new_node()
form_tree(points,tree,all_points=points)
pp = pprint.PrettyPrinter(indent=4)
print "tree"
pp.pprint(tree)
newp = np.array([7.,7.])
dummyp = [np.Inf,np.Inf] # it should be removed immediately
res = search_tree(newp,[np.Inf, [dummyp]], tree, k=2)
print "done", res
\end{minted}

\begin{verbatim}
tree
[   array([ 3.,  4.]),
    7.0710678118654755,
    None,
    [   [   array([ 8.,  9.]),
            3.1622776601683795,
            array([[ 8.,  9.],
       [ 7.,  6.]]),
            [None, None]],
        [   array([ 3.,  4.]),
            6.324555320336759,
            None,
            [   [   array([ 9.,  2.]),
                    3.6055512754639891,
                    None,
                    [   [   array([ 7.,  5.]),
                            1.4142135623730951,
                            array([[ 7.,  5.],
       [ 8.,  4.]]),
                            [None, None]],
                        [   array([ 9.,  2.]),
                            3.0,
                            array([[ 9.,  2.],
       [ 6.,  2.]]),
                            [None, None]]]],
                [   array([ 3.,  4.]),
                    2.2360679774997898,
                    None,
                    [   [   array([ 5.,  5.]),
                            0.0,
                            array([[ 5.,  5.]]),
                            [None, None]],
                        [   array([ 3.,  4.]),
                            1.019803902718557,
                            array([[ 3. ,  4. ],
       [ 3.2,  5. ]]),
                            [None, None]]]]]]]]
None
done [1.0, [[[8.0, 9.0]], [[7.0, 6.0]]]]
\end{verbatim}


Bu iki grup, o anda islemekte oldugumuz agac dugumun (node) iki
cocuklari olacaktir. Cocuk noktalari kararlastirildiktan sonra artik
sonraki asamaya gecilir, fonksiyon \verb!form_tree! bu cocuk
noktalari alarak, ayri ayri, her cocuk grubu icin ozyineli (recursive)
olarak kendi kendini cagirir. Kendi kendini cagiran
\verb!form_tree!, tekrar basladiginda kendini yeni (bir) nokta
grubu ve yeni bir dugum objesi ile basbasa bulur, ve hicbir seyden
habersiz olarak isleme koyulur. Tabii her ozyineli cagri yeni dugum
objesini yaratirken bir referansi ustteki ebeveyn dugume koymayi
unutmamistir, boylece ozyineli fonksiyon dunyadan habersiz olsa bile,
agacin en ustunden en altina kesintisiz bir baglanti zinciri hep
elimizde olur.

Not: \verb!form_tree! icinde bir numara yaptik, tum noktalarin
$f_1$'e olan uzakligi \verb!dist1!, $f_2$'e olan uzakligi ise
\verb!dist2!. Sonra \verb!diffs = dist1-dist2! ile bu iki
uzakligi birbirinden cikartiyoruz ve mesela \verb!points[diffs <= 0]!
ile $f_1$'e yakin olanlari buluyoruz, cunku bir tarafta
$f_1$'e yakinlik 4 diger tarafta $f_2$'ye yakinlik 6 ise, 4-6=-2 ie o
nokta $f_1$'e yakin demektir. Ufak bir numara ile Numpy dilimleme
(slicing) teknigini kullanabilmis olduk ve bu onemli cunku boylece
\verb!for! dongusu yazmiyoruz, Numpy'in arka planda C ile
yazilmis hizli rutinlerini kullaniyoruz.

Ek bazi bilgiler: kurelerin sinirlari kesisebilir. 

Arama 

Ustte sozde program (pseudocode) $BallKNN$ olarak gosterilen ve bizim
kodda \verb!search_tree! olarak anilan fonksiyon arama
fonksiyonu. Aranan \verb!new_point!'e olan k en yakin diger veri
noktalar. Disaridan verilen degisken \verb!knn_matches! uzerinde
fonksiyon ozyineli bir sekilde arama yaparken "o ana kadar bulunmus en
yakin k nokta" ve o noktalarin \verb!new_point!'e olan en yakin
mesafesi saklanir, arama isleyisi sirasinda \verb!knn_matches!,
\verb!knn_matches_out! surekli verilip geri dondurulen
degiskenlerdir, sozde programdaki $P^{in},P^{out}$'un karsiligidirlar.

Arama algoritmasi soyle isler: simdi onceden olusturulmus kure
hiyerarisisini ustten alta dogru gezmeye baslariz. Her basamakta yeni
nokta ile o kurenin mihenkini, yaricapini kullanarak bir "alt sinir
mesafe hesabi" yapariz, bu mesafe hesabinin arkasinda yatan dusunceyi
yazinin basinda anlatmistik. Bu mesafe kure icindeki tum noktalara
olan bir en az mesafe idi, ve eger eldeki \verb!knn_matches!
uzerindeki simdiye kadar bulunmus mesafelerin en azindan daha az ise,
o zaman bu kure "bakmaya deger" bir kuredir, ve arama algoritmasi bu
kureden isleme devam eder. Simdiye kadar bulunmus mesafelerin en azi
\verb!knn_matches! veri yapisi icine \verb!min_so_far!
olarak saklaniyor, sozde programdaki $D_{sofar}$.

Bu irdeleme sonrasi (yani vs kuresinden yola devam karari arkasindan)
isleme iki sekilde devam edilebilir, cunku bir kure iki turden
olabilir; ya nihai en alt kurelerden biridir ve uzerinde gercek
noktalar depolanmistir, ya da ara kurelerden biridir (sona gelmedik
ama dogru yoldayiz, daha alta inmeye devam), o zaman fonksiyon yine
ozyineli bir sekilde bu kurenin cocuklarina bakacaktir - her cocuk
icin kendi kendini cagiracaktir. Ikinci durumda, kurede noktalar
depolanmistir, artik basit lineer bir sekilde o tum noktalara teker
teker bakilir, eldekilerden daha yakin olani alinir, eldeki liste
sismeye baslamissa (k'den daha fazla ise) en buyuk noktalardan biri
atilir [3], vs.

Daha alta inmemiz gereken birinci durumda yapilan iki cagrinin bir
ozelligine dikkat cekmek isterim. Yeni noktanin bu cocuklara olan
uzakligi da olculuyor, ve en once, en yakin olan cocuga dogru bir
ozyineleme yapiliyor.  Bu nokta cok onemli: niye boyle yapildi? Cunku
icinde muhtemelen daha yakin noktalarin olabilecegi kurelere dogru
gidersek, ozyineli cagrilarin teker teker bitip yukari dogru cikmaya
baslamasi ve kaldiklari yerden bu sefer ikinci cocuk cagrilarini
yapmaya baslamasi ardindan, elimizdeki \verb!knn_matches!
uzerinde en yakin noktalari buyuk bir ihtimalle zaten bulmus
olacagiz. Bu durumda ikinci cagri yapilsa bile tek bir alt sinir
hesabi o kurede dikkate deger hicbir nokta olamayacagini ortaya
cikaracak (cunku en iyiler zaten elimizde), ve ikinci cocuga olan
cagrilar hic alta inmeden pat diye geri donecektir, hic asagi
inilmeyecektir.

Bu muthis bir kazanimdir: zaten bu stratejiye liteturde "budamak
(pruning)" adi veriliyor, bu da cok uygun bir kelime aslinda, cunku
agaclarla ugrasiyoruz ve bir dugum (kure) ve onun altindaki hicbir alt
kureye ugramaktan kurtularak o dallarin tamamini bir nevi "budamis"
oluyoruz. Bir suru gereksiz islemden de kurtuluyoruz bu arada, ve aramayi
hizlandiriyoruz.

Mesafeler

Algoritmanin mesafeleri anlatan kisminda norm ve uzaylar gibi kavramlardan
bahsettik. Yeni noktanin mihenke olan uzakliginin o kure icindeki tum diger
noktalara olan uzakligini temsil edebilecegini soyledik: peki niye bu
kavramlari direk bu sekilde anlatmadik, ve norm, ucgensel esitsizlik gibi
kavramlardan bahsettik? Cunku 2 ve 3 boyut sonrasi uzaylari gorsel olarak
dusunmek mumkun degildir, istedigimiz kadar ellerimizi kollarimizi
sallayalim, bu kavramlari gorsel olarak tarif edemeyiz, ve degisik bir norm
(mesafe) olcutu kullanmayi secebiliriz. Bu her iki durumda da elimizde
soyut matematik baglaminda saglam bir temel oldugunu bilmek algoritmanin
genelligini, ve degisik sartlarda uygulanabilirligini arttirir. Mesela
Oklit mesafesi yerine Manhattan mesafesi kullansam bile, bu mesafenin
olcutunun norm kurallarini uydugunu bildigim icin kNN yapisinin geri
kalanini oldugu gibi kullanabilirim, cunku o yapinin gecerliligini normlar
uzerinde gecerli ucgensel esitsizlik uzerinde ispat ettim. 

Model

kNN'in model kullanmayan, model yerine verinin kendisini kullanan bir
algoritma olarak tanittik. Peki ``egitim'' evresi sonrasi ele gecen kureler
ve agac yapisi bir nevi model olarak gorulebilir mi? 

Bu onemli bir soru, ve bir bakima, evet agac yapisi sanki bir modelmis gibi
duruyor. Fakat, mesela istatistiksel, grafiksel, yapay sinir aglari (neural
net) baglaminda bakilirsa bu yapiya tam bir model denemez. Model bazli
metotlarda model kurulunca veri atilir, ona bir daha bakilmaz. Fakat kNN,
kure ve agac yapisini hala eldeki veriye erismek icin kullanmaktadir. Yani
bir bakima veriyi ``indeksliyoruz'', ona erisimi kolaylastirip
hizlandiriyoruz, ama ondan model cikartmiyoruz. 

Not: Verilen Python kodu ve algoritma yakin noktalari hesapliyor sadece,
onlarin etiketlerinden hareketle yeni noktanin etiketini tahmin etme
asamasini gerceklestirmiyor. Fakat bu son asama isin en basit tarafi,
egitim veri yapisina eklenecek bir etiket bilgisi ve siniflama sonrasi k
noktanin agirlikli etiketinin hesabi ile basit sekilde
gerceklestirilebilir.

\begin{minted}[fontsize=\footnotesize]{python}
!python plot_circles.py
\end{minted}


Agaci olusumu sirasinda kurelerin grafigi alttadir. 

\includegraphics[height=6cm]{knn0.png}
\includegraphics[height=6cm]{knn1.png}
\includegraphics[height=6cm]{knn2.png}
\includegraphics[height=6cm]{knn3.png}
\includegraphics[height=6cm]{knn4.png}
\includegraphics[height=6cm]{knn5.png}
\includegraphics[height=6cm]{knn6.png}
\includegraphics[height=6cm]{knn7.png}
\includegraphics[height=6cm]{knn8.png}
\includegraphics[height=6cm]{knn9.png}

Kaynaklar, Notlar

[1] Liu, Moore, Gray, New Algorithms for Efficient High Dimensional
Non-parametric Classification

[2] Alpayd�n, Introduction to Machine Learning

[3] Silme islemi ornek kodumuzda Python \verb!del! ile
gerceklestirildi. Eger bu islem de hizlandirilmak istenirse, en alt kure
seviyesindeki veriler bir oncelik kuyrugu (priority queue) uzerinde
tutulabilir, ve silme islemi hep en sondaki elemani siler, ekleme islemi
ise yeni elemani (hep sirali olan) listede dogru yere koyar.


\end{document}
