\documentclass[12pt,fleqn]{article}
\setlength{\parindent}{0pt}
\usepackage{graphicx}
\usepackage{cancel}
\usepackage{listings}
\usepackage[latin5]{inputenc}
\setlength{\parskip}{8pt}
\setlength{\parsep}{0pt}
\setlength{\headsep}{0pt}
\setlength{\topskip}{0pt}
\setlength{\topmargin}{0pt}
\setlength{\topsep}{0pt}
\setlength{\partopsep}{0pt}
\setlength{\mathindent}{0cm}

\begin{document}
Filtrelemek

Filtreler dis dunyadaki bir aksiyon hakkinda elde edilen gurultulu
sinyalleri, tersine cevirerek arka plandaki aksiyon hakkinda hesaplama
yapabilmemizi saglar. Mesela Kalman Filtreleri (KF) icin gizlenmis konum
bir robotun nerede oldugu, bir senetin fiyati gibi bir sey olabilir, gizli
konum bilgisi $x_t$ degiskeninde o konum hakkindaki gurultulu olcum $y_t$
icindedir. Hem gizli konumlar arasindaki gecis, hem de olcumun gurultusu
lineer bir fonksiyon uzerindendir.

\[ x_{t+1} = Ax_t + v \]

\[ y_t = Hx_t + w \]

$v$ ve $w$'in dagilimi Gaussian'dir ve kovaryans sirasiyla $Q$ ve $R$
icindedir. 









\end{document}
