\documentclass[12pt,fleqn]{article}\usepackage{../common}
\begin{document}
Matrislerin, Vekt�rlerin T�revleri

Gradyan

$m$ boyutlu vektor $x$'i alan ve geriye tek sayi sonucu donduren bir $f$
fonksiyonunun $x$'e gore turevini nasil aliriz? Yani $x \in \mathbb{R}^m$ ve

$$ x = 
\left[\begin{array}{ccc}
x_1 \\ \vdots \\ x_m
\end{array}\right]
 $$

Bu durumda $x$'in her ogesine gore kismi turevler (partial derivatives)
alinir, ve tek boyutlu / tekil sayi boyutundaki sonuclar $m$ boyutlu bir
sonuc vektorune teker teker yerlestirilir,

$$
\frac{\partial f}{\partial x}  =
\left[\begin{array}{c}
\frac{\partial f}{\partial x_1} \\
\\
\frac{\partial f}{\partial x_2} \\
\vdots \\
\frac{\partial f}{\partial x_m} 
\end{array}\right]
$$

Bu sonuc tanidik gelmis olabilir, cunku ustteki ifade gradyan olarak ta
biliniyor,

$$ \frac{\partial f}{\partial x}  = \nabla f = grad \ f(x) $$

Elde edilen vektor surpriz degil cunku tek, skalar bir deger veren bir
fonksiyonun $x$ icindeki {\em her ogensinin} nasil degistigine gore bunun
fonksiyon uzerindeki etkilerini merak ediyorduk, ustteki vektor oge bazinda
bize aynen bunu gosteriyor. Yani tek skalar sonuc $m$ tane sonuca
ayriliyor, cunku tek sonucun $m$ tane secenege gore degimini gormek
istedik. Not olarak belirtelim, gradyan vektoru matematiksel bir rahatlik
olarak matematikcilerin kullandigi bir kisayol, bir ziplama noktasi, yani
matematiksel olarak turetilerek ulasilan ``kurallardan'' biri degil.

Tek Parametreye Gore Matris Turevi

Eger bir $A$ matrisinin tum ogeleri bir $\theta$ parametresine bagli ise, o
matrisin $\theta$'ya gore turevi icin tum elemanlarinin teker teker
$\theta$'ya gore turevleri alinir,

$$ 
\frac{\partial A}{\partial \theta} = 
\left[\begin{array}{cccc}
\frac{\partial a_{11}}{\partial \theta} & 
\frac{\partial a_{12}}{\partial \theta} & \dots & 
\frac{\partial a_{1n}}{\partial \theta} \\

\frac{\partial a_{21}}{\partial \theta} & 
\frac{\partial a_{22}}{\partial \theta} &  \dots & 
\frac{\partial a_{2n}}{\partial \theta}  \\

\vdots & \vdots & \ddots & \vdots \\

\frac{\partial a_{m1}}{\partial \theta} & 
\frac{\partial a_{m2}}{\partial \theta} &  \dots & 
\frac{\partial a_{mn}}{\partial \theta}  

\end{array}\right]
$$

Vektor Turevleri

Eger bir $x \in \mathbb{R}^m$ vektorunden bagimsiz bir $A$ matrisi o $x$ ile carpiliyor ise,
bunlarin $x$'e gore turevi nedir? 

$$ \frac{\partial}{\partial x^T} [Ax] = A
$$

Ustteki sonuc aslinda tek sayili / boyutlu ortamda $2x$ gibi bir ifadenin
$x$'e gore turevini alinca $2$ elde etmeye esdeger. Ispat icin soyle
dusunelim, eger $a_i \in \mathbb{R}^n$ ise (ki devrigi alininca bu vektor
yatay hale gelir, yani altta bu yatay vektorleri ust uste istifliyoruz), 

$$ A = \left[\begin{array}{c}
a_1^T \\ \vdots \\ a_m^T
\end{array}\right] $$

Bu durumda $Ax$ ne olur? {\em Matris Carpimi} yazisindaki satir bakis acisi
dusunulurse, $A$'in bir satirinin her ogesi $x$'in tum satirlarini (burada
$x$ vektor oldugu icin her satir tek bir sayidan ibaret) kombine ederek o
sonuc satirini olusturmaktadir, o zaman

$$ A = \left[\begin{array}{c}
a_1^Tx_1 \\ \vdots \\ a_m^Tx_m
\end{array}\right] $$





\end{document}
