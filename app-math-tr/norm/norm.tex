\documentclass[12pt,fleqn]{article}\usepackage{../common}
\begin{document}
Uzakliklar, Norm, Benzerlik

Literaturdeki anlatim bu kavramlar etrafinda biraz kafa karisikligi
yaratabilir, bu yazida biraz aciklik getirmeye calisalim. Norm bir buyukluk
olcusudur, vektor uzaylari ile olan alaksini gormek icin {\em Fonksiyonel
  Analiz} notlarina bakilabilir. Buyukluk derken yani basitce bir $x$
vektorunun buyuklugunden bahsediyoruz. Cogunlukla $||x||$ gibi bir kullanim
gorulur kitaplarda, eger altsimge yok ise, o zaman bu 2 kabul edilir, yani
$||x||_2$ (bu cogunlukla atlaniyor), ve bu ifade bir L2 norm'unu ifade
eder. $||x||_1$ varsa L1 norm'u olurdu.

L1,L2 normalari, ya da genel olarak $L_p$ normlari soyle gosterilir

$$ ||x||_p = (\sum_i |x_i|^p)^{1/p} $$





\end{document}
