\documentclass[12pt,fleqn]{article}\usepackage{../common}
\begin{document}
S�n�rl� Elementler Metodu (Finite Elements Method)

Bu metot differansiyel, kismi differansiyel denklemleri (partial
differential equations) yaklasiksal olarak modelleme ve cozmenin
yontemleridir.

Formul: Baslangic denklemi

$$ \frac{-d}{dx} \bigg( c(x) \ \frac{du}{dx} \bigg) = f(x) $$

Iki tarafi da  $v(x)$ ile carpiyoruz ve 0 to 1 sinirlariyla entegralini aliyoruz.

$$ \int_0^1 \frac{-d}{dx} \bigg( c(x) \ \frac{du}{dx} \bigg) v(x)dx = \int_0^1 f(x)v(x)dx $$

Parcali entegral (integration by parts) formulu soyledir:

$$ \int y \ dz = y  z \int z \ dy $$

Ana formulun bolumlerini, parcali entegrale gore bolusturursek:

$$ dz = \frac{-d}{dx} \bigg( c(x) \ \frac{du}{dx} \bigg) dx  $$

$$ z = - c(x) \ \frac{du}{dx}  $$

$$ y = v(x)  $$

$$ dy = \frac{dv}{dx}dx $$

Yukarida $dz$ icinde $dx$ ve $\frac{1}{dx}$ birbirini iptal eder. Parcali
entegral formulunun sag tarafina gore yerlerine koyarsak:

$$ \int_0^1 v(x)dx \frac{-d}{dx} \bigg( c(x) \ \frac{du}{dx} \bigg) = - [v(x) c(x) \frac{du}{dx} \bigg]_{x=0}^{x=1} \int_0^1 c(x) \ \frac{du}{dx} \frac{dv}{dx}dx $$

Ustteki parcali entegral aciliminda sol taraf entegrale sinir
degerleri aldiginda, sag taraftaki $yz$ sonucunun ayni sinir
degerlerine tabi olduguna dikkat edelim.

Differansiyel denklemde sinir kosullari $x=1$ durumunda $c(1)u'(1)=0$,
ve $x=0$ durumunda $v(0)=0$ olarak biliniyor. O zaman ustteki
denklemin sol tarafinda $x=0$ ve $x=1$ kosullari icin tanimli bolum $0
- 0 = 0$ olacaktir ve denklemden atilabilir. Geriye kalanlar

$$ \int_0^1 c(x) \frac{du}{dx} \frac{dv}{dx} dx  = \int_0^1 f(x)v(x)dx $$

Bu fonksiyonu Galerkin adli bir matematikci bulmus, "zayif form (weak
form)" olarak adlandiriliyor.

Simdi diyelim ki n tane test fonksiyonu sectik $\phi_1(x),..,\phi(n)$
ve bu fonksiyonlarin $U_j$ sayilari ile carpiminin toplamini, yani bir
tur kombinasyonunu $u(x)$ yerine kullanmaya karar verdik.

$$ U(x) = U_1 \phi_1+ ... + U_n\phi_n $$

O zaman

$$ U'(x) = U_1 \phi_1'+ ... + U_n\phi_n' $$

$$ = \sum_1^n U_j \frac{d\phi_j}{dx} $$

Simdi $du / dx$ yerine $U'(x)$ koyarsak

$$ \int_0^1 c(x) \bigg( \sum_1^n U_j \frac{d\phi_j}{dx}\bigg)  \frac{dV_i}{dx} dx  = \int_0^1 f(x)V_i(x)dx $$

Dikkat edelim, $v(x)$ yerine $V_i(x)$ kullandik. Ustteki formul her i icin yeni
bir formul "uretecek". Niye $V_i$? Zayif formdaki $v(x)$ formulunu de zaten
biz uydurmustuk, yani $v(x)$ biz ne istersek o olur. O zaman bu fonksiyonu n
tane formul uretmek icin bir numara olarak kullaniliyoruz, n tane formul olunca
matrisin n x n elemanini doldurabilecegiz ve cozume erisebilecegiz. Ek not,
cogunlukla $V_i(x)$ icin $\phi_i$ formulleri kullaniliyor. 

Ayrica formuldeki $U_j$ kismini cekip cikartirsak ve bir vektor icine koyarsak,
geri kalanlar bir $K_{ij}$ matrisi icinde tutulabilir. 

$$ K_{ij} = \int_0^1 c(x) \frac{d\phi_j}{dx} \frac{dV_i}{dx} dx  $$

Sag taraf ayni sekilde i tane formul uretir

$$ F_i = \int_0^1 f(x)V_i(x)dx $$

Final formul matrix formunda basit bir sekilde temsil edilebilecektir. 

$$ KU = F $$

Ornek

Ornek olarak $-u'' = 1$ denklemini cozelim. Not: Differansiyel
denklemlerde sonuc bulmak demek bir "fonksiyon" bulmak
demektir. Normal cebirsel denklemlerde sonuc bulmak degiskenlerin
"sayisal" degerini bulmak demektir. Birazdan bulacagimiz sonuc
$u(x)$ "fonksiyonu" olacak.

Eger denklem $-u''=1$ ise o zaman bu formulu ana forma uygun hale
getirmek icin $c(x) = 1$ olarak almamiz gerekir. $-u''=1$ denkleminde
esitligin sag tarafi 1 olduguna gore $f(x) = 1$ demektir.

Artik $\phi$ fonksiyonlarini secme zamani geldi. Bu fonksiyonlarin
"toplami" hedefledigimiz fonksiyonu yaklasiksal (approximate) olarak
temsil edecek. Ornek olarak secebilecegimiz bir fonksiyon "sapka
fonksiyonu (hat function)" olarak bilinen ucgen fonksiyonlar
olabilir. Alttaki figurde bu fonksiyonlari goruyoruz.

\includegraphics[height=4cm]{fem_hat.png}

Bu figurde x ekseninin h buyuklugundeki parcalara bolundugunu goruyoruz. 

Entegralleri hesaplayalim

$$ F_1 = \int_0^1 V_1(x)dx $$

Daha once $V_1$ ve $\phi_1$'i ayni kabul ettigimizi belirtmistik. 

Yukaridaki entegralin aslinda bir alan hesabi yaptigini
goruyoruz. Sinirlar $0$ ve $1$ arasinda, ama $2h$ otesinde zaten
$\phi_1$ fonksiyonu yok. $\phi_1$'in alani nedir? Alan ucgenin alani:
Taban carpi yukseklik bolu 2: $2h$, yuksekligi $1$, o zaman alan $(2h
\times 1) / 2 = 1/3$

Benzer mantikla bakarsak, $F_2$ ile $F_1$ ayni, yani $1/3$. $F_3$ ise
onlarin yarisi, yani $1/6$.

$K_{ij}$ nasil hesaplanacak? $c(x) = 1$ oldugu icin formulden
cikarilabilir ve $V_1$ ve $\phi_1$'in ayni olduguna soyledik:

$$ K_{ij} = \int_0^1 c(x) \frac{d\phi_j}{dx} \frac{dV_i}{dx} dx $$

$$ K_{11} = \int_0^1 \bigg( \frac{dV_1}{dx} \bigg) ^2 dx  $$

$dV_1/dx$ nedir? Birinci sapka fonksiyonunun turevidir. Bu tureve
bakarsak, $0$ ve $h$ arasinda arti egim (slope) $1/h$, $h$ ve $2h$
arasinda eksi egim $-1/h$ oluyor. Ama kare aldigimiz icin sonuc ayni,
$1/h^2$. O zaman h = 1/3 olduguna gore $1/(1/3)^2$, yani $dV_1/dx =
9$.

$$ K_{11} = \int_0^{2/3} 9 dx = 9x \ \bigg|_0^{2/3} = (9)(2/3) - 0 = 6 $$

$K_{22}$ seklen ayni fonksiyon parcasini temel aldigi icin ayni degere
sahip: 6. $K_{33}$ onlarin yarisi, esittir 3.

$K_{12}$ farkli egimlerin carpimi anlamina gelir, yani $V_1'$ ile
$V_2'$ carpimi olur. Bu iki fonksiyona bakalim, 0 ile h arasinda $V_2$
yok, egim 0. Ikisinin de sifir olmadigi, carpimda kullanilabilecek bir
egiminin oldugu tek aralik h ve 2h arasi. Burada $V_1' = -3, V_2 = 3$.

$$ K_{12} = \int_{1/3}^{2/3} (3)(-3) dx = -9x \bigg|_{1/3}^{2/3} = -6 - (-3) = -3 $$

Ayni sekilde $K_{23} = -3$. Ama $K_{13} = 0$ cunku hic cakisma yok.

Matrisi doldurursak, 

$$
KU = F
$$

$$ 
\left[\begin{array}{ccc}
    6 & -3 & 0 \\
    -3 & 6 & -3 \\
    0 & -3 & 3     
\end{array}\right]
\left[\begin{array}{c}
    U_1 \\
    U_2 \\
    U_3
\end{array}\right]
=
\left[\begin{array}{c}
    1/3 \\
    1/3 \\
    1/6
\end{array}\right]
$$

Python kodu 

\begin{minted}{python}
import numpy as np
K = [[6., -3., 0],
     [-3., 6., -3.],
     [0., -3., 3.]]

f = [1./3., 1./3., 1./6.]

print np.linalg.solve(K,f)
\end{minted}

\begin{verbatim}
[ 0.27777778  0.44444444  0.5       ]
\end{verbatim}

\begin{minted}{python}
print 5./18., 4./9., 1./2.
\end{minted}

\begin{verbatim}
0.277777777778 0.444444444444 0.5
\end{verbatim}

Rapor edilen degerler bu denklemin bilinen cozumu $u(x) = x -
\frac{1}{2}x^2$ ile 0, h, 2h noktalarinda (mesh points) birebir uyum
gosterdigini goruyoruz. Yani yaklasiksal olarak differansiyel denklemi
cozmeyi basardik.

Kaynaklar

Strang, G., Computational Science and Engineering, 2007




\end{document}
