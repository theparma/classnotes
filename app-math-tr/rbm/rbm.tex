\documentclass[12pt,fleqn]{article}\usepackage{../common}
\begin{document}
Kisitli Boltzmann Makinalari (Restricted Boltzmann Machines -RBM-)

Bir RBM icinde ikisel (binary) degerler tasiyan gizli (hidden) $h$
degiskenler, ve yine ikisel gorunen (visible) degiskenler $v$ vardir. $Z$
aynen once gordugumuz Boltzman Makinalarinda (BM) oldugu gibi normalizasyon
sabitidir.

$$ p(x,h;W) = \exp (-E(x,h)) / Z $$

$E$ tanimina ``enerji'' olarak atif yapildigini da gorebilirsiniz bazen. 

$$ E(x,h) = -h^TWx - c^Tx - b^Th $$

$$ = - \sum_j \sum_k W_{j,k}h_jx_k - \sum_k c_kx_k - \sum_j b_jh_j  $$

Dikkat: $h,x$ degiskenleri rasgele degiskenlerdir. Yani hem $x$'e hem de
$h$'e ``zar attirabiliriz'' / bu degiskenler uzerinden orneklem
toplayabiliriz. Bu kritik bir konu. Ayrica, ustteki tanimlarda net sekilde
goruluyor ama bir daha vurgulayalim; RBM'ler aynen BM'ler gibi, bir
olasilik dagilimidirlar. Yani tum mumkun degerleri uzerinden entegralleri
(ya da toplamlari) 1 olur, vs.

RBM'lerin alttaki gibi resmedildigini gorebilirsiniz.

\includegraphics[height=4cm]{rbm_01.png}
\includegraphics[height=4cm]{rbm_02.png}

RBM'lerin ``kisitli'' olarak tanimlanmalarinin sebebi, gizli degiskenlerin
kendi aralarinda, ve gorunen degiskenlerin kendi aralarinda direk
baglantiya izin verilmis olmasidir, bu bakimdan
``kisitlanmislardir''. Baglantilara, $W$ uzerinden sadece gizli ve gorunen
degiskenler (tabakalar) arasinda izin verilmistir. Bu tabii ki matematiksel
olarak bazi kolayliklar sagliyor.

Devam edelim, ana formulden hareketle cebirsel olarak sunlar da dogrudur,

$$ p(x,h;W) = \exp (-E(x,h)) / Z $$

$$ = \exp (h^TWx + c^Tx + b^Th ) / Z $$

$$ = \exp (h^TWx) \exp (c^Tx) \exp(b^Th) / Z $$

cunku bir toplam uzerindeki $\exp$, ayri ayri $\exp$'lerin carpimi
olur. Ayni mantikla, eger ana formulu matris / vektor yerine ayri
degiskenler olarak gormek istersek,

$$ 
p(x,h;W) = \frac{1}{Z}
\prod_j \prod_k \exp (W_{jk}h_jx_k) \prod_k \exp(c_kx_k) \prod_j \exp(b_jh_j) 
 $$

























\end{document}
