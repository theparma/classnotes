\documentclass[12pt,fleqn]{article}\usepackage{../common}
\begin{document}

Imaj Bolgelerinin Ikili Karsilastirmasi 

Bu bolumde bir beyani $D$ ortaya koyacagiz, ki bu beyan, imajdaki iki
bilesen (ki imaj gruplamasinin dogru olarak bulmaya calisacagi bilesenler)
arasinda bir sinir olup olmadigina dair kanitin olcusu olacak. Beyanin
temeli sudur: iki bilesen arasindaki sinirin boyunda yer alan her iki
tarafin ogelerinin farkliligina bak, ve onu her bilesenin kendi icindeki
farkliliga gore oranla. Yani bu beyan, bir bilesenin ic farkliligini dis
farkliligina kiyaslar, ve bu sebeple verinin yerel karakteristikleri
gozetmis olur. Kiyaslama mesela, global, verinin her yerinde aynen gecerli
olacak bir sabit esik degerine vs. bagli degildir.

Tanimlar; Bir bilesenin {\em ic farkliligi} $C \in V$, ki $C$ bir
bilesendir (component), ve $V$ cizitin tum noktalaridir, o $C$'nin minimum
kapsayan agacinin, yani $MST(C)$'sinin en buyuk kenar agirligi olarak
aliyoruz. Ic farkliligi $Int(C)$ olarak belirtelim, 

$$ Int(C) = \max_{e \in MST(C,E)} w(e) $$

ki $w((v_i , v_j))$ bir cizit $G = (V,E)$'yi olusturan bir kenar
$(v_i,v_j) \in E$ agirligi olarak belirtilir. 













\end{document}
