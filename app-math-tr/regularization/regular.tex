%% This file was auto-generated by IPython.
%% Conversion from the original notebook file:
%% regular.ipynb
%%
\documentclass[11pt,english,fleqn]{article}

%% This is the automatic preamble used by IPython.  Note that it does *not*
%% include a documentclass declaration, that is added at runtime to the overall
%% document.

\usepackage{amsmath}
\usepackage{amssymb}
\usepackage{graphicx}
\usepackage{ucs}
\usepackage[utf8x]{inputenc}

% needed for markdown enumerations to work
\usepackage{enumerate}

% Slightly bigger margins than the latex defaults
\usepackage{geometry}
\geometry{verbose,tmargin=3cm,bmargin=3cm,lmargin=2.5cm,rmargin=2.5cm}

% Define a few colors for use in code, links and cell shading
\usepackage{color}
\definecolor{orange}{cmyk}{0,0.4,0.8,0.2}
\definecolor{darkorange}{rgb}{.71,0.21,0.01}
\definecolor{darkgreen}{rgb}{.12,.54,.11}
\definecolor{myteal}{rgb}{.26, .44, .56}
\definecolor{gray}{gray}{0.45}
\definecolor{lightgray}{gray}{.95}
\definecolor{mediumgray}{gray}{.8}
\definecolor{inputbackground}{rgb}{.95, .95, .85}
\definecolor{outputbackground}{rgb}{.95, .95, .95}
\definecolor{traceback}{rgb}{1, .95, .95}

% Framed environments for code cells (inputs, outputs, errors, ...).  The
% various uses of \unskip (or not) at the end were fine-tuned by hand, so don't
% randomly change them unless you're sure of the effect it will have.
\usepackage{framed}

% remove extraneous vertical space in boxes
\setlength\fboxsep{0pt}

% codecell is the whole input+output set of blocks that a Code cell can
% generate.

% TODO: unfortunately, it seems that using a framed codecell environment breaks
% the ability of the frames inside of it to be broken across pages.  This
% causes at least the problem of having lots of empty space at the bottom of
% pages as new frames are moved to the next page, and if a single frame is too
% long to fit on a page, will completely stop latex from compiling the
% document.  So unless we figure out a solution to this, we'll have to instead
% leave the codecell env. as empty.  I'm keeping the original codecell
% definition here (a thin vertical bar) for reference, in case we find a
% solution to the page break issue.

%% \newenvironment{codecell}{%
%%     \def\FrameCommand{\color{mediumgray} \vrule width 1pt \hspace{5pt}}%
%%    \MakeFramed{\vspace{-0.5em}}}
%%  {\unskip\endMakeFramed}

% For now, make this a no-op...
\newenvironment{codecell}{}

 \newenvironment{codeinput}{%
   \def\FrameCommand{\colorbox{inputbackground}}%
   \MakeFramed{\advance\hsize-\width \FrameRestore}}
 {\unskip\endMakeFramed}

\newenvironment{codeoutput}{%
   \def\FrameCommand{\colorbox{outputbackground}}%
   \vspace{-1.4em}
   \MakeFramed{\advance\hsize-\width \FrameRestore}}
 {\unskip\medskip\endMakeFramed}

\newenvironment{traceback}{%
   \def\FrameCommand{\colorbox{traceback}}%
   \MakeFramed{\advance\hsize-\width \FrameRestore}}
 {\endMakeFramed}

% Use and configure listings package for nicely formatted code
\usepackage{listingsutf8}
\lstset{
  language=python,
  inputencoding=utf8x,
  extendedchars=\true,
  aboveskip=\smallskipamount,
  belowskip=\smallskipamount,
  xleftmargin=2mm,
  breaklines=true,
  basicstyle=\small \ttfamily,
  showstringspaces=false,
  keywordstyle=\color{blue}\bfseries,
  commentstyle=\color{myteal},
  stringstyle=\color{darkgreen},
  identifierstyle=\color{darkorange},
  columns=fullflexible,  % tighter character kerning, like verb
}

% The hyperref package gives us a pdf with properly built
% internal navigation ('pdf bookmarks' for the table of contents,
% internal cross-reference links, web links for URLs, etc.)
\usepackage{hyperref}
\hypersetup{
  breaklinks=true,  % so long urls are correctly broken across lines
  colorlinks=true,
  urlcolor=blue,
  linkcolor=darkorange,
  citecolor=darkgreen,
  }

% hardcode size of all verbatim environments to be a bit smaller
\makeatletter 
\g@addto@macro\@verbatim\small\topsep=0.5em\partopsep=0pt
\makeatother 

% Prevent overflowing lines due to urls and other hard-to-break entities.
\sloppy

\setlength{\mathindent}{0pt}
\setlength{\parindent}{0pt}
\setlength{\parskip}{8pt}
\begin{document}

Regresyon, Ridge, Lasso, Capraz Saglama, Regularize Etmek

Konumuz regresyon cesitleri, ve ornek veri olarak diyabet hastaligi olan
kisilerden alinmis bazi temel verilerle hastaligin bir sene sonraki
ilerleme miktari kullanilacak. Regresyon sayesinde temel veriler ile
hastaligin ilerlemesi arasinda bir baglanti bulunabilir, bu sayede hem
veri aciklanir / daha iyi anlasilir (hangi degisken onemlidir, hangisi
degildir), hem de baska bir hastanin temel verilerini kullanarak o
hastanin diyabetinin bir sene sonra ne olacagini tahmin etmek mumkun
olur. Kullanilan temel veriler kisinin yasi, cinsiyeti, vucut kutle
endeksi (body mass index) ortalama tansiyonu ve alti kere alinmis kan
serum olcumleridir.

\begin{codecell}
\begin{codeinput}
\begin{lstlisting}
from pandas import *
diabetes = read_csv("diabetes.csv",sep=';')
diabetes_y = diabetes['response']
diabetes_x = diabetes.drop("response",axis=1)
diabetes_x_train = diabetes_x[:-20]
diabetes_x_test  = diabetes_x[-20:]
diabetes_y_train = diabetes_y[:-20]
diabetes_y_test  = diabetes_y[-20:]
\end{lstlisting}
\end{codeinput}
\end{codecell}
Ilk once basit regresyonu hatirlayalim (ordinary least regression). Bu
teknigi daha once pek cok yonden gorduk. \emph{Lineer Cebir}, \emph{Cok
Degiskenli Calculus} ders notlarinda, ve bizim \emph{Uygulamali
Matematik} yazilarinin hepsinde bu teknigin turetilmesi mevcut. Formul
\[ \hat{w} = (X^TX) ^{-1} X^{T}y \]
Numerik olarak hemen bu hesabi yapabiliriz. Bir hatirlatma: veri setine
y ekseninin nerede kesildiginin bulunabilmesi icin suni bir ekstra
`intercept' adli kolon ekleyecegiz, bu kolon iki boyutta $y=ax+c$
formulundeki $c$'nin bulunabilmesi icindir. Pandas ile bu ekstra kolonu
eklemek cok basit, ismen mevcut olmayan kolon erisildigi anda o kolon
hemen yoktan yaratilir.

\begin{codecell}
\begin{codeinput}
\begin{lstlisting}
import numpy.linalg as la
x_tmp = diabetes_x_train.copy()
x_tmp['intercept'] = 1
xTx = np.dot(x_tmp.T,x_tmp )
ws = np.dot(la.inv(xTx),np.dot(x_tmp.T,diabetes_y_train))
ws
\end{lstlisting}
\end{codeinput}
\begin{codeoutput}
\begin{verbatim}
array([  3.03499452e-01,  -2.37639315e+02,   5.10530605e+02,
         3.27736981e+02,  -8.14131711e+02,   4.92814589e+02,
         1.02848453e+02,   1.84606489e+02,   7.43519617e+02,
         7.60951724e+01,   1.52764307e+02])
\end{verbatim}
\end{codeoutput}
\end{codecell}
Ayni hesabi bir de scikit-learn paketini kullanarak yapalim. Bu paketin
LinearRegression cagrisi intercept isini otomatik olarak hallediyor,
eger intercept olmasin isteseydik, fit\_intercept=False diyecektik.

\begin{codecell}
\begin{codeinput}
\begin{lstlisting}
from sklearn import linear_model, cross_validation
lin = linear_model.LinearRegression()
lin.fit(diabetes_x_train, diabetes_y_train)
print lin.coef_
print "score", lin.score(diabetes_x_test, diabetes_y_test), 
\end{lstlisting}
\end{codeinput}
\begin{codeoutput}
\begin{verbatim}
[  3.03499452e-01  -2.37639315e+02   5.10530605e+02   3.27736981e+02
  -8.14131711e+02   4.92814589e+02   1.02848453e+02   1.84606489e+02
   7.43519617e+02   7.60951724e+01]
score 0.585075302278
\end{verbatim}
\end{codeoutput}
\end{codecell}
Sonuclar birbirine oldukca yakin. Simdi diger tekniklere gelelim.

Sırt Regresyonu (Ridge Regression)

Klasik regresyon ile
\[ \hat{w} = \arg \min_w ||y-Xw||^2  \]
problemini cozdugumuzu biliyoruz, ki $||\cdot||^2$ Oklit normunun
karesini temsil ediyor. Fakat bazi durumlarda $X^TX$'in tekil (singular)
olmasi mumkun ki boyle bir durumda $(X^TX)^{-1}$'in tersini almamiz
mumkun olmazdi. Tekillik ne zaman ortaya cikar? Eger elimizde veri
noktasindan daha fazla boyut var ise mesela\ldots{} Diyelim ki veri
olarak 10 tane kolon var, ama sadece 9 tane veri satiri. Sirt
Regregyonunun cikis noktasi budur.

Fakat ek olarak bu teknik kestirme hesaplarimiza (estimation) bir meyil
/ yanlilik (bias) eklemek icin de kullanilabilir ve bu meyil tahminlerin
/ kestirme hesaplarin iyilesmesine faydali olabilir.

Meyili nasil ekleriz? Diyelim ki bizim tanimlayacagimiz bir $\lambda$
ile tum $ws$'lerin toplamina bir ust sinir tanimlayabiliriz. Boylelikle
regresyonun bulacagi katsayilarin cok fazla buyumesine bir ``ceza''
getirmis olacagiz, ve bu cezayi iceren regresyon hesabi o cezadan
kacinmak icin mecburen bulacagi katsayilari ufak tutacak, hatta
bazilarini sifira indirebilecek. Bu azaltmaya istatistikte kuculme
(shrinkage) ismi veriliyor.

Sirt regresyonu icin bu kucultme soyle
\[ \hat{w}_{sirt} = \arg \min_w ( ||y-Xw||^2 + \lambda||w||^2 )  \]
Goruldugu uzere $w$'nin buyuklugunu, bir $\lambda$ katsayisi uzerinden
minimizasyon problemine dahil ettik, boylece diger parametreler ile
buyukluk te minimize edilecek. Ustteki tanim siniri tanimlanmamis
(unconstrained) bir optimizasyon problemidir. Sinirli olarak
\[ \min_w ||y-Xw||^2  \]\[ \textit{Su kosula gore (subject to) }||w||^2 \le \tau \]
ki $\lambda$ Lagrange carpanidir. Aslinda simdiye kadar ustteki cevrimin
tersini gorduk cogunlukla (yani sinirli problemden sinirsiza gitmeyi),
bu gidis tarzini gormek te iyi oldu.

Neyse bastaki sinirsiz problemi cozmek icin ifadenin gradyanini alalim,
\[ \nabla \big( ||y-Xw||^2 + \lambda||w||^2 \big) \]\[ \nabla \big( (y-Xw)^T (y-Xw) + \lambda w^Tw \big) \]\[ \nabla \big(  (y^T-w^TX^T)(y-Xw) + \lambda w^Tw  \big) \]\[ \nabla ( y^Ty - y^TXw - w^TX^Ty + w^TX^TXw + \lambda w^Tw   )  \]\[  - y^TX - X^Ty + 2X^TXw + 2\lambda w   \]\[  - 2X^Ty + 2X^TXw + 2\lambda w   \]\[   2X^TXw + 2\lambda w - 2X^Ty   \]\[   2(X^TX + \lambda I ) w - 2X^Ty   \]
Minimizasyon icin ustteki ifadeyi sifira esitleyebiliriz
\[   2(X^TX + \lambda I ) w - 2X^Ty  = 0 \]
O zaman
\[   (X^TX + \lambda I ) w = X^Ty  \]\[   \hat{w} = (X^TX + \lambda I)^{-1} X^Ty  \]
Bu son ifade en az kareler (least squares) yani normal regresyon cozum
formulune cok benziyor, sadece ek olarak bir $\lambda I$ toplama islemi
var. Demek ki sirt regresyonunu kullanmak icin zaten yaptigimiz hesaba,
zaten bizim kendimizin karar verdigi bir $\lambda$ uzerinden $\lambda I$
eklersek, geri kalan tum islemler ayni olacak.

Kontrol edelim

\begin{codecell}
\begin{codeinput}
\begin{lstlisting}
lam = 0.2
wridge = np.dot(la.inv(xTx+lam*np.eye(xTx.shape[0])),\
                np.dot(x_tmp.T,diabetes_y_train))
wridge
\end{lstlisting}
\end{codeinput}
\begin{codeoutput}
\begin{verbatim}
array([  16.70807829, -179.42288145,  447.64999897,  285.41866481,
        -51.7991733 ,  -75.09876191, -192.46341288,  123.61066573,
        387.91385823,  105.53294479,  152.7637018 ])
\end{verbatim}
\end{codeoutput}
\end{codecell}
Simdi scikit-learn ile ayni hesabi yapalim

\begin{codecell}
\begin{codeinput}
\begin{lstlisting}
ridge = linear_model.Ridge(alpha=0.2)
ridge.fit(diabetes_x_train, diabetes_y_train) 
print ridge.score(diabetes_x_test, diabetes_y_test), ridge.coef_
\end{lstlisting}
\end{codeinput}
\begin{codeoutput}
\begin{verbatim}
0.553680030106 [  16.69330211 -179.414259    447.63706059  285.40960442  -51.79094255
  -75.08327488 -192.45037659  123.60400024  387.91106403  105.55514774]
\end{verbatim}
\end{codeoutput}
\end{codecell}
Bir yontem daha var, bu yonteme Lasso ismi veriliyor. Lasso'ya gore
cezalandirma
\[ \sum_{k=1}^{n} w_k^2 \le \lambda \]
uzerinden olur. Bu yontemin tum detaylarina simdilik inmeyecegiz.

Ornek olarak bir $\lambda$ ile onun buldugu katsayilara bakalim.

\begin{codecell}
\begin{codeinput}
\begin{lstlisting}
lasso = linear_model.Lasso(alpha=0.3)
lasso.fit(diabetes_x_train, diabetes_y_train)
lasso.coef_
\end{lstlisting}
\end{codeinput}
\begin{codeoutput}
\begin{verbatim}
array([   0.        ,   -0.        ,  497.3407568 ,  199.17441037,
         -0.        ,   -0.        , -118.89291549,    0.        ,
        430.93795945,    0.        ])
\end{verbatim}
\end{codeoutput}
\end{codecell}
Lasso bazi katsayilari sifira indirdi! Bu katsayilarin agirlik verdigi
degiskenleri, eger Lasso'ya inanirsak, modelden tamamen atmak mumkundur.

Bu arada Sirt ve Lasso yontemlerinin metotlarina ``regularize etmek
(regularization)'' ismi de veriliyor.

k-Katlamali Capraz Saglama (k-fold Cross-Validation)

Bir yapay ogrenim algoritmasini kullanmadan once veriyi iki parcaya
ayirmak ise yarar; bu parcalar tipik olarak egitim verisi (training set)
digeri ise test verisi (validation set) olarak isimlendirilir.
Isimlerden belli olacagi uzere, algoritma egitim seti uzerinde egitilir;
ve basarisi test verisi uzerinden rapor edilir. Bir bakima modelin
olusturulmasi bir set uzerindendir, sonra ``al simdi hic gormedigin bir
veri seti, bakalim ne yapacaksin'' sorusunun cevabi, saglamasi bu
sekilde yapilir.

k-Katlamali Capraz Saglama bu iki parcali egitim / test kavramini bir
adim oteye tasir. Ufak bir k seceriz, ki bu genellikle 5 ila 10 arasinda
bir sayi olur, ve tum verimizi rasgele bir sekilde ama k tane ve esit
buyuklukte olacak sekilde parcalara ayiririz. Bu parcalara ``katlar
(folds)'' ismi verilir bazen (ki isim buradan geliyor). Sonra teker
teker her parcayi test verisi yapariz ve geri kalan tum parcalari egitim
verisi olarak kullaniriz. Bu islemi tum parcalar icin tekrarlariz.

Bu yaklasim niye faydalidir? Cunku veriyi rasgele sekillerde bolup, pek
cok yonden egitim / test icin kullaninca verinin herhangi bir sekilde
bizi yonlendirmesi / aldatmasi daha az mumkun hale gelir.

Ve iste bu ozelligi, ek olarak, capraz saglamayi ``model secmek'' icin
vazgecilmez bir arac haline getirir.

Model secmek nedir? Model secimi ustteki baglamda optimal bir $\lambda$
bulmaktir mesela, yani her modeli temsil eden bir $\lambda$ var ise, en
iyi $\lambda$'yi bulmak, en iyi modeli bulmak anlamina geliyor, capraz
saglama bunu sagliyor. Capraz saglama icin scikit-learn'un sagladigi
fonksiyonlar vardir, once katlari tanimlariz, sonra bu degistirilmis
regresyon fonksiyonlarina katlama usulunu geceriz.

\begin{codecell}
\begin{codeinput}
\begin{lstlisting}
k_fold = cross_validation.KFold(n=420, n_folds=7)
\end{lstlisting}
\end{codeinput}
\end{codecell}
Katlari ustteki gibi tanimladik. 420 tane veri noktasini 7 kata bol
dedik. Simdi bu katlari kullanalim,

\begin{codecell}
\begin{codeinput}
\begin{lstlisting}
ridge_cv = linear_model.RidgeCV(cv=k_fold)
ridge_cv.fit(np.array(diabetes_x), np.array(diabetes_y))
ridge_cv.alpha_
\end{lstlisting}
\end{codeinput}
\begin{codeoutput}
\begin{verbatim}
0.10000000000000001
\end{verbatim}
\end{codeoutput}
\end{codecell}
Ustteki sonuc $\lambda = 0.1$'i gosteriyor. Bu $\lambda$ daha optimalmis
demek ki. Lasso icin benzer sekilde

\begin{codecell}
\begin{codeinput}
\begin{lstlisting}
lasso_cv = linear_model.LassoCV(cv=k_fold)
lasso_cv.fit(diabetes_x, diabetes_y)
\end{lstlisting}
\end{codeinput}
\begin{codeoutput}
\begin{verbatim}
LassoCV(alphas=None, copy_X=True,
    cv=sklearn.cross_validation.KFold(n=420, n_folds=7), eps=0.001,
    fit_intercept=True, max_iter=1000, n_alphas=100, normalize=False,
    precompute='auto', tol=0.0001, verbose=False)
\end{verbatim}
\end{codeoutput}
\end{codecell}
\begin{codecell}
\begin{codeinput}
\begin{lstlisting}
lasso_cv.alpha_
\end{lstlisting}
\end{codeinput}
\begin{codeoutput}
\begin{verbatim}
0.0028395871911771854
\end{verbatim}
\end{codeoutput}
\end{codecell}
\begin{codecell}
\begin{codeinput}
\begin{lstlisting}
lasso_cv.score(diabetes_x_test, diabetes_y_test) 
\end{lstlisting}
\end{codeinput}
\begin{codeoutput}
\begin{verbatim}
0.59709035088351892
\end{verbatim}
\end{codeoutput}
\end{codecell}
Simdi veri setinin bir kismi uzerinde teker teker hangi algoritmanin
daha basarili oldugunu gorelim.

\begin{codecell}
\begin{codeinput}
\begin{lstlisting}
def predict(row):
    j = row; i = row-1
    new_data = diabetes_x[i:j]
    print diabetes_y[i:j], "lasso",lasso_cv.predict(new_data), \
    	    "ridge",ridge_cv.predict(new_data), \
      	    "linear",lin.predict(new_data)	    

predict(-2) # sondan ikinci veri satiri
predict(-3)
predict(-4)
predict(-5)
predict(-8)

\end{lstlisting}
\end{codeinput}
\begin{codeoutput}
\begin{verbatim}
439    132
Name: response lasso [ 122.23614365] ridge [ 127.1821212] linear [ 123.56604986]
438    104
Name: response lasso [ 101.85154941] ridge [ 108.89678818] linear [ 102.5713971]
437    178
Name: response lasso [ 192.95669774] ridge [ 189.58095011] linear [ 194.03798086]
436    48
Name: response lasso [ 52.89039744] ridge [ 57.66611598] linear [ 52.5445869]
433    72
Name: response lasso [ 60.42852011] ridge [ 66.3661042] linear [ 61.19831285]
\end{verbatim}
\end{codeoutput}
\end{codecell}
Ustteki sonuclara gore gercek degeri 132 olan 439. satirda lasso 122.2,
sirt (ridge) 127.1, basit regresyon ise 123.5 bulmus. O veri noktasi
icin sirt yontemi daha basarili cikti.

Sonuclara bakinca bazen sirt, bazen normal regresyon basarili cikiyor.
Hangi yontem kazanmis o zaman? Bir o, o bir bu ondeyse, hangi yontemi
kullanacagimizi nasil bilecegiz?

Aslinda her seferinde tek bir metotu kullanmak gerekmiyor. Bu metotlari
bir takim (ensemble) halinde isletebiliriz. Her test noktasini, her
seferinde tum metotlara sorariz, gelen sonuclarin mesela.. ortalamasini
aliriz. Bu sekilde tek basina isleyen tum metotlardan tutarli olarak her
seferinde daha iyi sonuca ulasacak bir sonuc elde edebiliriz. Zaten
Kaggle gibi yarismalarda cogunlukla birinciligi kazanan metotlar bu
turden takim yontemlerini kullanan metotlar, mesela Netflix yarismasini
kNN ve SVD metotlarini takim halinde isleten bir grup kazandi.

Kaynaklar

http://www.lx.it.pt/\ensuremath{\sim}mtf/Figueiredo\_Linear\_Regression.pdf

http://www.cs.nyu.edu/\ensuremath{\sim}mohri/mls/lecture\_8.pdf

Harrington, P., \emph{Machine Learning in Action}

Shalizi, C., Data Analysis from an Elementary Point of View

\end{document}
