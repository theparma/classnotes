\documentclass[12pt,fleqn]{article}\usepackage{../common}
\begin{document}
Bazi $e$, Log Ozellikleri

Bazen $\log$ bazen $\ln$ kullanildigini gorursunuz. Eger $\ln$
kullanilirsa bu $\log$ ifadesinin $e$ baz kullanan hali demektir, yani
$\log_e$. 

Carpim log'u, loglarin toplami olur.

\[ \log xy = \log x + \log y \]

Bolum log'u, loglarin cikartilmasi olur.

\[ \log x/y = \log x - \log y  
\mlabel{1}
\]

Ayni baz kullanan carpimlar, tek baza indirilip ustleri toplanabilir.

\[ e^a \cdot e^b = e^{a+b} \]

Ustu olan bir ifadenin log'u, ust degerini asagi indirir.

\[ \log(x^y) = y \log(x) \]

Ustun ustu (son ust tum bazi kapsayacak sekilde ise) direk ustlerin
carpimina cevirilebilir.

\[ (e^x)^y = e^{xy} \]

$\ln$'nin bazi $e$ olduguna gore, $e$ uzeri $\ln$ birbirini iptal eder,
yani

\[ x = e^{\ln x} \]

Bu ifade (1)'den turetilebilir ama yine de ayri vermek iyi olur, $1/x$'in
log'u $x$'in negatifini verir. 

\[ \log(1/x) = -\log(x) \]


\end{document}
