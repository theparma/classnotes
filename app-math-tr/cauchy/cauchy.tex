\documentclass[12pt,fleqn]{article}
\setlength{\parindent}{0pt}
\usepackage{graphicx}
\usepackage{cancel}
\usepackage{listings}
\usepackage[latin5]{inputenc}
\usepackage{color}
\setlength{\parskip}{8pt}
\setlength{\parsep}{0pt}
\setlength{\headsep}{0pt}
\setlength{\topskip}{0pt}
\setlength{\topmargin}{0pt}
\setlength{\topsep}{0pt}
\setlength{\partopsep}{0pt}
\setlength{\mathindent}{0cm}
\usepackage{latexsym}
\usepackage{showkeys}
\renewcommand*\showkeyslabelformat[1]{(#1)}

\begin{document}
Cauchy Ortalama Deger Teorisi (Cauchy Mean-value Theorem)

Teori soyle

Eger $f,g$ fonksiyonlari $[a,b]$ araliginda surekli ise ve $g'(x) \ne 0$
farz edildigi durumda $[a,b]$ arasinda oyle bir $c$ vardir ki,

\[ \frac{f'(c)}{g'(c)} = \frac{f(b)-f(a)}{g(b)-g(a)} \]

ifadesi dogrudur. 

Ispat

Simdi daha onceden gordugumuz Ortalama Deger Teorisi'ni (Cauchy olmayan)
iki kere kullanacagiz. Teoriyi once $g(a) \ne g(b)$ oldugunu gostermek icin
kullanacagiz. Cunku eger bu dogru olsaydi, Ortalama Deger Teorisi 

\[ g'(c) = \frac{g(b) - g(a)}{b-a} = 0\]

olurdu, ki bu $[a,b]$ arasindaki bi4 $c$ icin basta yaptigimiz faraziyemiz
$g'(x) \ne 0$ ile ters duserdi. 

Ikinci kullanim: $F(x)$ adinda, $f,g$ fonksiyonlarini kullanan baska bir
fonksiyon kurgulayalim.

\[ F(x) = f(x) - f(a) - \frac{f(b)-f(a) }{g(b)-g(a)}[g(x)-g(a)] \]

Bu fonksiyonun turevi, $f,g$'nin turevi alinabildigi her yerde alinabilir
olur. Ayrica $F(b) = F(a) = 0$. $a,b$ degerlerini yerine koyarsak bunu
gorebiliriz, mesela $x=a$ icin

\[ F(a) = \cancelto{0}{f(a) - f(a)} -
\frac{f(b)-f(a) }{g(b)-g(a)}
[\cancelto{0}{g(a)-g(a)}] 
\]

\[  = 0 - 0 = 0 \]

O zaman, $F(b) = F(a) = 0$'dan bir sonuca daha erisiriz. Bir fonksiyon
$a,b$ uclarinda sifir ise, bu fonksiyon bir sekilde azalip, cogaliyor, ya
da cogalip azaliyor demektir, yani kesinlikle bir yerde tepe yapiyor
demektir. Tepe yapmanin Calculus'taki tercumesi $[a,b]$ arasindaki bir $c$
icin $F'(c)=0$ olmasidir. O zaman ustteki $F(x)$'in turevini alirsak, ve
$x=c$ dersek, 

\[ F'(c) = f'(c) - \frac{f(b)-f(a)}{g(b)-g(a)}[g'(c)] = 0\]

dogru olmalidir. Turev alirken $f(a)$ yokoldu cunku sabitti, buyuk bolum
yerinde kaldi cunku tamami $g(x)$ icin katsayi. Eger tekrar duzenlersek,
negatif terimi sola alirsak, ve iki tarafi $g'(c)$'ye bolersek,

\[ \frac{f'(c)}{g'(c)} = \frac{f(b)-f(a)}{g(b)-g(a)} \]

ifadesini elde ederiz. Yani bastaki teoriyi elde etmis oluruz. 

Ortalama Deger Teorisini ilk kez kullanmamizin sebebi, ustteki bolenin sifir
olmamasini istedigimiz icindi, cunku sifirla bolum tanimsizdir. 



Kaynaklar 

[1] Thomas Calculus, 11. Baski sf. 294


\end{document}
