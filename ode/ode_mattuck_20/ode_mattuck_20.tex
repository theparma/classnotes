\documentclass[12pt,fleqn]{article}
\setlength{\parindent}{0pt}
\usepackage{graphicx}
\usepackage{cancel}
\usepackage{listings}
\usepackage[latin5]{inputenc}
\usepackage{color}
\setlength{\parskip}{8pt}
\setlength{\parsep}{0pt}
\setlength{\headsep}{0pt}
\setlength{\topskip}{0pt}
\setlength{\topmargin}{0pt}
\setlength{\topsep}{0pt}
\setlength{\partopsep}{0pt}
\setlength{\mathindent}{0cm}
\usepackage{latexsym}
\usepackage{showkeys}
\renewcommand*\showkeyslabelformat[1]{(#1)}

\begin{document}
Ders 20

Konumuz Laplace Transformu ile diferansiyel denklem cozmek. Fakat onu
yapmadan once, Laplace Transformunun mumkun oldugundan emin olmamiz
gerekiyor. Bazilariniz dusunebilir, ``ama hocam alttaki formul her zaman
hesaplanaz mi?''

\[ F(s) = \int_0^{\infty} f(t)e^{-st} \ dt \]

Cevap hayir cunku ustteki bir uygunsuz (improper) entegral, ust sinir
sonsuzluga gidiyor ve bildigimiz gibi uygunsuz entegraller her zaman bir
degere yaklasmiyorlar (converge). 

Laplace tranformunun mumkun olmasinin kontrolu, entegre edilen $f(t)$'nin
``cok hizli buyumemesi''yle alakali. Fonksiyonun buyuyebilir tabii,, ama
cok hizli buyurse o zaman $e^{-st}$ ile carpilmak onu asagi cekemez, ve
entegralin tamami bir degere yaklasamaz. O zaman $f(t)$'nin olmasi
gerektigi sarti ``ustel tipte (exponential type)'' olarak tarif
edebiliriz. Yani 

\[ |f(t)| \le C e^{kt}, \ \textit{ sabit } C > 0, \forall t \ge 0 
\textit{ herhangi bir } k > 0
\]

$f(t)$'nin kesin (absolute) degerini kullandik, fonksiyonun eksi yonde mi,
arti yonde mi oldugu onemli degil, onemli olan buyuyus (ya da kuculus)
hizi. Bu arada, ustteki tanimda bir suru dehset hizla buyuyen fonksiyon
mumkun, mesela $e^{100t}$'yi dusunelim; bu fonksiyonun nasil buyudugunu
goren var mi? Roket gibi yukari firlar, o kadar hizli bir buyumeden
bahsediyoruz. Yani ustteki sart genis bir yelpazedeki fonksiyonlari
kapsayabilir, cok kisitlayici sayilmaz. Bazi ornekler gorelim.

$sin(t)$ ustel tipte midir? Evet, cunku $|sin(t)| \le 1$ daha dogrusu
$|sin(t)| \le 1 \cdot e^{0t}$. 

Ya $t^n$? Bu fonksiyon da roket gibi yukari firlar ama acaba
karsilastirilacak ustel fonksiyonda $k$'yi yeterince buyuk yaparsak onun
ustesinden gelebilir miyiz? Sasirtici gelebilir ama buna gerek yok, $k=1$
bu isi hallediyor. 

\[ t^n \le M e^t, \ \textit{ M herhangi bir sabit }, \forall t > 0 \]

Yani bu ustel tip bir fonksiyon. Bunun tabii ki boyle olacagini tahmin
edebilirdik, gecen derste $t^n$'nin Laplace tranformunu hesapladik ne de
olsa (Laplace tranformu yapabilmek teoride her zaman ustel tip olmasi
anlamina gelmiyor aslinda, ama pratikte her zaman oyle), neyse, ama ustteki
karsilastirma nasil isledi? Su hesabi yapalim

\[ \frac{t^n}{e^t} \]

Iddia ediyorum ki ustteki ifade bir $M$ sabiti ile ``sinirlanmistir
(bounded by)'', yani hep $M$'in altindadir. Kontrol etmek icin yani

\[ \frac{t^n}{e^t} \le  M\]

sorusunu kontrol ederiz. Ya da onunla direk ilgili bir soru  $t \to \infty$ iken

\[ \frac{t^n}{e^t} \to ?\]

sorusunu. Ustteki bolum sifira yaklasir. Niye? Yine onceki dersten,
L'Hospital Kurali sebebiyle! Bolumun sifira gitmesi ne anlama gelir? 

\includegraphics[height=2cm]{20_1.png}

Bolum sifirda baslamistir, ve sifira gider. Bu demektir ki arada bir yerde
muhakkak bir maksimum degere (max value) sahiptir, bu noktaya $M$ dersek,
bolum $M$ tarafindan sinirlandirilmistir diyebiliriz.

























\end{document}
